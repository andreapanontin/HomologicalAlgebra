\documentclass[../Main]{subfiles}
\begin{document}
\section{Complexes}
In the following let $\mathsf{C}$ be an {\em abelian} category.

\begin{defn}[(Cochain) complex in an abelian category]
	A complex in $\mathsf{C}$ is a sequence
	\begin{equation}
	\begin{tikzcd}
		\ldots \arrow[r, "", rightarrow] &
		A^{n-1} \arrow[r, "d_A^{n-1}", rightarrow] &
		A^n \arrow[r, "d_A^n", rightarrow] &
		A^{n+1} \arrow[r, "d_A^{n+1}", rightarrow] &
		\ldots
	\end{tikzcd}
	\end{equation} 
	s.t. $A^{n} \in \mathrm{Ob} \left(\mathsf{C}\right)$
	and $d^{n+1} \circ d^n = 0$ for all $n \in \mathbb{Z}$.
	We will denote the complex by $\left(A^\bullet, d_A\right)$.
\end{defn}

\begin{defn}[Morphism of complexes]
	Let $\left(A^\bullet, d_A\right)$ and $\left( B^{\bullet}, d_{B} \right)$
	be two complexes in $\mathsf{C}$.
	A morphism between the complexes,
	\begin{equation}
		\begin{tikzcd}
			f^\bullet\colon A^\bullet \arrow[r, "", rightarrow] &
			B^\bullet
		\end{tikzcd}
	\end{equation} 
	is a family of morphisms
	\begin{equation}
	\begin{tikzcd}
		f^n: A^n \arrow[r, "", rightarrow] &
		B^n
	\end{tikzcd}
	\end{equation} 
	s.t. for any $n \in \mathbb{Z}$ the square commutes
	\begin{equation}
	\begin{tikzcd}
		\ldots \arrow[r, "", rightarrow] &
		A^n \arrow[r, "d^n_A", rightarrow] \arrow[d, "f^n", rightarrow] &
		A^{n+1} \arrow[d, "f^{n+1}", rightarrow] \arrow[r, "", rightarrow] &
		\ldots\\
		\ldots \arrow[r, "", rightarrow] &
		B^n \arrow[r, "d^n_B"', rightarrow] &
		B^{n+1} \arrow[r, "", rightarrow] &
		\ldots
	\end{tikzcd}
	.\end{equation} 	
\end{defn}

\begin{defn}[Category of complexes in $\mathsf{C}$]
	We denote the category whose objects are complexes and morphism
	are morphism of complexes, as just defined, by $\mathrm{Ch}(\mathsf{C})$
	and call it the category of complexes in $\mathsf{C}$.
\end{defn}

\begin{exr}
Show that the category $\mathrm{Ch}(\mathsf{C})$ is abelian.
\end{exr} 
\begin{exr}\leavevmode\vspace{-.2\baselineskip}
	\begin{enumerate}
		\item In particular, given $f^\bullet\colon A^\bullet \to B^\bullet$
			any morphism of complexes, then $\ker f^\bullet$ is the complex
			\begin{equation}
			\begin{tikzcd}
				\ldots \arrow[r, "", rightarrow] &
				\ker f^{n-1} \arrow[r, "d^{n-1}", rightarrow] &
				\ker f^n \arrow[r, "d^n", rightarrow] &
				\ker f^{n+1} \arrow[r, "d^{n+1}", rightarrow] &
				\ldots
			\end{tikzcd}
			,\end{equation} 
			where $d^n$ are the maps induced by $d^n_A$.
		\item Analogously show that the cokernel of $f$ is the complex
			\begin{equation}
			\begin{tikzcd}
				\ldots \arrow[r, "", rightarrow] &
				\coker f^{n-1} \arrow[r, "d^{n-1}", rightarrow] &
				\coker f^n \arrow[r, "d^n", rightarrow] &
				\coker f^{n+1} \arrow[r, "d^{n+1}", rightarrow] &
				\ldots
			\end{tikzcd}
			,\end{equation} 
			where $d^n$ is induced by $d^n_B$.
	\end{enumerate}
\end{exr} 

\begin{rem}[]
	Recall that condition $d^{n+1} \circ d^n = 0$ induces
	a map
	\begin{equation}
	\begin{tikzcd}
		\ima d^n \arrow[r, "", rightarrowtail] &
		\ker d^{n+1}
	\end{tikzcd}
	.\end{equation} 
\end{rem}

\begin{defn}[Cohomology of the complex]
	Fixed a complex $\left( A^{\bullet}, d_{A} \right)$,
	from the above map, for every $n \in \mathbb{Z}$, one denotes by
	\begin{equation}
		H^n(A^\bullet) := \coker \left( \ima d^{n-1}
		\hookrightarrow \ker d^{n}\right) =:
		\ker d^{n} / \ima d^{n-1}
	.\end{equation} 
	(The last notation can be viewed as purely formal, but it recalls
	the intuition developed in $\mathsf{Ab}$).
	We call $H^n(A^\bullet)$ the $n$-th cohomology of $\left( A^{\bullet}, d_{A} \right)$.
\end{defn}

\begin{exr}[Functoriality of cohomology]
	Any morphism of complexes $f^\bullet\colon A^\bullet \to B^\bullet$
	induces, for every index $n$, a morphism at the level of cohomologies:
	\begin{equation}
	\begin{tikzcd}
		H^n(f^\bullet) \colon H^n(A^\bullet) \arrow[r, "", rightarrow] &
		H^n(B^\bullet)
	\end{tikzcd}
	\end{equation} 
	as a morphism in $\mathsf{C}$.
	Then, for any $n \in \mathbb{Z}$, we have defined a functor
	\begin{align}
		H^n\colon \mathrm{Ch}(\mathsf{C}) &\longrightarrow \mathsf{C} \\
		A^\bullet &\longmapsto H^n(A^\bullet) \nonumber \\
		f^\bullet &\longmapsto H^n(f^\bullet) \nonumber
	.\end{align} 
	Moreover this functor is additive.
\end{exr} 

\begin{lem}[ker-coker sequence]\label{lem:kerCokerSequence}
	Consider the following in $\mathsf{C}$:
	\begin{equation}
	\begin{tikzcd}
		X \arrow[r, "f", rightarrow] &
		Y \arrow[r, "g", rightarrow] &
		Z
	\end{tikzcd}
	.\end{equation} 
	We have an exact sequence in $\mathsf{C}$:
	\begin{equation}
	\begin{tikzcd}
		0 \arrow[r, "", rightarrow] &
		\ker f \arrow[r, "", rightarrow] &
		\ker g \circ f \arrow[r, "", rightarrow] 
		\arrow[d, phantom, ""{coordinate, name=Z}] &
		\ker g   
		\arrow[dll, "", rounded corners, bend left = 100,
		to path={ -- ([xshift=2ex]\tikztostart.east)
			|- (Z) [near end]\tikztonodes
			-| ([xshift=-2ex]\tikztotarget.west)
			-- (\tikztotarget)}] & \\
		&
		\coker f \arrow[r, "", rightarrow] &
		\coker g \circ f \arrow[r, "", rightarrow] &
		\coker g \arrow[r, "", rightarrow] &
		0
	\end{tikzcd}
	.\end{equation} 
\end{lem} 
\begin{proof}
	From the composition one obtains the diagram
	\begin{equation}
	\begin{tikzcd}
		&
		A \arrow[r, "f", rightarrow] 
		\arrow[d, "g \circ f", rightarrow] &
		Y \arrow[d, "g", rightarrow] \arrow[r, "", rightarrow] &
		\coker f \arrow[r, "", rightarrow] \arrow[d, "", rightarrow] &
		0\\
		0 \arrow[r, "", rightarrow] &
		Z \arrow[r, "", equal] &
		Z \arrow[r, "", rightarrow] &
		0
	\end{tikzcd}
	.\end{equation} 
	Clearly the diagram commutes and has exact rows.
	Then one can apply the snake lemma.
	We are only left to prove that
	\begin{equation}
		\ker \left( \ker f \circ g \to \ker g \right) \simeq \ker f
	\end{equation} 
	in order to conclude the proof.
	This is left as an exercise to the reader.
\end{proof}

\begin{lem}
	Consider a short exact sequence in $\mathrm{Ch}(\mathsf{C})$
	\begin{equation}
	\begin{tikzcd}
		0 \arrow[r, "", rightarrow] &
		A^\bullet \arrow[r, "", rightarrow] &
		B^\bullet \arrow[r, "", rightarrow] &
		C^\bullet \arrow[r, "", rightarrow] &
		0
	\end{tikzcd}
	.\end{equation} 
	Then there exists, for all $n \in \mathbb{Z}$, a canonical map
	\begin{equation}
	\begin{tikzcd}
		\delta^n\colon H^{n} \left( C^\bullet \right) \arrow[r, "", rightarrow] &
		H^{n+1} \left( A^\bullet \right)
	\end{tikzcd}
	\end{equation} 
	s.t. the following is a long exact sequence in $\mathsf{C}$:
	\begin{equation}
	\begin{tikzcd}[row sep=tiny]
		\ldots \arrow[r, "", rightarrow] &
		H^{n} \left( A^\bullet \right) \arrow[r, "", rightarrow] &
		H^{n} \left( B^\bullet \right) \arrow[r, "", rightarrow] 
		\arrow[d, phantom, ""{coordinate, name=Z}] &
		H^{n} \left( C^\bullet \right) \arrow[r, "\delta^n"', phantom, rightarrow] 
		\arrow[dll, "", rounded corners, bend left = 100,
		to path={ -- ([xshift=2ex]\tikztostart.east)\tikztonodes
			|- (Z) [near end]
			-| ([xshift=-2ex]\tikztotarget.west)
			-- (\tikztotarget)}] & 
			\phantom{a} \\
		& 
		H^{n+1} \left( A^\bullet \right) \arrow[r, "", rightarrow] &
		H^{n+1} \left( B^\bullet \right) \arrow[r, "", rightarrow] &
		H^{n+1} \left( C^\bullet \right) \arrow[r, "", rightarrow] &
		\ldots
	\end{tikzcd}
	.\end{equation} 
	Moreover morphisms of short exact sequences of complexes induce morphisms
	of long exact sequences in $\mathsf{C}$.
	More explicitly, given any commutative diagram
	\begin{equation}
	\begin{tikzcd}
		0 \arrow[r, "", rightarrow] &
		A^\bullet \arrow[r, "", rightarrow] \arrow[d, "", rightarrow] &
		B^\bullet \arrow[r, "", rightarrow] \arrow[d, "", rightarrow] &
		C^\bullet \arrow[r, "", rightarrow] \arrow[d, "", rightarrow] &
		0\\
		0 \arrow[r, "", rightarrow] &
		('A)^\bullet \arrow[r, "", rightarrow] &
		('B)^\bullet \arrow[r, "", rightarrow] &
		('C)^\bullet \arrow[r, "", rightarrow] &
		0
	\end{tikzcd}
	\end{equation} 
	one obtains a commutative diagram, whose arrows are induced by the above:
	\begin{equation*}
	\begin{tikzcd}
		\ldots \arrow[r, "", rightarrow] &
		H^{n} \left( A^\bullet \right)\arrow[r, "", rightarrow] \arrow[d, "", rightarrow] &
		H^{n} \left( B^\bullet \right) \arrow[r, "", rightarrow] \arrow[d, "", rightarrow] &
		H^{n} \left( C^\bullet \right) \arrow[r, "\delta^n", rightarrow] \arrow[d, "", rightarrow] &
		H^{n+1} \left( A^\bullet \right) \arrow[d, "", rightarrow] 
		\arrow[r, "", rightarrow] &
		\ldots \\
		\ldots \arrow[r, "", rightarrow] &
		H^{n} \left( 'A^\bullet \right)\arrow[r, "", rightarrow] &
		H^{n} \left( 'B^\bullet \right) \arrow[r, "", rightarrow] &
		H^{n} \left( 'C^\bullet \right) \arrow[r, "'\delta^n"', rightarrow] &
		H^{n+1} \left( 'A^\bullet \right) \arrow[r, "", rightarrow] &
		\ldots
	\end{tikzcd}
	.\end{equation*} 
\end{lem} 
\begin{proof}
	For any $\left( A^{\bullet}, d_{A} \right) \in \mathrm{Ch}(\mathsf{X})$
	we denote by $Z^n(A^\bullet) := \ker d_A^n$,
	a subobject of $A^n$, i.e. we have a canonical inclusion
	\begin{equation}
	\begin{tikzcd}
		Z^n(A^\bullet) \arrow[r, "", hookrightarrow] &
		A^n
	\end{tikzcd}
	.\end{equation} 
	Analogously we define $I^n(A^\bullet) := \ima d^{n-1}$.
	Since we have a complex we also have an inclusion
	\begin{equation}
	\begin{tikzcd}
		I^n(A^\bullet) \arrow[r, "", hookrightarrow] &
		Z^n(A^\bullet) \arrow[r, "", hookrightarrow] &
		A^n
	\end{tikzcd}
	.\end{equation} 
	Then, by definition, we have the equality
	\begin{equation}
		H^{n} \left( A^\bullet \right) = Z^n(A^\bullet)/I^n(A^\bullet)
	.\end{equation} 
	In the following, when there will be no possibility of confusion,
	we will denote the above simply by $Z^n$ and $I^n$, omitting to specify the
	complex $A^\bullet$.
	Notice that we will only prove the first part of the lemma, since the last follows
	from the functoriality of snake lemma, the main ingredient of this proof.

	Since we have a morphism of short exact sequences of complexes,
	we obtain a morphism of short exact sequences in $\mathsf{C}$ for all $n \in \mathbb{Z}$.
	That induces the following commutative diagram, with exact rows in the middle.
	\begin{equation}
	\begin{tikzcd}
		0 \arrow[r, "", rightarrow] &
		Z^n (A^\bullet) \arrow[r, "", rightarrow] \arrow[d, "", rightarrow] &
		Z^n (B^\bullet) \arrow[r, "", rightarrow] \arrow[d, "", rightarrow] 
		\arrow[ddd, phantom, ""{coordinate, name=Z}] &
		Z^n (C^\bullet) \arrow[d, "", rightarrow] 
		\arrow[dddll, "", dashed, rounded corners, bend left = 100,
		to path={ -- ([xshift=2ex]\tikztostart.east)
			|- (Z) [near end]\tikztonodes
			-| ([xshift=-2ex]\tikztotarget.west)
			-- (\tikztotarget)}] & \\
		0 \arrow[r, "", rightarrow] &
		A^n \arrow[r, "", rightarrow] \arrow[d, "d^n_A" near start, rightarrow] &
		B^n \arrow[r, "", rightarrow] \arrow[d, "d^n_B", near start, rightarrow] &
		C^n \arrow[r, "", rightarrow] \arrow[d, "d^n_C", near start, rightarrow] &
		0\\
		0 \arrow[r, "", rightarrow] &
		A^{n+1} \arrow[r, "", rightarrow] \arrow[d, "", rightarrow] &
		B^{n+1} \arrow[r, "", rightarrow] \arrow[d, "", rightarrow] &
		C^{n+1} \arrow[r, "", rightarrow] \arrow[d, "", rightarrow] &
		0\\
		&
		A^{n+1}/I^{n+1}(A^\bullet) \arrow[r, "", rightarrow] &
		B^{n+1}/I^{n+1}(B^\bullet) \arrow[r, "", rightarrow] &
		C^{n+1}/I^{n+1}(C^\bullet) \arrow[r, "", rightarrow] &
		0\\
	\end{tikzcd}
	.\end{equation} 
	Then we apply snake lemma and obtain the following long exact sequence:
	\begin{equation}
	\begin{tikzcd}[row sep=tiny]
		0 \arrow[r, "", rightarrow] &
		Z^n(A^\bullet) \arrow[r, "", rightarrow] &
		Z^n(B^\bullet) \arrow[r, "", rightarrow]
		\arrow[d, phantom, ""{coordinate, name=Z}] &
		Z^n(C^\bullet) 
		\arrow[dlll, "", rounded corners, bend left = 100,
		to path={ -- ([xshift=2ex]\tikztostart.east)
			|- (Z) [near end]\tikztonodes
			-| ([xshift=-2ex]\tikztotarget.west)
			-- (\tikztotarget)}] \\
		A^{n+1}/I^{n+1} \arrow[r, "", rightarrow] &
		B^{n+1}/I^{n+1} \arrow[r, "", rightarrow] &
		C^{n+1}/I^{n+1} \arrow[r, "", rightarrow] &
		0.
	\end{tikzcd}
	\end{equation} 
	Since $d^{n+1}_A \circ d^n_A = 0$, we obtain that $d^n_A$ factors
	through $i^{n+1}\colon Z^{n+1} \hookrightarrow A^{n+1}$.
	Then we obtain the commutative
	\begin{equation}
	\begin{tikzcd}
		I^n \arrow[r, "i^n_A", hookrightarrow] \arrow[rr, "0", rightarrow, bend left] &
		A^n \arrow[r, "d^n_A", rightarrow] \arrow[d, "p_n^A"', twoheadrightarrow] 
		\arrow[rd, "\widetilde{d^n_A}", dashrightarrow] &
		A^{n+1} \arrow[r, "", rightarrow] &
		\ldots\\
						  &
		A^n/I^n \arrow[r, "\overline{d^n_A}"', rightarrow] &
		Z^{n+1}(A^\bullet) \arrow[u, "i^{n+1}"', hookrightarrow]  &
	\end{tikzcd}
	.\end{equation} 
	Then, since $i^{n+1} \circ \widetilde{d^n_A} \circ i^n_A = 
	d^n_A \circ i^n_A = 0$ and $i^{n+1}$ is a mono, we obtain that
	$\widetilde{d^n_A}$ factors through $\coker i^n_A \simeq \coker d^{n-1}_A = A^n/I^n$
	(by lemma 2.13, tk: fix the reference).
	Then, patching together the information from the snake lemma and the above
	construction, we obtain the following commutative diagram, with exact rows:
	\begin{equation}
	\begin{tikzcd}
		&
		A^n/I^n(A^\bullet) \arrow[r, "", rightarrow] \arrow[d, "\overline{d^n_A}", rightarrow] &
		B^n/I^n(B^\bullet) \arrow[r, "", rightarrow] \arrow[d, "\overline{d^n_B}", rightarrow] &
		C^n/I^n(C^\bullet) \arrow[r, "", rightarrow] \arrow[d, "\overline{d^n_C}", rightarrow] &
		0\\
		0 \arrow[r, "", rightarrow] &
		Z^{n+1}(A^\bullet) \arrow[r, "", rightarrow] &
		Z^{n+1}(A^\bullet) \arrow[r, "", rightarrow] &
		Z^{n+1}(C^\bullet) &
	\end{tikzcd}
	.\end{equation} 
	Now we claim that
	\begin{align}
		\ker \overline{d^n_A} &\simeq H^{n} \left( A^\bullet \right)\\
		\coker \overline{d^n_A} &\simeq H^{n+1} \left( A^\bullet \right)
	.\end{align} 
	(This will be proved after the theorem).
	Applying the snake lemma now grants the desired sequence:
	\begin{equation}
	\begin{tikzcd}[row sep=tiny]
		\ldots \arrow[r, "", rightarrow] &
		H^{n} \left( A^\bullet \right) \arrow[r, "", rightarrow] &
		H^{n} \left( B^\bullet \right) \arrow[r, "", rightarrow] 
		\arrow[d, phantom, ""{coordinate, name=Z}] &
		H^{n} \left( C^\bullet \right) \arrow[r, "\delta^n"', phantom, rightarrow] 
		\arrow[dll, "", rounded corners, bend left = 100,
		to path={ -- ([xshift=2ex]\tikztostart.east)\tikztonodes
			|- (Z) [near end]
			-| ([xshift=-2ex]\tikztotarget.west)
			-- (\tikztotarget)}] & 
			\phantom{a} \\
		& 
		H^{n+1} \left( A^\bullet \right) \arrow[r, "", rightarrow] &
		H^{n+1} \left( B^\bullet \right) \arrow[r, "", rightarrow] &
		H^{n+1} \left( C^\bullet \right) \arrow[r, "", rightarrow] &
		\ldots
	\end{tikzcd}
	.\end{equation} 
	Then functoriality follows from functoriality of snake lemma
	and functoriality of the claim, which is proved thanks to the following lemma.
\end{proof}

\begin{proof}[Proof of the claim]
	We have the commutative diagram
	(as seen before)
	\begin{equation}
	\begin{tikzcd}[column sep=tiny]
		A^n \arrow[rr, "\widetilde{d^n_A}", rightarrow] 
		\arrow[rd, "p^n_A"', rightarrow] & &
		Z^{n+1}\\
		&
		A^n/I^n \arrow[ur, "\overline{d^n_A}"', rightarrow] 
	\end{tikzcd}
	.\end{equation} 
	Which means that $\widetilde{d^n_A} = \overline{d^n_A} \circ p^n_A$.
	Applying the ker-coker sequence lemma to this composition we obtain the exact sequence
	\begin{equation}
	\begin{tikzcd}[row sep=tiny]
		0 \arrow[r, "", rightarrow] &
		\ker p^n_A \arrow[r, "", rightarrow] &
		\ker \widetilde{d^n_A} \arrow[r, "", rightarrow] 
		\arrow[d, phantom, ""{coordinate, name=Z}] &
		\ker \overline{d^n_A} 
		\arrow[dll, "", rounded corners, bend left = 100,
		to path={ -- ([xshift=2ex]\tikztostart.east)\tikztonodes
			|- (Z) [near end]
			-| ([xshift=-2ex]\tikztotarget.west)
			-- (\tikztotarget)}] & \\
		& 
		\coker p^n_A \arrow[r, "", rightarrow] &
		\coker \widetilde{d^n_A} \arrow[r, "", rightarrow] &
		\coker \overline{d^n_A} \arrow[r, "", rightarrow] &
		0
	\end{tikzcd}
	.\end{equation} 
	Since $p^n_A$ is an epimorphism it has zero cokernel, moreover its kernel,
	by definition is $I^n$.
	Then, substituting in the above we obtain
	\begin{equation}\label{eqn:exactKerCokerClaim}
	\begin{tikzcd}[row sep=tiny]
		0 \arrow[r, "", rightarrow] &
		I^n \arrow[r, "", rightarrow] &
		\ker \widetilde{d^n_A} \arrow[r, "", rightarrow] 
		\arrow[d, phantom, ""{coordinate, name=Z}] &
		\ker \overline{d^n_A} 
		\arrow[dll, "", rounded corners, bend left = 100,
		to path={ -- ([xshift=2ex]\tikztostart.east)\tikztonodes
			|- (Z) [near end]
			-| ([xshift=-2ex]\tikztotarget.west)
			-- (\tikztotarget)}] & \\
		& 
		0 \arrow[r, "", rightarrow] &
		\coker \widetilde{d^n_A} \arrow[r, "", rightarrow] &
		\coker \overline{d^n_A} \arrow[r, "", rightarrow] &
		0
	\end{tikzcd}
	.\end{equation} 
	From exactness of the sequence we immediately obtain
	\begin{equation}
		\coker \overline{d^n} \simeq \coker \widetilde{d^n} \simeq
		Z^{n+1}/I^{n+1} = H^{n+1} \left( A^\bullet \right)
	.\end{equation} 
	Then, since the inclusion $i^n_A\colon Z^n \hookrightarrow A^n$ is injective
	and $d^n_A = i^n_A \circ \widetilde{d^n_A}$,
	one obtains 
	\begin{equation}
	Z^n = \ker d^n_A \simeq \ker \widetilde{d^n}
	.\end{equation} 
	Then, again, from exactness of \eqref{eqn:exactKerCokerClaim}
	one can conclude with:
	\begin{equation}
		\ker \overline{d^n} \simeq Z^n/I^n = H^{n} \left( A^\bullet \right)
	.\end{equation} 
\end{proof}

\begin{defn}[Nullhomotopic morphism]
	Consider two cochain complexes $\left( A^{\bullet}, d_{A} \right),
	\left( B^{\bullet}, d_{B} \right) \in \mathrm{Ch}(\mathsf{C})$.
	A morphism $f^\bullet\colon A^\bullet \to B^\bullet$ is said
	to be {\em nullhomotopic} iff there exist a family 
	$\left\{ h^n \right\}_{n \in \mathbb{Z}}$ of morphisms
	\begin{equation}
	\begin{tikzcd}
		h^n \colon A^n \arrow[r, "", rightarrow] &
		B^{n-1}
	\end{tikzcd}
	\end{equation} 
	s.t. for all $n \in \mathbb{Z}$,
	\begin{equation}
	f^n = h^{n+1} \circ d_A^n + d_B^{n-1} \circ h^n
	.\end{equation} 
	In pictures:
	\begin{equation}
	\begin{tikzcd}
		\ldots \arrow[r, "", rightarrow] &
		A^{n-1} \arrow[r, "d^{n-1}_A", rightarrow] \arrow[d, "f^{n-1}", rightarrow] &
		A^n \arrow[r, "d_A^n", rightarrow] \arrow[d, "f^n", rightarrow] 
						 \arrow[ld, "h^n", rightarrow] &
		A^{n+1} \arrow[r, "d_A^{n+1}", rightarrow] \arrow[d, "f^{n+1}", rightarrow]
						\arrow[ld, "h^{n+1}", rightarrow] &
		\ldots\\
		\ldots \arrow[r, "", rightarrow] &
		B^{n-1} \arrow[r, "d_B^{n-1}"', rightarrow] &
		B^n \arrow[r, "d_B^n"', rightarrow] &
		B^{n+1} \arrow[r, "d_B^{n+1}"', rightarrow] &
		\ldots
	\end{tikzcd}
	.\end{equation} 
\end{defn}

\begin{prop}[]
	If $f^\bullet\colon A^\bullet \to B^\bullet$ is nullhomotopic, then,
	for any $n \in \mathbb{Z}$, it induces the zero morphism
	at the level of cohomologies, i.e.
	\begin{equation}
		\begin{tikzcd}
		0 = H^{n} \left( f \right)\colon H^{n} \left( A^\bullet \right) \arrow[r, "", rightarrow] &
		H^{n} \left( B^\bullet \right).
		\end{tikzcd}
	\end{equation} 
\end{prop}
\begin{proof}
	By definition (using Freyd-Mitchell, operating in a category of modules) we have
	\begin{align}
		H^{n} \left( f \right)\colon Z^n(A^\bullet)/I^n(A^\bullet) 
		&\longrightarrow Z^n(B^\bullet)/I^n(B^\bullet) \\
		\overline{x} &\longmapsto \overline{f^n(x)} \nonumber
	,\end{align} 
	for $x \in Z^n(A^\bullet)$.
	By assumption we have the following equality:
	\begin{equation}
		f^n(x) = h^{n+1} \circ d_A^n(x) + d^{n+1} \circ h^n(x)
	.\end{equation} 
	Since $x \in Z^n(A^\bullet) = \ker d_A^n(x)$ the first summand is zero.
	Then $f^n(x) \in \ima d^{n+1}_B$ and
	\begin{equation}
		0 = \overline{f^n(x)} \in Z^n(B^\bullet)/I^n(B^\bullet)
	.\end{equation} 
\end{proof}

\begin{defn}[Homotopic morphisms]
	Two morphisms $f^\bullet, g^\bullet\colon A^\bullet \to B^\bullet$
	are called {\em homotopic} iff $f^\bullet - g^\bullet$ is nullhomotopic.
	The family $\left( h^n \right)_{n \in \mathbb{Z}}$ making $f^\bullet - g^\bullet$
	nullhomotopic is called {\em homotopy}.
	More explicitly we have, for every $n \in \mathbb{Z}$
	\begin{equation}
	f^n - g^n = h^{n+1} \circ d^n_A + d^n_B \circ h^n
	.\end{equation} 
\end{defn}

\begin{rem}[]
	If $f^\bullet, g^\bullet\colon A^\bullet \to B^\bullet$ are homotopic, then
	for all $n \in \mathbb{Z}$ we have
	\begin{equation}
	\begin{tikzcd}
		H^{n} \left( f^\bullet \right) = H^{n} \left( g^\bullet \right) \colon
		H^{n} \left( A^\bullet \right) \arrow[r, "", rightarrow] &
		H^{n} \left( B^\bullet \right).
	\end{tikzcd}
	\end{equation} 
	This is true, since $H^n$ is an additive functor, hence
	\begin{equation}
	H^{n} \left( f^\bullet - g^\bullet \right) =
	H^{n} \left( f^\bullet \right) - H^{n} \left( g^\bullet \right)
	.\end{equation} 
\end{rem}

\begin{rem}[]
	Let $f,g\colon X \to Y$ be continuous maps of nice (e.g. CW-complexes) topological spaces.
	One can define the singular cochain complex $C^\bullet(X,\mathbb{Z})$.
	Suppose that $f$ and $g$ are homotopic, i.e. there is a continuous map
	\begin{equation}
		\begin{tikzcd}
		h\colon X \cross \left[ 0,1 \right] \arrow[r, "", rightarrow] &
		Y
		\end{tikzcd}
	\end{equation} 
	s.t. $\left.h\right|_{X \cross \left\{ 0 \right\}} = f$ and
	$\left.h\right|_{X \cross \left\{ 1 \right\}} = g$.
	Then they induce two maps of complexes
	\begin{equation}
	\begin{tikzcd}
		g^\bullet, f^\bullet: C^\bullet(X, \mathbb{Z}) \arrow[r, "", rightarrow] &
		C^\bullet(Y, \mathbb{Z})
	\end{tikzcd}
	\end{equation} 
	which are homotopic as in the above definition.
\end{rem}

\subsection{Resolutions}
Let $\mathsf{C}$ be an {\em abelian} category.
\begin{defn}[Resolution]
	Consider $A \in \mathrm{Ob} \left(\mathsf{C}\right)$, then a {\em resolution}
	of $A$ is an exact sequence
	\begin{equation}
	\begin{tikzcd}
		0 \arrow[r, "", rightarrow] &
		A \arrow[r, "", rightarrow] &
		I^0 \arrow[r, "d^0", rightarrow] &
		I^1 \arrow[r, "d^1", rightarrow] &
		I^2 \arrow[r, "d^2", rightarrow] &
		\ldots
	\end{tikzcd}
	\end{equation}
	If, moreover, each $I^i$ is {\em injective}, then we say that
	it is an {\em injective resolution}.
	More compactly we are going to denote it by $A \to I^\bullet$.
\end{defn}

\begin{prop}[]
	If $\mathsf{C}$ has enough injectives, then any
	$A \in \mathrm{Ob} \left(\mathsf{C}\right)$ admits an injective resolution.
\end{prop}
\begin{proof}
	Since $\mathsf{C}$ has enough injectives we have
	\begin{equation}
	\begin{tikzcd}
		0 \arrow[r, "", rightarrow] &
		A \arrow[r, "", rightarrow] &
		I^0
	\end{tikzcd}
	\end{equation} 
	for an injective object $I^0$.
	Set $I^0/A := \coker \left( A \hookrightarrow I^0 \right)$.
	Again, $\mathsf{C}$ has enough injectives, then there is
	\begin{equation}
	\begin{tikzcd}
		0 \arrow[r, "", rightarrow] &
		I^0/A \arrow[r, "", rightarrow] &
		I^1
	\end{tikzcd}
	\end{equation} 
	for $I^1$ injective.
	We define
	\begin{equation}
	\begin{tikzcd}
		d^0 \colon I^0 \arrow[r, "", twoheadrightarrow] &
		I^0/A \arrow[r, "", rightarrowtail] &
		I^1
	\end{tikzcd}
	.\end{equation} 
	The kernel of the above map is just $A$, since the second map is injective, hence it coincides with
	$A = \ker I^0 \twoheadrightarrow I^0/A$.
	Then, as above, we can choose $I^2$ injective and
	\begin{equation}
	\begin{tikzcd}
		0 \arrow[r, "", rightarrow] &
		\coker d^0 \arrow[r, "", rightarrow] &
		I^2
	\end{tikzcd}
	.\end{equation} 
	We define again
	\begin{equation}
	\begin{tikzcd}
		d^1 \colon I^1 \arrow[r, "", twoheadrightarrow] &
		\coker d^0 \arrow[r, "", rightarrowtail] &
		I^2
	\end{tikzcd}
	.\end{equation} 
	Then, again reasoning as before, we obtain
	\begin{equation}
		\ker d^1 \simeq \ker \left( \coker d^0 \right) =: \ima d^0
	.\end{equation} 
	Then by induction one defines $\left( I^{\bullet}, d_{I} \right)$
	s.t. $\ker d^n \simeq \ima d^{n-1}$.
\end{proof}

\begin{rem}[Another point of view]
	Given $A \in \mathrm{Ob} \left(\mathsf{C}\right)$, one can view 
	an injective resolution
	of $A$ as a morphism of complexes
	\begin{equation}
	\begin{tikzcd}
		A^\bullet \arrow[r, "", rightarrow] &
		I^\bullet.
	\end{tikzcd}
	\end{equation} 
	If we consider the complex $A^\bullet$ concentrated in degree $0$, i.e.
	\begin{equation}
	\begin{tikzcd}
		\ldots \arrow[r, "", rightarrow] &
		0 \arrow[r, "", rightarrow] &
		A \arrow[r, "", rightarrow] &
		0 \arrow[r, "", rightarrow] &
		\ldots
	\end{tikzcd}
	\end{equation} 
	$A^0 = A$ and $A^n = 0$ for all $n \neq 0$.
	Then the morphism of complexes
	\begin{equation}
	\begin{tikzcd}
		A^\bullet \arrow[r, "", rightarrow] &
		I^\bullet
	\end{tikzcd}
	\end{equation} 
	is given by the commutative
	\begin{equation}
	\begin{tikzcd}
		\ldots \arrow[r, "", rightarrow] &
		0 \arrow[r, "", rightarrow] 
		\arrow[d, "", rightarrow] &
		A \arrow[r, "", rightarrow] 
		\arrow[d, "", rightarrow] &
		0 \arrow[r, "", rightarrow] 
		\arrow[d, "", rightarrow] &
		0 \arrow[r, "", rightarrow]
		\arrow[d, "", rightarrow] &
		\ldots\\
		\ldots \arrow[r, "", rightarrow] &
		0 \arrow[r, "", rightarrow] &
		I^0 \arrow[r, "d^0"', rightarrow] &
		I^1 \arrow[r, "d^1"', rightarrow] &
		I^2 \arrow[r, "", rightarrow] &
		\ldots
	\end{tikzcd}
	\end{equation} 
	such that have the following isomorphisms:
	\begin{equation}
		\begin{tikzcd}[row sep=tiny]
			A \simeq H^{0} \left( A^\bullet \right) \arrow[r, "\sim", rightarrow] &
			H^{0} \left( I^\bullet \right) \simeq 
			\ker d^0\\
			0 = H^{n} \left( A^\bullet \right) 
			\arrow[r, "", equal] &
			H^{n} \left( I^\bullet \right)
		\end{tikzcd}
	.\end{equation} 
	By exactness of $I^\bullet$ and triviality of $A^\bullet$ 
	the above mean that $A^\bullet \to I^\bullet$ induces isomorphism
	for each $n$ at the level of cohomologies.
	And we give a name to such maps.
\end{rem}
				
\begin{defn}[Quasi-isomorphism]
	Consider a morphism of complexes in $\mathrm{Ch}(\mathsf{C})$
	\begin{equation}
	\begin{tikzcd}
		f\colon X^\bullet \arrow[r, "", rightarrow] &
		Y^\bullet.
	\end{tikzcd}
	\end{equation} 
	It is a {\em quasi-isomorphism} iff, for all $n \in \mathbb{Z}$, it induces
	an isomorphism at the level of the $n$-th cohomology:
	\begin{equation}
		\begin{tikzcd}
			H^{n} \left( f \right)\colon H^{n} \left( X^\bullet \right)
			\arrow[r, "\sim", rightarrow] &
			H^{n} \left( Y^\bullet \right)
		\end{tikzcd}
	.\end{equation} 	
\end{defn}
\begin{rem}[Resolution, take 2]
	Actually an injective resolution of $A$ is just the data of a quasi-isomorphism
	between $A^\bullet$ the complex concentrated in $0$ defined above, 
	and a complex $I^\bullet$ of injective objects, starting from degree $0$.
	This point of view will allow to generalize the definition later on.
\end{rem}

\begin{defn}[Extension of a morphism]
	Let $A, B \in \mathsf{C}$, with a morphism
	$f\colon A \to B$.
	Consider two resolutions
	\begin{equation}
	\begin{tikzcd}
		A \arrow[r, "", rightarrow] &
		I^\bullet
	\end{tikzcd}
	\qquad \text{ and } \qquad
	\begin{tikzcd}
		B \arrow[r, "", rightarrow] &
		J^\bullet
	\end{tikzcd}
	.\end{equation} 
	An extension 
	of $f\colon A \to B$
	is a morphism of complexes
	\begin{equation}
		\begin{tikzcd}
		f^\bullet: I^\bullet \arrow[r, "", rightarrow] &
		J^\bullet
		\end{tikzcd}
	\end{equation} 
	s.t. the following diagram commutes
	\begin{equation}
	\begin{tikzcd}
		0 \arrow[r, "", rightarrow] &
		A \arrow[r, "", rightarrow] \arrow[d, "f", rightarrow] &
		I^0 \arrow[r, "", rightarrow] \arrow[d, "f^0", rightarrow] &
		I^1 \arrow[r, "", rightarrow] \arrow[d, "f^1", rightarrow] &
		\ldots\\
		0 \arrow[r, "", rightarrow] &
		B \arrow[r, "", rightarrow] &
		J^0 \arrow[r, "", rightarrow] &
		J^1 \arrow[r, "", rightarrow] &
		\ldots
	\end{tikzcd}
	.\end{equation} 
	In other words the following diagram of morphisms of complexes commute:
	\begin{equation}
	\begin{tikzcd}
		A^\bullet \arrow[r, "", rightarrow] 
		\arrow[d, "f"', rightarrow] &
		I^\bullet \arrow[d, "f^\bullet", rightarrow] \\
		B^\bullet \arrow[r, "", rightarrow] &
		J^\bullet
	\end{tikzcd}
	.\end{equation} 
\end{defn}

\begin{prop}[]\label{prop:ExistanceUniquenessLift}
	Consider $A, B \in \mathsf{C}$,
	$A \to  I^\bullet$ a resolution of $A$
	and $B \to  J^\bullet$ an injective resolution of $B$.
	Then, for any $f\colon A \to B$, there is an extension
	$f^\bullet\colon I^\bullet \to J^\bullet$,
	which is unique up to homotopy.
\end{prop}
\begin{proof}\leavevmode\vspace{-.2\baselineskip}
	\begin{description}
		\item[Uniqueness by homotopy:]
	Suppose one has two extensions $f^\bullet, g^\bullet\colon I^\bullet \to J^\bullet$
	of $f\colon A \to B$.
	Then $f^\bullet - g^\bullet\colon I^\bullet \to J^\bullet$
	extends the zero map $f-f = 0\colon A \to B$.
	Then we only need to show that any extension of the zero map
	is nullhomotopic.
	Consider
	\begin{equation}
	\begin{tikzcd}
		A \arrow[d, "0"', rightarrow] \arrow[r, "", rightarrow] &
		I^\bullet \arrow[d, "f^\bullet", rightarrow] \\
		B \arrow[r, "", rightarrow] &
		J^\bullet
	\end{tikzcd}
	.\end{equation} 
	In particular, at degree $0$, we have
	\begin{equation}
	\begin{tikzcd}
		0 \arrow[r, "", rightarrow] &
		A \arrow[d, "0"', rightarrow] \arrow[r, "i_A", rightarrow] \arrow[rd, "", dashrightarrow] &
		I^0 \arrow[d, "f^0", rightarrow] \\
		0 \arrow[r, "", rightarrow] &
		B \arrow[r, "i_B"', rightarrow] &
		J^0
	\end{tikzcd}
	.\end{equation} 
	By commutativity of the diagram the dashed arrow is the zero map,
	hence $f^0$ factorizes through
	$\coker (A \hookrightarrow I^0) = I^0/A$.
	This gives rise to the map
	\begin{equation}
	\begin{tikzcd}
		I^0/A \arrow[r, "\overline{f^0}", rightarrow] &
		J^0
	\end{tikzcd}
	.\end{equation} 
	Then we recall that, since $A \to I^\bullet$ is a complex, $d_I^0 \circ i_A = 0$,
	hence also this map factors through $I^0/A$ and we have 
	$\ker d^0_I \simeq \ima i_A = \ker \coker i_A$.
	But then, applying lemma \ref{lem:kerCokerSequence}, we obtain that the map
	through which $d^0_I$ factorizes is a mono, i.e. $I^0/A \hookrightarrow I^1$.
	Moreover, since $J^0$ is injective, we obtain the diagram
	\begin{equation}
	\begin{tikzcd}
		I^0 \arrow[r, "", twoheadrightarrow] \arrow[rd, "f^0"', rightarrow] 
		\arrow[rr, "d_I^0", rightarrow, bend left] &
		I^0/A \arrow[r, "", rightarrowtail]  \arrow[d, "\overline{f^0}", rightarrow] &
		I^1 \arrow[ld, "\exists h^1", dashrightarrow] \\
		&
		J^0
	\end{tikzcd}
	.\end{equation} 
	More explicitly this diagram states that $f^0 = h^1 \circ d_I^0$.
	Suppose now, by strong induction on $n \geq 1$, that there exist
	$h^n\colon I^n \longrightarrow J^{n-1}$
	and
	$h^{n+1}\colon I^{n+1} \longrightarrow J^{n}$
	s.t.
	\begin{equation}
	f^n = h^{n+1} \circ d^n_I + d^{n-1}_J \circ h^n
	.\end{equation} 
	The above construction grants this for $n = 0$, with $h^0 = 0$. In fact
	\begin{equation}
	\begin{tikzcd}
		0 \arrow[r, "", rightarrow] &
		I^0 \arrow[ld, "h^0", rightarrow] \arrow[r, "d^0_I", rightarrow] 
		\arrow[d, "f^0", rightarrow] &
		I^1 \arrow[d, "f^1", rightarrow] 
		\arrow[ld, "h^1", rightarrow] \\
		0 \arrow[r, "", rightarrow] &
		J^0 \arrow[r, "", rightarrow] &
		J^1
	\end{tikzcd}
	.\end{equation} 
	By induction we obtain the following diagram:
	\begin{equation}
	\begin{tikzcd}
		\ldots \arrow[r, "", rightarrow] &
		I^{n-1} \arrow[r, "d^{n-1}_I", rightarrow] 
		\arrow[d, "f^1", rightarrow] &
		I^n \arrow[r, "d^n_I", rightarrow] \arrow[ld, "h^n", rightarrow] 
		\arrow[d, "f^n", rightarrow] &
		I^{n+1} \arrow[d, "f^{n+1}", rightarrow] \arrow[ld, "h^{n+1}", rightarrow] 
		\arrow[r, "d^{n+1}_I", rightarrow] &
		I^{n+2} \arrow[r, "", rightarrow] \arrow[d, "f^{n+2}", rightarrow] &
		\ldots \\
		\ldots \arrow[r, "", rightarrow] &
		J^{n-1} \arrow[r, "d^{n-1}_J"', rightarrow] &
		J^n \arrow[r, "d^n_J"', rightarrow] &
		J^{n+1} \arrow[r, "d^{n+1}_J"', rightarrow] &
		J^{n+2} \arrow[r, "", rightarrow] &
		\ldots
	\end{tikzcd}
	.\end{equation} 
	Consider now the following composition:
	\begin{align}
		(f^{n+1} - d_I^n \circ h^{n+1}) \circ d_I^n &=
		(f^{n+1} \circ d_I^n) - d_I^n \circ h^{n+1} \circ d_I^n\\
		&=
		\left( f^{n+1} \circ d_I^n \right) - d_J^n \circ h^{n-1} =
		d^n_I \circ \left( f^n - h^{n+1} \circ d_I^n \right)\nonumber\\
		&=
		d^n_J \circ d_J^{n-1} \circ h^n = 0\nonumber
	,\end{align} 
	where the last equality holds from induction hypothesis.
	Hence the map $f^{n+1} - d_J^n \circ h^{n+1} = (*)$ 
	factorizes through $\coker d^n_I$,
	giving rise to the following commutative diagram:
	\begin{equation}
		\begin{tikzcd}[row sep=small]
		I^{n+1} \arrow[r, "", rightarrow] \arrow[rr, "(*)", bend left, rightarrow] &
		\coker d_I^n \arrow[r, "", rightarrow] &
		J^{n+1}\\
	\end{tikzcd}
	.\end{equation} 
	Again, since $d^{n+1}_I$ factors through $\coker d^n_I$,
	using lemma \ref{lem:kerCokerSequence}, we obtain that the factorization is injective.
	Then combining it with the above diagram we get
	\begin{equation}
	\begin{tikzcd}
		I^{n+1} \arrow[r, "", rightarrow] \arrow[d, "(*)"', rightarrow] &
		\coker d_I^{n} \arrow[r, "", rightarrowtail] 
		\arrow[ld, "", rightarrow] &
		I^{n+2} \arrow[dll, "\exists h^{n+2}", dashrightarrow, bend left] \\
		J^{n+1}
	\end{tikzcd}
	.\end{equation} 		
	And $h^{n+2}$ exists since $J^{n+1}$ is injective.
	Finally commutativity of the diagram grants that
	\begin{equation}
	f^{n+1} - d^n_J \circ h^{n+1} = h^{n+2} \circ d_I^{n+1}
	.\end{equation} 

		\item[Existance of the extension:]
			By hypothesis we are given the diagram, with exact rows:
	\begin{equation}
	\begin{tikzcd}
		0 \arrow[r, "", rightarrow] &
		A \arrow[r, "i_A", rightarrow] 
		\arrow[d, "f"', rightarrow] \arrow[rd, "", dashrightarrow] &
		I^0\\
		0 \arrow[r, "", rightarrow] &
		B \arrow[r, "i_B", rightarrow] &
		J^0
	\end{tikzcd}
	.\end{equation} 
	Since $J^0$ is injective and $i_A$ a mono, there exists a map
	$f^0\colon I^0 \to J^0$ making the diagram commute:
	\begin{equation}
	\begin{tikzcd}
		A \arrow[r, "i_A", rightarrow] 
		\arrow[d, "f"', rightarrow] &
		I^0 \arrow[d, "f^0", rightarrow] \\
		B \arrow[r, "i_B", rightarrow] &
		J^0
	\end{tikzcd}
	.\end{equation} 
	As proved before $d^0_I$ factors through $\coker i_A$, via a monomorphism.
	Moreover we can see that $(\coker i_B \circ f^0) \circ i_A = \coker i_B \circ i_B \circ f = 0$,
	hence $f^0$ factorizes via $\overline{f^0}$ through $\coker i_A$.
	This gives rise to the commutative diagram:
	\begin{equation}
	\begin{tikzcd}
		A \arrow[r, "i_A", rightarrow] 
		\arrow[d, "f"', rightarrow] &
		I^0 \arrow[d, "f^0", rightarrow] 
		\arrow[r, "", rightarrow] &
		I^0/A \arrow[r, "", rightarrowtail] \arrow[d, "\overline{f^0}", rightarrow] 
		\arrow[rd, "", dashrightarrow] &
		I^1 \\
		B \arrow[r, "i_B", rightarrow] &
		J^0 \arrow[r, "", rightarrow] &
		J^0/B \arrow[r, "", rightarrowtail] &
		J^1
	\end{tikzcd}
	.\end{equation} 
	Again, by injectivity of $J^1$ one obtains the map
	$f^1\colon I^1 \to J^1$.
	Notice that commutativity of the whole diagram, in particular, grants commutativity of
	\begin{equation}
	\begin{tikzcd}
		I^0 \arrow[r, "d^0_I", rightarrow] \arrow[d, "f^0"', rightarrow] &
		I^1 \arrow[d, "f^1", rightarrow] \\
		J^0 \arrow[r, "d^0_J"', rightarrow] &
		J^1
	\end{tikzcd}
	.\end{equation} 
	Now we can procede by induction on $n \geq 1$.
	Assume that we have constructed $f^{n-1}$ and $f^n$, then reasoning similarly to before we obtain
	the commutative diagram
	\begin{equation}
	\begin{tikzcd}
		I^{n-1} \arrow[r, "i_A", rightarrow] 
		\arrow[d, "f^{n-1}", rightarrow] &
		I^n \arrow[d, "f^n", rightarrow] 
		\arrow[r, "", rightarrow] &
		\coker d^{n-1}_I \arrow[r, "", rightarrowtail] 
		\arrow[d, "\overline{f^n}", rightarrow] 
		\arrow[rd, "", dashrightarrow] &
		I^{n+1} \arrow[d, "f^{n+1}", dashrightarrow] \\
		J^{n-1} \arrow[r, "i_B", rightarrow] &
		J^n \arrow[r, "", rightarrow] &
		\coker d^{n-1}_J \arrow[r, "", rightarrowtail] &
		J^{n+1}
	\end{tikzcd}
	.\end{equation} 
	Again the arrow $f^{n+1}$ exists by injectivity of $J^{n+1}$, and the following square commutes:
	\begin{equation}
	\begin{tikzcd}
		I^n \arrow[r, "d^n_I", rightarrow] \arrow[d, "f^n"', rightarrow] &
		I^{n+1} \arrow[d, "f^{n+1}", rightarrow] \\
		J^n \arrow[r, "d^n_J"', rightarrow] &
		J^{n+1}
	\end{tikzcd}
	.\end{equation} 
	Then this construction gives rise to a morphism of complexes $f^\bullet\colon I^\bullet \to J^\bullet$,
	and by construction it is an extension of $f\colon A \to B$.\qedhere
	\end{description} 
\end{proof}
\end{document}
