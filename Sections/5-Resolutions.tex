\documentclass[../Main]{subfiles}
\begin{document}
\section{Resolutions}
\subsection{Injective and projective objects}
\begin{rem}[]
	Let $\mathsf{C}$ be an abelian category, and $M \in \mathrm{Ob} \left(\mathsf{C}\right)$.
	Then the covariant functors
	\begin{align}
		\mathrm{Hom}_{\mathsf{C}} \left( M, - \right)\colon \mathsf{C} &\longrightarrow \mathsf{Ab} \\
		A &\longmapsto \mathrm{Hom}_{\mathsf{C}} \left( M, A \right) \nonumber
	\end{align} 
	and
	\begin{align}
		\mathrm{Hom}_{\mathsf{C}} \left( - ,M \right)\colon \mathsf{C}^{op} &\longrightarrow \mathsf{Ab} \\
		A &\longmapsto \mathrm{Hom}_{\mathsf{C}} \left( A, M \right) \nonumber
	\end{align} 
	are both left exact.
	Both of these are consequences of lemma \ref{lem:HomExactnessEquivalence}.
\end{rem}

Though general it is not true, 
some objects make one of the two above functors (or both)
also right exact.
These are special objects, and we are going to study
them in the following pages:

\begin{defn}[Injective and projective objects]
	Let $\mathsf{C}$ be an abelian category.
	\begin{enumerate}
		\item An object $I \in \mathrm{Ob} \left(\mathsf{C}\right)$ is {\em injective}
			iff
			\begin{align}
				\mathrm{Hom}_{\mathsf{C}} \left( -, I \right)\colon \mathsf{C}^{op} &\longrightarrow \mathsf{Ab}
			\end{align} 
			is exact.
		\item An object $P \in \mathrm{Ob} \left(\mathsf{C}\right)$ is {\em projective}
			iff
			\begin{align}
				\mathrm{Hom}_{\mathsf{C}} \left( P, - \right) \colon \mathsf{C} &\longrightarrow \mathsf{Ab}
			\end{align} 
			is exact.
	\end{enumerate}
\end{defn}

\begin{rem}[]\leavevmode\vspace{-.2\baselineskip}
	\begin{itemize}
		\item An object $I \in \mathrm{Ob} \left(\mathsf{C}\right)$ is injective iff,
	given any monomorphism $i:A \hookrightarrow B$, then
	the induced map
	\begin{equation}
	\begin{tikzcd}
		\mathrm{Hom}_{\mathsf{C}} \left( B, I \right) \arrow[r, "", twoheadrightarrow] &
		\mathrm{Hom}_{\mathsf{C}} \left( A, I \right)
	\end{tikzcd}
	\end{equation} 
	is surjective.
	In other words the following diagram commutes:
	\begin{equation}
	\begin{tikzcd}
		0 \arrow[r, "", rightarrow] &
		A \arrow[r, "i", rightarrow] \arrow[d, "\alpha", rightarrow] &
		B \arrow[ld, "\exists\, \tilde{\alpha}", dashrightarrow] \\
		&
		I &
	\end{tikzcd}
	.\end{equation} 
	More explicitly, given any morphismh $\alpha\colon A \to I$,
	one can lift it to $\tilde{\alpha}\colon B \to I$ s.t.
	$\alpha = \tilde{\alpha} \circ i$.
	(Notice that the lift might not be unique).

		\item An object $P \in \mathrm{Ob} \left(\mathsf{C}\right)$ is projective iff,
	given any epimorphism $p:B \twoheadrightarrow C$, then
	the induced map
	\begin{equation}
	\begin{tikzcd}
		\mathrm{Hom}_{\mathsf{C}} \left( P, B \right) \arrow[r, "", twoheadrightarrow] &
		\mathrm{Hom}_{\mathsf{C}} \left( P, C \right)
	\end{tikzcd}
	\end{equation} 
	is surjective.
	In other words the following diagram commutes:
	\begin{equation}
	\begin{tikzcd}
		B \arrow[r, "p", rightarrow] &
		C \arrow[r, "", rightarrow] &
		0\\
		&
		P \arrow[u, "\beta", rightarrow] \arrow[ul, "\exists\, \tilde{\beta}", dashrightarrow] &
	\end{tikzcd}
	.\end{equation} 
	More explicitly, given any morphismh $\beta\colon P \to C$,
	one can lift it to $\tilde{\beta}\colon P \to B$ s.t.
	$\beta = p \circ \tilde{\beta}$.
	(Notice that the lift might not be unique).
	\end{itemize}
\end{rem}

\subsubsection{Injectives in $R\text{-}\mathsf{Mod}$}
In the following we will assume $R$ is a commutative ring.
Everything works also for non commutative rings, but in this case
there are fewer notation problems.

We will denote by $R\text{-}\mathsf{Mod}$
the category of left $R$-modules.
Clearly this is an {\em abelian} category.

\begin{lem}[Baer]
	A module $M \in R\text{-}\mathsf{Mod}$ is injective iff,
	given an ideal $I \triangleleft R$, any $R$-linear map
	\begin{align}
		\alpha\colon I &\longrightarrow M
	\end{align} 
	extends to a morphism $\tilde{\alpha}\colon R \to M$.
	In other words we have the following diagram:
	\begin{equation}
	\begin{tikzcd}
		0 \arrow[r, "", rightarrow] &
		I \arrow[r, "i", rightarrow] \arrow[d, "\alpha", rightarrow] &
		R \arrow[dl, "\tilde{\alpha}", dashrightarrow] \\
		&
		M &
	\end{tikzcd}
	.\end{equation} 
\end{lem} 
\begin{proof}
	tk: read again the proof, there will obviously be quite a few things to fix...
	The direct implication is obvious.
	Let's prove the converse.
	Let $A \subset B$ be a submodule of $B$, then the inclusion is a monomorphism.
	Consider a map $\alpha\colon A \to M$, we want to extend it to $B$.
	Let's consider the set
	$\mathcal{S}$ of pairs $\left( A', \alpha' \right)$ s.t. $A' \subset B$ and
	$\alpha'\colon A' \to M$ coincides with $\alpha$ on $A$,
	i.e. $\left.\alpha'\right|_{A} = \alpha$.
	This set is clearly endowed with a partial order:
	$(A', \alpha') \leq (A'', \alpha'')$ iff $A' \subset A''$
	and $\left.\alpha''\right|_{A'} = \alpha'$.

	Consider now $T$ a totally ordered subset of $\mathcal{S}$.
	\begin{equation}
		T = \left\{ \left( A_a, \alpha_a \right) \right\}_{a \in \mathcal{A}} 
	.\end{equation} 
	We want to define $A_{\infty} := \bigcup_{a \in \mathcal{A}} A_a \subset B$
	an $R$-submodule, and
	\begin{align}
		\alpha_{\infty}\colon A_{\infty} &\longrightarrow M 
	\end{align} 
	s.t. $\left.\alpha_{\infty}\right|_{A_a} = \alpha_a$ for all $a$.
	Then $(A_\infty, \alpha_\infty) \in \mathcal{S}$ is an upper bound for $T$,
	and by Zorn's lemma we obtain the existance of a maximal element $(C, \gamma) \in S$.
	Suppose that $C \neq B$, then we can take $x \in B \setminus C \neq \emptyset$.
	Let's define now
	\begin{equation}
	I := \left\{ r \in R \ \middle|\ r \cdot x \in C \right\} \triangleleft R
	.\end{equation} 
	Let's denote with $r_*$ the multiplication by $r \in R$.
	Then we can define $\gamma \circ r_*\colon I \to M$.
	By assumption we can extend this to a map $\psi \colon R \to M$, making the diagram (tk) commute.
	In particular we can extend these two maps to $C \subsetneq C + R\cdot x \subset B$, by
	\begin{align}
		\tilde{\gamma}\colon C + R \cdot x &\longrightarrow M \\
		c + r \cdot x &\longmapsto \gamma(c) + \psi(r) \nonumber
	.\end{align} 
	This map is $R$-linear, well defined (on the intersection $C \cap R \cdot x$ $\psi$ acts
	exactly as $\gamma \circ r_*$), and extends $\gamma$.
	Since $C \subsetneq C + R \cdot x$, we have a contradiction with maximality
	of $(C, \gamma)$.
	Then $C = B$.
\end{proof}

\begin{rem}[]
	An element $r \in R$ is a non-zero-divisor
	iff the multiplication map 
	\begin{align}
		r_*\colon R &\longrightarrow R \\
		x &\longmapsto r \cdot x \nonumber
	\end{align} 
	is injective.
\end{rem}

\begin{defn}[Divisible $R$-Module]
	Consider $M \in R\text{-}\mathsf{Mod}$.
	We say that $M$ is {\em divisible} iff, for any $r \in R$ non-zero-divisor,
	the multiplication map
	\begin{align}
		r_*\colon M &\longrightarrow M \\
		x &\longmapsto r \cdot x \nonumber
	\end{align} 
	is surjective.
\end{defn}

\begin{prop}[]
	Consider $M \in R\text{-}\mathsf{Mod}$.
	\begin{enumerate}
		\item If $M$ is an {\em injective} module, then it is also {\em divisible}.
		\item If, moreover, $R$ is a PID, then we have also the converse,
			i.e. $M$ is {\em injective} iff it is {\em divisible}.
	\end{enumerate}
\end{prop}
\begin{proof}\leavevmode\vspace{-.2\baselineskip}
	\begin{enumerate}
		\item Consider a non-zero-divisor $r \in R$,
			let $x \in M$.
			Then, by injectivity, we can extend the map 
			\begin{align}
				\alpha\colon R &\longrightarrow M \\
				r &\longmapsto r \cdot x \nonumber
			\end{align} 
			to a map $\tilde{\alpha}$ making the diagram commute:
			\begin{equation}
			\begin{tikzcd}
				0 \arrow[r, "", rightarrow] &
				R \arrow[r, "r_*", rightarrow] \arrow[d, "\alpha", rightarrow] &
				R \arrow[dl, "\tilde{\alpha}", rightarrow] \\
				&
				M &
			\end{tikzcd}
			.\end{equation} 
			In other words we know that $\alpha = \tilde{\alpha} \circ r_*$.
			In particular $x = \alpha(1) = \tilde{\alpha} \circ r_* (1) = r \cdot \tilde{\alpha}(1)$,
			i.e. $M$ is divisible.

		\item Assume now that $R$ is a PID.
			Let $J \triangleleft R$ be an ideal, then $J = (r)$ for some $r \in R$.
			Let's use Baer's criterion:
			consider a map $\alpha\colon (r) \to M$ any ideal, let's extend this to the whole
			ring $R$.
			Let $x := \alpha(r) \in M$.
			Since $M$ is divisible, then there exists $y \in M$
			s.t. $r \cdot y = x$.
			We can now define $\tilde{\alpha}(1) := y$, then $\tilde{\alpha}$ is
			uniquely defined.
			Moreover it is clear that $\tilde{\alpha}$ extends $\alpha$, since
			\begin{equation*}
			\tilde{\alpha}(r \cdot 1) = r \cdot \tilde{\alpha}(1) = r \cdot y = x
			= \alpha(r).\qedhere
			\end{equation*} 
	\end{enumerate}
\end{proof}

\begin{cor}
	Let $R = \mathbb{Z}$, then in $\mathbb{Z}\text{-}\mathsf{Mod} \cong \mathsf{Ab}$, we have
	\begin{enumerate}
		\item $\mathbb{Q}$ and $\mathbb{Q}/\mathbb{Z}$ are both injective.
		\item $\mathbb{Z}$ is not injective.
	\end{enumerate}
\end{cor} 

\begin{rem}[]
	For all $n \in \mathbb{Z}$, the multiplication by $n$ is an iso
	between $\mathbb{Q}$ and $\mathbb{Q}$, but not
	between $\mathbb{Q}/\mathbb{Z}$ and $\mathbb{Q}/\mathbb{Z}$.
	In fact we have the following short exact sequence:
	\begin{equation}
	\begin{tikzcd}
		0 \arrow[r, "", rightarrow] &
		\mathbb{Z}/n\mathbb{Z} \arrow[r, "", rightarrow] &
		\mathbb{Q}/\mathbb{Z} \arrow[r, "n\cdot", rightarrow] &
		\mathbb{Q}/\mathbb{Z} \arrow[r, "", rightarrow] &
		0
	\end{tikzcd}
	.\end{equation} 
	In other words $\mathbb{Q}$ is uniquely divisible,
	i.e. the preimage of $x$ from the multiplication by $n$
	is unique (for any $n$ and $x$).
	On the other hand $\mathbb{Q}/\mathbb{Z}$ is not uniquely divisible.
	Instead the preimage of the multiplication by $n$ is isomorphic to
	$\mathbb{Z}/n\mathbb{Z}$.
\end{rem}

\begin{defn}[]
	Let $\mathsf{C}$ be an abelian category.
	\begin{enumerate}
		\item We say that $\mathsf{C}$ has {\em enough injectives}
			iff for every $A \in \mathrm{Ob} \left(\mathsf{C}\right)$
			there is an injective $I \in \mathrm{Ob} \left(\mathsf{C}\right)$
			and a monomorphism
			\begin{equation}
			\begin{tikzcd}
				0 \arrow[r, "", rightarrow] &
				A \arrow[r, "", rightarrow] &
				I
			\end{tikzcd}
			.\end{equation} 
		\item We say that $\mathsf{C}$ has {\em enough projectives}
			iff for every $A \in \mathrm{Ob} \left(\mathsf{C}\right)$
			there is a projective $P \in \mathrm{Ob} \left(\mathsf{C}\right)$
			and an epimorphism
			\begin{equation}
			\begin{tikzcd}
				P \arrow[r, "", rightarrow] &
				A \arrow[r, "", rightarrow] &
				0
			\end{tikzcd}
			.\end{equation} 
	\end{enumerate}
\end{defn}

\begin{prop}[]
	$\mathsf{Ab}$ has enough injectives.
\end{prop}
\begin{proof}
	Let $A \in \mathsf{Ab}$ and $S \subset A$ be a
	set of generators (not necessarily finite).
	Then one gets a surjective map
	\begin{equation}
	\begin{tikzcd}
		\bigoplus_{s \in S} \mathbb{Z}s \arrow[r, "p", rightarrow] &
		A \arrow[r, "", rightarrow] &
		0
	\end{tikzcd}
	.\end{equation} 
	This is clearly defined as the unique map induced by the inclusions $\mathbb{Z}s \hookrightarrow A$.
	In general this map is not injective, let's denote by
	\begin{equation}
	K := \ker p
	.\end{equation} 
	Then, by the first isomorphism theorem
	\begin{equation}
	\bigoplus_{s \in S}\mathbb{Z}s / K \simeq A
	.\end{equation} 
	Moreover we clearly have an injection
	(tk: define it explicitly)
	\begin{equation}
	\begin{tikzcd}
		0 \arrow[r, "", rightarrow] &
		\bigoplus_{s \in S} \mathbb{Z}s/K \arrow[r, "", rightarrow] &
		\bigoplus_{s \in S} \mathbb{Q}s/K
	\end{tikzcd}
	\end{equation} 
	where $\bigoplus_{s \in S}\mathbb{Q}s/K$ is divisible, since it is the direct sum
	of divisible modules.
\end{proof}

\begin{prop}
	Let $R$ be a {\em commutative} ring.
	Then $R\text{-}\mathsf{Mod}$ has enough injectives.
\end{prop} 
\begin{proof}
	We have an adjucntion
	\begin{equation}
	\begin{tikzcd}
		\mathrm{res}: R\text{-}\mathsf{Mod} \arrow[r, "", rightarrow, shift left] &
		\mathbb{Z}\text{-}\mathsf{Mod} : \mathrm{Hom}_{\mathbb{Z}} \left( R, - \right)
		\arrow[l, "", rightarrow, shift left] 
	\end{tikzcd}
	.\end{equation} 
	Where $\mathrm{res}$ is the restricition of scalars.
	fix $M \in R\text{-}\mathsf{Mod}$ and $A \in \mathbb{Z}\text{-}\mathsf{Mod}$.
	Then we can let $R$ act on $\mathrm{Hom}_{\mathbb{Z}} \left( R, A \right)$
	giving it a structure of $R$-module.

	In particular the adjunction is given by the following isomorphism
	\begin{equation}
	\mathrm{Hom}_{R} \left( M, \mathrm{Hom}_{\mathbb{Z}} \left( R, A \right) \right) \simeq
	\mathrm{Hom}_{\mathbb{Z}} \left( \mathrm{res}(M), A \right)
	.\end{equation} 	
	functorial in both variables.
	Let's now fix $M \in R\text{-}\mathsf{Mod}$.
	Clearly $\mathrm{res}(M) \in \mathsf{Ab}$.
	By adjunction any morphism
	\begin{align}
		f\colon \mathrm{res}(M) &\longrightarrow I
	,\end{align} 
	for $I$ an injective abelian group, gets mapped to
	\begin{align}
		\tilde{f}\colon M &\longrightarrow \mathrm{Hom}_{\mathbb{Z}} \left( R, I \right) \\
		m &\longmapsto 
		\begin{pmatrix}
			R \to I\\
			r \mapsto f(r\cdot m)
		\end{pmatrix} 
		\nonumber
	.\end{align} 
	Then $\tilde{f}$ is both $R$-linear and injective.
	We just need to show that $\mathrm{Hom}_{\mathbb{Z}} \left( R, I \right)$
	is injective, i.e. that
	\begin{equation}
	\mathrm{Hom}_{\mathsf{C}} \left( -, \mathrm{Hom}_{\mathbb{Z}} \left( R, I \right) \right)
	\end{equation} 
	is exact. By adjunction this is isomorphic (tk: check this part)
	to the functor
	\begin{equation}
		\mathrm{Hom}_{\mathbb{Z}} \left( \mathrm{res}(-), I \right):
		R\text{-}\mathsf{Mod}^{op} \to \mathsf{Ab}		
	.\end{equation} 
	In particular
	\begin{equation}
		\mathrm{Hom}_{\mathbb{Z}} \left( \mathrm{res}(-), I \right) =
		\mathrm{Hom}_{\mathbb{Z}} \left( -, I \right) \circ \mathrm{res}
	\end{equation} 
	is the composition of two exact functors, hence it is exact.
\end{proof}

\begin{prop}[]
	Let $\mathsf{C}$ and $\mathsf{D}$ be abelian categories.
	Consider an adjunction $(F,G)$.
	Then
	\begin{enumerate}
		\item $F$ is right exact.
		\item $G$ is left exact.
		\item If $F$ is exact, then
			$G$ preserves injectives.
	\end{enumerate}
\end{prop}

\begin{proof}\leavevmode\vspace{-.2\baselineskip}
\begin{enumerate}
	\item See the following.
	\item Take an exact sequence in $\mathsf{D}$:
	       \begin{equation}
	       \begin{tikzcd}
		       0 \arrow[r, "", rightarrow] &
		       A \arrow[r, "", rightarrow] &
		       B \arrow[r, "", rightarrow] &
		       C
	       \end{tikzcd}
	       .\end{equation} 
	       Consider the image sequence in $\mathsf{C}$:
	       \begin{equation}
	       \begin{tikzcd}
		       0 \arrow[r, "", rightarrow] &
		       G(A) \arrow[r, "", rightarrow] &
		       G(B) \arrow[r, "", rightarrow] &
		       G(C)
	       \end{tikzcd}
	       .\end{equation} 
	       In order to show
	       that the above is exact 
	       it's enough to show that, for any $M \in \mathrm{Ob} \left(\mathsf{C}\right)$,
	       the following
	       \begin{equation}
	       \begin{tikzcd}
		       0 \arrow[r, "", rightarrow] &
		       \mathrm{Hom}_{\mathsf{C}} \left( M, G(A) \right) \arrow[r, "", rightarrow] &
		       \mathrm{Hom}_{\mathsf{C}} \left( M, G(B) \right) \arrow[r, "", rightarrow] &
		       \mathrm{Hom}_{\mathsf{C}} \left( M, G(C) \right)
	       \end{tikzcd}
	       \end{equation} 
	       is exact.
	       By adjunction (and functoriality of the isomorphism)
	       the above is isomorphic (as a sequence) to
	       \begin{equation}
	       \begin{tikzcd}
		       0 \arrow[r, "", rightarrow] &
		       \mathrm{Hom}_{\mathsf{D}} \left( F(M), A \right) \arrow[r, "", rightarrow] &
		       \mathrm{Hom}_{\mathsf{D}} \left( F(M), B \right) \arrow[r, "", rightarrow] &
		       \mathrm{Hom}_{\mathsf{D}} \left( F(M), C \right)
	       \end{tikzcd}
	       \end{equation} 
	       which is exact, by exacntess of the original sequence.
       \item Let $I \in \mathrm{Ob} \left(\mathsf{D}\right)$ injective.
	       We have a functorial isomorphism
	       \begin{equation}
		       \mathrm{Hom}_{\mathsf{C}} \left( -, G(I) \right) \simeq
		       \mathrm{Hom}_{\mathsf{D}} \left( F(-), I \right)
	       .\end{equation} 
	       The last is exact since both $I$ is injective and $F$ is exact by hypothesis.
	       Then $G(I)$ is injective, i.e. $G$ preserves injectivity.\qedhere
\end{enumerate}
\end{proof}

\begin{cor}
	Let $f\colon X \to Y$ be a continuous map of topological spaces.
	Then $f_*\colon \mathsf{Sh}(X) \to \mathsf{Sh}\left(Y\right)$
	preserves injectives.
\end{cor} 
\begin{proof}
	We have the adjunction $\left( f^*, f_* \right)$.
	Since $f^*$ behaves well with stalks it is exact, hence $f_*$
	preserves injectives.
\end{proof}

\end{document}
