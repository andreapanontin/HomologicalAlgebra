\documentclass[../Main]{subfiles}
\begin{document}
\subsection{Functors between additive categories}
\begin{defn}[Additive functor]
	Let $\mathsf{C}, \mathsf{D}$ be (pre)additive categories.
	A functor $F\colon \mathsf{C} \to \mathsf{D}$ is said to be {\em additive} iff
	for any $A, B \in \mathrm{Ob} \left(\mathsf{C}\right)$, the map
	\begin{equation}
	F_{A,B}\colon \mathrm{Hom}_{\mathsf{C}} \left( A, B \right) \to \mathrm{Hom}_{\mathsf{D}} \left( FA, FB \right)
	\end{equation} 
	is a morphism of abelian group.
\end{defn}

%\begin{rem}[Recall]
%	An additive category is a preadditive category with finite products.
%	Actually an additive category can also be defined as a category with a zero object, binary product
%	and coproduct s.t. 
%	(tk: copy the definition)
%	\begin{enumerate}
%		\item There is an isomorphism $X \cross Y$
%	\end{enumerate}
%\end{rem}

\begin{rem}[]
	It would be natural to define an additive functor 
	\begin{align}
		F\colon \mathsf{C} &\longrightarrow \mathsf{D} 
	\end{align} 
	as simply a functor which preserves finite products, (since finite products and coproducts
	coincide and are the only required structure in an additive category).
	In fact such two definitions coincide in the case of an additive category.
	Not in a preadditive category though.
\end{rem}

\begin{rem}[]
	In the next proof we will use the following fact, true in any preadditive category $\mathsf{C}$:
	if $T \in \mathrm{Ob} \left(\mathsf{C}\right)$ is a final object, then for
	any other $X \in \mathrm{Ob} \left(\mathsf{C}\right)$, we have
	\begin{equation}
	X \cross T \simeq X
	.\end{equation} 
\end{rem}

\begin{lem}\label{lem:CharactAddFct}
	Let $F\colon \mathsf{C} \to \mathsf{D}$ be a functor between additive categories. TFAE:
	\begin{enumerate}
		\item $F$ is additive (in the sense of the definition).
		\item $F$ preserves finite products (in particular the final object).
		\item $F$ preserves the product of any pair of objects.
		\item $F$ preserves finite coproducts (in particular the initial object).
		\item $F$ preserves the coproduct of any pair of objects.
	\end{enumerate}
\end{lem} 
\begin{proof}\leavevmode\vspace{-.2\baselineskip}
	\begin{description}
		\item[2 $\iff$ 3:] one direction is obvious, let's prove the inverse.
			We want to procede by induction on the number of objects in
			the product, in case $n=3$ we use the associativity in the following way
			and generalize to arbitrary $n$:
				\begin{align}
					F(A_1 \cross A_2 \cross A_3) &=
					F((A_1 \cross A_2) \cross A_3) =
					F(A_1 \cross A_2) \cross F(A_3)\\
					&=
					F(A_1) \cross F(A_2) \cross F(A_3)
				.\end{align} 
				Moreover we need to prove that $F$ sends the empty product into
				the empty product, i.e. $F(0) = 0$:
				By definition of product we have
				\begin{equation}
					\mathrm{Hom}_{\mathsf{C}} \left( D, F(0) \cross F(A) \right)
					\simeq \mathrm{Hom}_{\mathsf{C}} \left( D, F(0) \right)
					\cross \mathrm{Hom}_{\mathsf{C}} \left( D, F(A) \right)
				.\end{equation} 
				Then, from the above remark we obtain that 
				\begin{equation}
					F(0 \cross A) = F(A)
				.\end{equation} 
				Then in the above we must have
				\begin{equation}
					\mathrm{Hom}_{\mathsf{C}} \left( D, F(0) \right) = \left\{ * \right\}
				\end{equation} 
				for all $D \in \mathrm{Ob} \left(\mathsf{D}\right)$,
				i.e. $F(0)$ is a final (hence zero in an additive category) object.

		\item[3 $\iff$ 5:] By additivity of $\mathsf{C}$ and $\mathsf{D}$ we obtain
				\begin{align}
					F(A \coprod B) \simeq F(A \cross B) \simeq
					F(A) \cross F(B) \simeq
					F(A) \coprod F(B)	
				.\end{align} 
				We also need to show that $F(i_A) = i_{F(A)}$.
				In fact we have $i_A := (id_A, 0)$,
				hence $F(i_A) = (id_{F(A)}, F(0)) = i_{F(A)}$, since
				$F$ preserves terminal objects.
				Then $\left(F(A \coprod B), F(i_A), F(i_B)\right)$
				is a coproduct in $\mathsf{D}$.
				The converse is the same, inverting product and coproduct.

		\item[2 $\implies$ 1:] Let $A, B \in \mathrm{Ob} \left(\mathsf{C}\right)$.
			Consider the zero map
			\begin{equation}
			\begin{tikzcd}
				A \arrow[r, "", rightarrow] &
				0_{\mathsf{C}} \arrow[r, "", rightarrow] &
				B
			\end{tikzcd}
			.\end{equation} 
			Then $F$ sends it into
			\begin{equation}
			\begin{tikzcd}
				F(A) \arrow[r, "", rightarrow] &
				0_{\mathsf{D}} \arrow[r, "", rightarrow] &
				F(B)
			\end{tikzcd}
			,\end{equation} 
			where we used the property we proved above, i.e. that
			a functor $F$ preserving finite products sends zero
			objects into zero objects.
			In other words $F$ preserves the zero map.

			Let's now show that it is compatible with addition.
			In fact we have seen, in lemma \ref{lem:MorphismSumAdditiveCat},
			that in additive categories we have
			\begin{equation}
				f + g = \delta_B \circ (f,g)
			,\end{equation} 
			where the above represents the diagram
			\begin{equation}
			\begin{tikzcd}
				f+g : A \arrow[r, "(f g)", rightarrow] &
				B \cross B \arrow[r, "", equals] &
				B \coprod B \arrow[r, "\delta_B", rightarrow] &
				B
			\end{tikzcd}
			.\end{equation} 
			As shown before $F$ preserves both finite products and coproducts, then
			\begin{equation}
			\begin{tikzcd}
				F(f+g): F(A) \arrow[r, "(F(f) F(g))", rightarrow] &
				F(B) \cross F(B) \arrow[r, "", equals] &
				F(B) \coprod F(B) \arrow[r, "F(\delta_B)", rightarrow] &
				F(B)
			\end{tikzcd}
			.\end{equation} 
			Here we need to notice that $F(\delta_B) = \delta_{F(B)}$, which
			is also a consequence of the fact that $F$ preserves finite coproducts.
			Then, by lemma \ref{lem:MorphismSumAdditiveCat} again,
			we obtain $F(f+g) = F(f) + F(g)$.

		\item[1 $\implies$ 3:]
			Consider $A, B \in \mathrm{Ob} \left(\mathsf{C}\right)$, then we know
			\begin{equation}
			id_{A \cross B} = i_A \circ p_A + i_B \circ p_B
			.\end{equation} 
			Being $F$ an additive functor (preserves identity, composition and
			group operation, i.e. sum) we obtain that
			\begin{equation}
				id_{F(A \cross B)} = F(id_{A \cross B}) =
				F(i_A) \circ F(p_A) + F(i_B) \circ F(p_B)
			.\end{equation} 
			Given any $D \in \mathrm{Ob} \left(\mathsf{D}\right)$
			and a pair of maps $f\colon D \to F(A)$
			and $g\colon D \to F(B)$, we want to prove
			that there is a unique $h\colon D \to F(A \cross B)$
			making the diagram commute
			\begin{equation}
			\begin{tikzcd}
				&
				F(A) \arrow[d, "F(i_A)", rightarrow, bend left] \\
				D \arrow[ru, "f", rightarrow] \arrow[rd, "g"', rightarrow] 
				\arrow[r, "\exists\, ! h", dashrightarrow] &
				F(A \cross B) \arrow[u, "F(p_A)" near start, rightarrow]
				\arrow[d, "F(p_B)"' near start, rightarrow] \\
				&
				F(B) \arrow[u, "F(i_B)"', rightarrow, bend right] 
			\end{tikzcd}
			.\end{equation} 
			Let's start by showing uniqueness, assuming it exists.
			Clearly
			\begin{align}
				h = id_{F(A \cross B)} \circ h &=
				\left( F(i_A) \circ F(p_A) + F(i_B) \circ F(p_B) \right) \circ h\\
				&=
				\left( F(i_A) \circ F(p_A) \circ h \right) + 
				\left( F(i_B) \circ F(p_B) \circ h \right) \nonumber\\
				&=
				F(i_A) \circ f + F(i_B) \circ g\nonumber
			,\end{align} 
			where the last equality is due to the commutativity of the diagram.
			Then $h$ is uniquely determined by $f$ and $g$, hence it is clearly unique.

			Let's now check that the above formula satisfies the conditions.
			If
			\begin{equation}
				h := F(i_A) \circ f + F(i_B) \circ g
			.\end{equation} 
			We need to show that the triangles commute, i.e.
			\begin{equation}
				F(p_A) \circ h = F(\underbrace{p_A \circ i_A}_{id_A}) \circ f
				+ F(\underbrace{p_A \circ i_B}_{0}) \circ g =
				f + 0 = f
			.\end{equation} 
			Analogously that
			\begin{equation}
				F(p_B) \circ h = F(\underbrace{p_B \circ i_A}_{0}) \circ f
				+ F(\underbrace{p_B \circ i_B}_{id_B}) \circ g =
				g + 0 = g
			.\end{equation} 
			Then $\left(F(A \cross B), F(p_A), F(p_B)\right)$
			is a product of $F(A)$ and $F(B)$.\qedhere
	\end{description} 
\end{proof}

\begin{defn}[]
	Let $\mathsf{C}$ and $\mathsf{D}$ be {\em abelian} categories.
	Let $F\colon \mathsf{C} \to \mathsf{D}$ be a functor.
	\begin{enumerate}
		\item 
	We say that $F$ is {\em left exact} iff, for any exact sequence
	\begin{equation}
	\begin{tikzcd}
		0 \arrow[r, "", rightarrow] &
		A \arrow[r, "", rightarrow] &
		B \arrow[r, "", rightarrow] &
		C
	\end{tikzcd}
	.\end{equation} 
	in $\mathsf{C}$, the image sequence is exact in $\mathsf{D}$
	\begin{equation}
	\begin{tikzcd}
		0 \arrow[r, "", rightarrow] &
		F(A) \arrow[r, "", rightarrow] &
		F(B) \arrow[r, "", rightarrow] &
		F(C)
	\end{tikzcd}
	.\end{equation} 
		\item 
	We say that $F$ is {\em right exact} iff, for any exact sequence
	\begin{equation}
	\begin{tikzcd}
		A \arrow[r, "", rightarrow] &
		B \arrow[r, "", rightarrow] &
		C \arrow[r, "", rightarrow] &
		0
	\end{tikzcd}
	.\end{equation} 
	in $\mathsf{C}$, the image sequence is exact in $\mathsf{D}$
	\begin{equation}
	\begin{tikzcd}
		F(A) \arrow[r, "", rightarrow] &
		F(B) \arrow[r, "", rightarrow] &
		F(C) \arrow[r, "", rightarrow] &
		0
	\end{tikzcd}
	.\end{equation} 
\item We say that $F$ is {\em exact} iff it is both left and right exact.
	In other words iff given any exact sequence
	\begin{equation}
	\begin{tikzcd}
		0 \arrow[r, "", rightarrow] &
		A \arrow[r, "", rightarrow] &
		B \arrow[r, "", rightarrow] &
		C \arrow[r, "", rightarrow] &
		0
	\end{tikzcd}
	.\end{equation} 
	in $\mathsf{C}$, the image sequence is exact in $\mathsf{D}$
	\begin{equation}
	\begin{tikzcd}
		0 \arrow[r, "", rightarrow] &
		F(A) \arrow[r, "", rightarrow] &
		F(B) \arrow[r, "", rightarrow] &
		F(C) \arrow[r, "", rightarrow] &
		0
	\end{tikzcd}
	.\end{equation} 
	\end{enumerate}
\end{defn}

\begin{lem}
	Let $F\colon \mathsf{C} \to \mathsf{D}$ be a left exact functor between
	abelian categories, then $F$ is additive.
\end{lem} 
\begin{proof}
	It's enough to show that $F$ preserves the coproduct of
	two objects, by lemma \ref{lem:CharactAddFct}.
	Let's consider the split exact sequence in $\mathsf{C}$:
	\begin{equation}
	\begin{tikzcd}
		0 \arrow[r, "", rightarrow] &
		A \arrow[r, "i_A", rightarrow] &
		A \oplus B \arrow[r, "p_B", rightarrow] &
		B \arrow[r, "", rightarrow] &
		0
	\end{tikzcd}
	.\end{equation} 
	Then, since $F$ is left exact, we obtain the exact sequence in $\mathsf{D}$:
	\begin{equation}
	\begin{tikzcd}
		0 \arrow[r, "", rightarrow] &
		F(A) \arrow[r, "F(i_A)", rightarrow] &
		F(A \oplus B) \arrow[r, "F(p_B)", rightarrow] &
		F(B)
	\end{tikzcd}
	.\end{equation} 
	Since the first sequence is split, we know that there is the section $i_B: B \to A \oplus B$,
	s.t. $p_B \circ i_B = id_B$.
	This implies that $F(i_B)$ is a section of $F(p_B)$, i.e.
	\begin{equation}
		F(p_B) \circ F(i_B) = id_B
	.\end{equation} 
	One can easily show that this condition implies that $F(p_B)$ is an epimorphism.
	Then the following is actually a split exact sequence:
	\begin{equation}
	\begin{tikzcd}
		0 \arrow[r, "", rightarrow] &
		F(A) \arrow[r, "F(i_A)", rightarrow] &
		F(A \oplus B) \arrow[r, "F(p_B)", rightarrow] &
		F(B) \arrow[r, "", rightarrow] &
		0
	\end{tikzcd}
	.\end{equation} 
	By the characterization of split exact sequences we know we have a commutative diagram
	\begin{equation}
	\begin{tikzcd}
		0 \arrow[r, "", rightarrow] &
		F(A) \arrow[r, "F(i_A)", rightarrow] \arrow[d, "", equals] &
		F(A \oplus B) \arrow[r, "F(p_B)", rightarrow] \arrow[d, "\varphi", rightarrow] &
		F(B) \arrow[r, "", rightarrow] \arrow[d, "", equals] &
		0\\
		0 \arrow[r, "", rightarrow] &
		F(A) \arrow[r, "i_{F(A)}", rightarrow] &
		F(A) \oplus F(A) \arrow[r, "p_{F(B)}", rightarrow] &
		F(B) \arrow[r, "", rightarrow] &
		0
	\end{tikzcd}
	.\end{equation} 
	Since the diagram commutes, $\varphi$ is an isomorphism, and by definition
	$\left(F(A) \oplus F(B), i_{F(A)}, i_{F(B)}\right)$ is a coproduct,
	it easily follows that $\left(F(A \oplus B), F(i_A), F(i_B)\right)$
	is a coproduct of $F(A)$ and $F(B)$.
\end{proof}

\begin{rem}[]
	For the sake of completeness let's show that
	\begin{equation}
	\begin{tikzcd}
		0 \arrow[r, "", rightarrow] &
		A \arrow[r, "i_A", rightarrow] &
		A \oplus B \arrow[r, "p_B", rightarrow] &
		B \arrow[r, "", rightarrow] &
		0
	\end{tikzcd}
	\end{equation} 
	is exact in $\mathsf{C}$.
	In fact it is exact on the left iff, for any $M \in \mathrm{Ob} \left(\mathsf{C}\right)$,
	the following
	\begin{equation}
	\begin{tikzcd}
		0 \arrow[r, "", rightarrow] &
	\mathrm{Hom}_{\mathsf{C}} \left( M, A \right) \arrow[r, "i_{A*}", rightarrow] &
		\mathrm{Hom}_{\mathsf{C}} \left( M, A \oplus B \right) \arrow[r, "p_{B*}", rightarrow] &
		\mathrm{Hom}_{\mathsf{C}} \left( M, B \right)
	\end{tikzcd}
	.\end{equation} 
	This is clearly exact in $\mathsf{Ab}$, since by universal property of product we have
	the following isomorphism:
	\begin{equation}
	\mathrm{Hom}_{\mathsf{C}} \left( M, A \oplus B \right) \simeq
	\mathrm{Hom}_{\mathsf{C}} \left( M, A \right) \cross 
	\mathrm{Hom}_{\mathsf{C}} \left( M, B \right)
	.\end{equation} 
	Finally the whole sequence is exact, since $p_B$ is an
	epimorphism (it has a section, but can be proven in other ways too).
\end{rem}

\begin{rem}[]
	A morphism which admits a right inverse is an epimorphism:
	Recall that $f$ is an epimorphism iff it is right erasable,
	i.e. for any pair of composable maps s.t.
	\begin{equation}
	\alpha \circ f = \beta \circ f
	\end{equation} 
	implies $\alpha = \beta$.
	But this is obvious, since
	\begin{equation}
	\alpha = \alpha \circ f \circ s = \beta \circ f \circ s = \beta
	,\end{equation} 
	in which we denoted by $s$ the section (right inverse) of $f$.
\end{rem}
\end{document}
