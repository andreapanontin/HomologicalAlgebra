\documentclass[../Main]{subfiles}
\begin{document}

\section{Derived functors}
In the following assume that $\mathsf{C}$ and $\mathsf{D}$ are abelian categories
s.t. $\mathsf{C}$ has enough injectives.
We now want to define the family of right derived functors associated to a left exact
functor $F\colon \mathsf{C} \to \mathsf{D}$.

\begin{defn}[Right derived functors]
	Let $F\colon \mathsf{C} \to \mathsf{D}$ be a left exact functors,
	we define the family of morphisms 
	\begin{equation}
	\begin{tikzcd}
		R^nF\colon \mathsf{C} \arrow[r, "", rightarrow] &
		\mathsf{D}
	\end{tikzcd}
	\end{equation} 
	in the following way:
	fix $A \in \mathrm{Ob} \left(\mathsf{C}\right)$, chose any injective resolution
	\begin{equation}
	\begin{tikzcd}
		A \arrow[r, "", rightarrow] &
		I^\bullet
	\end{tikzcd}
	.\end{equation} 
	Then applying $F$ to $I^\bullet$ one obtains the complex
	(which, in general, is no longer exact)
	\begin{equation}
	\begin{tikzcd}
		0 \arrow[r, "", rightarrow] &
		F(I^0) \arrow[r, "F(d^0)", rightarrow] &
		F(I^1) \arrow[r, "F(d^1)", rightarrow] &
		\ldots
	\end{tikzcd}
	\end{equation} 
	denoted by $F(I^\bullet) \in \mathrm{Ch}(\mathsf{D})$.
	One defines 
	\begin{equation}
		R^n F(A) := H^{n} \left( F(I^\bullet) \right)
	.\end{equation} 
\end{defn}
It is not obvious that this is a good definition, it could depend on the chosen injective resolution,
but we will shortly prove this is not the case.
Before doing so a couple of important remarks:

\begin{rem}[]
	From this definition we immediately obtain that $R^0 F \simeq F$.
	In fact $F$ is left exact, hence it preserves kernels.
	In particular, since $I^\bullet$ is zero at any negative degree,
	and since $A \to  I^\bullet$ is a quasi isomorphism, we obtain that
	\begin{equation}
	\ker d^0_I \simeq
	H^{0} \left( I^\bullet \right) \simeq 
	H^{0} \left( A \right) \simeq A
	.\end{equation} 
	Applying the definition of right derived functor we obtain
	\begin{equation}
		R^{0}F \left( A \right) := 
		H^{0} \left( F (I^\bullet) \right) \simeq 
		\ker (F d^0_I) \simeq 
		F (\ker d^0_I) \simeq F(A)
	.\end{equation} 
\end{rem}

\begin{rem}[Case of exact functor]
	Assume $F$ is exact.
	Then $R^nF = 0$ for all $n > 0$.
\end{rem}
\begin{proof}
	Consider the injective resolution
	\begin{equation}
	\begin{tikzcd}
		0 \arrow[r, "", rightarrow] &
		A \arrow[r, "", rightarrow] &
		I^0 \arrow[r, "", rightarrow] &
		I^1 \arrow[r, "", rightarrow] &
		\ldots
	\end{tikzcd}
	.\end{equation} 
	If we apply the functor $F$ we obtain an exact sequence in $\mathsf{D}$:
	\begin{equation}
	\begin{tikzcd}
		0 \arrow[r, "", rightarrow] &
		F(A) \arrow[r, "", rightarrow] &
		F(I^0) \arrow[r, "", rightarrow] &
		F(I^1) \arrow[r, "", rightarrow] &
		\ldots
	\end{tikzcd}
	.\end{equation} 
	Then computing the cohomology we obtain that, for all $n > 0$, 
	$H^n(F(I^\bullet)) = 0$.
\end{proof}
These above remarks will actually be useful later on, since we want to prove that
the family of morphism $\left( R^nF \right)_{n \in \mathbb{N}}$ is universal with
respect to the families with $R^0 F \simeq F$.
Morever they give us the intuition that the
right derived functors $(R^nF)_{n \in \mathbb{N}}$
measure how far is $F$ from being exact.

Let's now prove good definition and functoriality of the
construction of right derived functors:

\begin{prop}[Good definition of derived functors]
	Assume that $\mathsf{C}$ and $\mathsf{D}$ are abelian categories,
	and $\mathsf{C}$ has enough injectives.
	Consider a left exact functor $F\colon \mathsf{C} \to \mathsf{D}$.
\begin{enumerate}
	\item 
	Consider $A \in \mathrm{Ob} \left(\mathsf{C}\right)$
	and two injective resolutions
	\begin{equation}
	\begin{tikzcd}
		0 \arrow[r, "", rightarrow] &
		A \arrow[r, "", rightarrow] &
		I^\bullet &
		\text{and} &
		0 \arrow[r, "", rightarrow] &
		A \arrow[r, "", rightarrow] &
		J^\bullet
	\end{tikzcd}
	\end{equation} 
	of $A$.
	Then for all $n \geq 0$ there is a canonical isomorphism
	\begin{equation}
		H^n(F(I^\bullet)) \simeq
		H^n(F(J^\bullet))
	.\end{equation} 
\item
	Given injective resolutions of $A, B \in \mathrm{Ob} \left(\mathsf{X}\right)$
	\begin{equation}
	\begin{tikzcd}
		0 \arrow[r, "", rightarrow] &
		A \arrow[r, "", rightarrow] &
		I^\bullet &
		\text{and} &
		0 \arrow[r, "", rightarrow] &
		B \arrow[r, "", rightarrow] &
		J^\bullet
	\end{tikzcd}
	\end{equation} 
	and a morphism
	\begin{equation}
	\begin{tikzcd}
		f\colon A \arrow[r, "", rightarrow] &
		B,
	\end{tikzcd}
	\end{equation} 
	we can construct a canonical morphism
	\begin{equation}
	\begin{tikzcd}
		H^{n} \left( F(f^\bullet) \right)\colon
		H^n(F(I^\bullet)) \arrow[r, "", rightarrow] &
		H^n(F(J^\bullet))
	\end{tikzcd}
	.\end{equation} 
\item
	Given composable morphisms
	\begin{equation}
	\begin{tikzcd}
		A \arrow[r, "f", rightarrow] &
		B \arrow[r, "g", rightarrow] &
		C
	\end{tikzcd}
	\end{equation} 
	and respective resolutions $A \to I^\bullet, B \to J^\bullet$
	and $C \to K^\bullet$, then the traingle commutes:
	\begin{equation}
	\begin{tikzcd}
		H^{n} \left( F(I^\bullet) \right) 
		\arrow[rr, "H^{n} \left( F(g^\bullet \circ f^\bullet) \right)", rightarrow] 
		\arrow[rd, "H^{n} \left( F(f^\bullet) \right)"', rightarrow] & &
		H^{n} \left( F(K^\bullet) \right) \\
		&
		H^{n} \left( F(J^\bullet) \right) 
		\arrow[ru, "H^{n} \left( F(g^\bullet) \right)"', rightarrow] &
	\end{tikzcd}
	.\end{equation} 
\end{enumerate}
\end{prop}
\begin{proof}\leavevmode\vspace{-.2\baselineskip}
\begin{enumerate}
	\item[2.] Given the two resolutions, by proposition
		\ref{prop:ExistanceUniquenessLift}.
		there is a lift
		\begin{equation}
		\begin{tikzcd}
			f^\bullet\colon I^\bullet \arrow[r, "", rightarrow] &
			J^\bullet
		\end{tikzcd}
		\end{equation} 
		extending $f\colon A \to B$, unique up to homotopy.
		This induces, for all $n \in \mathbb{N}$ a map
		\begin{equation}
		\begin{tikzcd}
			H^n(F(f^\bullet))\colon H^n(F(I^\bullet)) \arrow[r, "", rightarrow] &
			H^n(F(J^\bullet))
		\end{tikzcd}
		.\end{equation} 
		Uniqueness up to homotopy means that any other lift of $f$
		\begin{equation}
		\begin{tikzcd}
			g^\bullet\colon I^\bullet \arrow[r, "", rightarrow] &
			J^\bullet
		\end{tikzcd}
		\end{equation} 
		satisfies $f \sim g$.
		More explicitly there exists a homotopy $\left\{ h^n \right\}_{n \in \mathbb{N}}$
		s.t.
		\begin{equation}
		f^n - g^n = d^{n-1} \circ h^n + h^{n+1} \circ d^n
		.\end{equation} 
		Since $F$ is left exact it is also additive,
		then it preserves sums and compositions, i.e.
		\begin{equation}
			F(f^n) - F(g^n) =
			F(d^{n-1}) \circ F(h^n) + F(h^{n+1}) \circ F(d^n)
		.\end{equation} 
		Which means that $F(f^\bullet) \sim F(g^\bullet)$, hence
		$F$ sends homotopic morphism of complexes
		to homotopic morphisms of complexes.
		Then $H^n(F(f)) = H^n(F(g))$ does not depend on the chosen extension of $f$.

	\item[3.] Let $f^\bullet$ and $g^\bullet$ be lifts of $f$ and $g$ respectively.
		Then $g^\bullet \circ f^\bullet$ clearly is an extension of $g \circ f$.
		Since $H^n$, for any $n \geq 0$, and $F$ are all functors, we have
		\begin{align}
			H^{n} \left( F(g^\bullet \circ f^\bullet) \right) = 
			H^{n} \left( F(g^\bullet) \circ F(f^\bullet) \right) = 
			H^{n} \left( F(g^\bullet)\right) \circ H^{n} \left( \circ F(f^\bullet) \right)
		.\end{align} 
	\item[1.] Given the two resolutions $A \to I^\bullet$ and $A \to B^\bullet$, 
		proposition \ref{prop:ExistanceUniquenessLift} grants that
		we can lift $id_A$ to two maps
		\begin{equation}
			\begin{tikzcd}[row sep=tiny]
			f^\bullet\colon I^\bullet \arrow[r, "", rightarrow] &
			J^\bullet\\
			g^\bullet\colon J^\bullet \arrow[r, "", rightarrow] &
			I^\bullet.
		\end{tikzcd}
		\end{equation} 
		Moreover, evidently, 
		$id_{I^\bullet}\colon I^\bullet \to I^\bullet$
		is and extension of $id_A$.
		By uniqueness up to homotopy, still from proposition
		\ref{prop:ExistanceUniquenessLift}, we obtain
		\begin{equation}
			H^{n} \left( F(g^\bullet \circ f^\bullet) \right) =
			H^{n} \left( F(id_{I^\bullet}) \right)
		.\end{equation} 
		Then functoriality grants
		\begin{equation}
			H^{n} \left( F(g^\bullet) \right) \circ H^{n} \left( F(f^\bullet) \right) =
			id_{H^{n} \left( F(I^\bullet) \right)}
		.\end{equation} 
		Arguing in the same way for $f^\bullet \circ g^\bullet$ one obtains that
		$H^{n} \left( F(f^\bullet) \right)$ and 
		$H^{n} \left( F(g^\bullet) \right)$ are isomorphism for all $n \geq 0$, hence
		\begin{equation}
			H^{n} \left( F(I^\bullet) \right) \simeq H^{n} \left( F(J^\bullet) \right)
		.\qedhere\end{equation} 
\end{enumerate}
\end{proof}

\begin{rem}[]
	This proposition grants that $R^{n}F$ is well defined, and moreover it is a functor.
	Actually this functor has quite a few interesting properties: let's investigate them.
\end{rem}

tk: decide whether to repeat the now correct definition of right derived functor.

\begin{prop}[]
	Let $F\colon \mathsf{C} \longrightarrow \mathsf{D}$ be a left exact functor between
	abelian categories, and let $\mathsf{C}$ have enough injectives.
	Then
	\begin{enumerate}
		\item $R^nF\colon \mathsf{C} \longrightarrow \mathsf{D}$ is additive for all $n \geq 0$.
		\item If $I \in \mathrm{Ob} \left(\mathsf{C}\right)$ is injective,
			then for all $n > 0$
			\begin{equation}
				R^nF(I) = 0
			.\end{equation} 
		\item Consider a short exact sequence in $\mathsf{C}$:
			\begin{equation}
			\begin{tikzcd}
				0 \arrow[r, "", rightarrow] &
				A \arrow[r, "", rightarrow] &
				B \arrow[r, "", rightarrow] &
				C \arrow[r, "", rightarrow] &
				0
			\end{tikzcd}
			.\end{equation} 
			Then there exist canonical maps
			$\delta^n\colon R^nF(C) \to R^{n+1}F(A)$ for all $n$
			s.t. the following is an exact sequence:
			\begin{equation}
			\begin{tikzcd}[row sep=tiny]
				0 \arrow[r, "", rightarrow] &
				R^{0}F \left( A \right) \arrow[r, "", rightarrow] &
				R^{0}F \left( B \right) \arrow[r, "", rightarrow]
				\arrow[d, phantom, ""{coordinate, name=Z}] &
				R^{0}F \left( C \right) \arrow[r, "\ \delta^0"', phantom, rightarrow]
				\arrow[dll, "", rounded corners, bend left = 100,
				to path={ -- ([xshift=2ex]\tikztostart.east)\tikztonodes
					|- (Z) [near end]
					-| ([xshift=-2ex]\tikztotarget.west)
					-- (\tikztotarget)}] &
					\phantom{a} \\
							     &
				R^{1}F \left( A \right) \arrow[r, "", rightarrow] &
				R^{1}F \left( B \right) \arrow[r, "", rightarrow]
				\arrow[d, phantom, ""{coordinate, name=Z}] &
				R^{1}F \left( C \right) \arrow[r, "\ \delta^1"', phantom, rightarrow]
				\arrow[dll, "", rounded corners, bend left = 100,
				to path={ -- ([xshift=2ex]\tikztostart.east)\tikztonodes
					|- (Z) [near end]
					-| ([xshift=-2ex]\tikztotarget.west)
					-- (\tikztotarget)}] &
					\phantom{a} \\
							     &
				R^{2}F \left( A \right)\arrow[r, "", rightarrow] &
				R^{2}F \left( B \right) \arrow[r, "", rightarrow] &
				R^{2}F \left( C \right) \arrow[r, "", rightarrow] &
				\ldots
			\end{tikzcd}
			.\end{equation}
			Moreover the above construction is functorial, i.e. given
			any morphism of short exact sequences in $\mathsf{C}$:
			\begin{equation}
			\begin{tikzcd}
				0 \arrow[r, "", rightarrow] &
				A \arrow[r, "", rightarrow] \arrow[d, "", rightarrow] &
				B \arrow[r, "", rightarrow] \arrow[d, "", rightarrow] &
				C \arrow[r, "", rightarrow] \arrow[d, "", rightarrow] &
				0\\
				0 \arrow[r, "", rightarrow] &
				A' \arrow[r, "", rightarrow] &
				B' \arrow[r, "", rightarrow] &
				C' \arrow[r, "", rightarrow] &
				0
			\end{tikzcd}
			,\end{equation} 
			one can define a morphism of long exact sequences:
			\begin{equation*}
			\begin{tikzcd}
				R^{0}F \left( A \right) \arrow[r, "", rightarrow] 
				\arrow[d, "", rightarrow] &
				R^{0}F \left( B \right) \arrow[r, "", rightarrow] 
				\arrow[d, "", rightarrow] &
				R^{0}F \left( C \right) \arrow[r, "\delta^0", rightarrow] 
				\arrow[d, "", rightarrow] &
				R^{1}F \left( A \right) \arrow[r, "", rightarrow] 
				\arrow[d, "", rightarrow] &
				\ldots \\
				R^{0}F \left( A' \right) \arrow[r, "", rightarrow] &
				R^{0}F \left( B' \right) \arrow[r, "", rightarrow] &
				R^{0}F \left( C' \right) \arrow[r, "\delta'^0", rightarrow] &
				R^{1}F \left( A' \right) \arrow[r, "", rightarrow] &
				\ldots
			\end{tikzcd}
			.\end{equation*} 
			In particular the following square commutes for all $n \geq 0$:
			\begin{equation}
			\begin{tikzcd}	
				R^{n}F \left( C \right) \arrow[r, "\delta^n", rightarrow] 
				\arrow[d, "", rightarrow] &
				R^{n+1}F \left( A \right) \arrow[d, "", rightarrow] \\
				R^{n}F \left( C' \right) \arrow[r, "\delta'^n"', rightarrow] &
				R^{n+1}F \left( A' \right)
			\end{tikzcd}
			.\end{equation} 
	\end{enumerate}
\end{prop}
\begin{proof}\leavevmode\vspace{-.2\baselineskip}
\begin{enumerate}
	\item It is enough to prove that $R^nF$ preserves preserves coproduct.
		To do so one need to check that given two resolutions
		\begin{equation}
		\begin{tikzcd}
				A \arrow[r, "", rightarrow] &
				I^\bullet &
				\qquad \text{ and } \qquad &
				B \arrow[r, "", rightarrow] &
				J^\bullet 
		\end{tikzcd}
		.\end{equation} 
		One can combine them to obtain the following resolution of $A \oplus B$:
		\begin{equation}
		\begin{tikzcd}
			A \oplus B \arrow[r, "", rightarrow] &
			I^\bullet \oplus J^\bullet
		\end{tikzcd}
		.\end{equation} 
		In particular the direct sum of two injective objects is still injective.
		More generally the direct sum of two functors is exact iff each of the two is.
		Moreover direct sum is an exact functor, hence the above is an exact sequence.
		Moreover, by left exactness of $F$, one obtains
		\begin{equation}
			F(I^\bullet \oplus J^\bullet) \simeq
			F(I^\bullet) \oplus F(J^\bullet)
		.\end{equation} 
		Then, since $H^n$ is an additive functor (it is defined as a cokernel),
		one has:
		\begin{align}
			R^{n}F \left( A \oplus B\right) &\simeq
		H^n(F(I^\bullet) \oplus F(J^\bullet)) \\
					&\simeq
		H^n(F(I^\bullet)) \oplus H^n(F(J^\bullet)) \simeq
		R^{n}F \left( A \right) \oplus R^{n}F \left( B \right)
		.\end{align} 

	\item Notice that the following is an injective resolution:
		\begin{equation}
		\begin{tikzcd}
			0 \arrow[r, "", rightarrow] &
			I \arrow[r, "\sim", rightarrow] &
			I \arrow[r, "", rightarrow] &
			0 \arrow[r, "", rightarrow] &
			\ldots
		\end{tikzcd}
		.\end{equation} 
		More explicitly we have
		\begin{equation}
		\begin{tikzcd}
			I^\bullet := I \arrow[r, "", rightarrow] &
			0 \arrow[r, "", rightarrow] &
			0 \arrow[r, "", rightarrow] &
			\ldots
		\end{tikzcd}
		.\end{equation} 
		Then, for all $n > 0$, $F(I^\bullet)^n = 0$, hence
		\begin{equation}
			R^{n}F \left( I \right) :=
			H^{n} \left( F(I^\bullet)\right) = 0
		.\end{equation} 
		In fact $F(I)$ appears at degree $0$.

	\item Consider the following injective resolutions for $A$ and $C$:
		\begin{equation}
		\begin{tikzcd}
				A \arrow[r, "", rightarrow] &
				I^\bullet &
				\qquad \text{ and } \qquad &
				C \arrow[r, "", rightarrow] &
				K^\bullet 
		\end{tikzcd}
		.\end{equation} 
		We now want to combine them into an injective resolution for $B$.
		Let's define $J^n := I^n \oplus K^n$. This is an injective object
		as remarked before.
		Moreover one can prove that in an additive category, the structural
		morphism for a product are epimorphisms, whereas the structural morphisms
		for coproducts are monomorphisms.
		Then, since direct sums in additive categories are both products and coproducts,
		one obtains, for all $n \geq 0$, the exact sequence:
		\begin{equation}
		\begin{tikzcd}
			0 \arrow[r, "", rightarrow] &
			I^n \arrow[r, "", rightarrow] &
			J^n \arrow[r, "", rightarrow] &
			K^n \arrow[r, "", rightarrow] &
			0.
		\end{tikzcd}
		\end{equation} 
		Then let's start at degree zero.
		We have the following commutative diagram:
		\begin{equation}
		\begin{tikzcd}
			&
			0 \arrow[d, "", rightarrow] & &
			0 \arrow[d, "", rightarrow] & \\
			0 \arrow[r, "", rightarrow] &
			A \arrow[r, "", rightarrow] \arrow[d, "", rightarrow] &
			B \arrow[r, "", rightarrow] \arrow[ld, "\alpha^0"', rightarrow] 
			\arrow[d, "", dashrightarrow] \arrow[rd, "\beta^0", rightarrow] &
			C \arrow[r, "", rightarrow] \arrow[d, "", rightarrow] &
			0 \\
			0 \arrow[r, "", rightarrow] &
			I^0 \arrow[r, "", rightarrow] &
			J^0 \arrow[r, "", rightarrow] &
			K^0 \arrow[r, "", rightarrow] &
			0
		\end{tikzcd}
		.\end{equation} 
		Here $\alpha^0$ exists since $I^0$ is injective and $A \to B$ is a mono.
		The map $\beta^0$, instead, is just the composite $B \to C \to K^0$.
		Then the dashed arrow is $\left(\alpha^0, \beta^0\right)$, defined
		viewing $J^0 = I^0 \oplus K^0$ as a product.
		Notice that the above diagram has both exact rows and columns.
		This implies that we can apply the snake lemma and obtain
		\begin{equation}
		\begin{tikzcd}[column sep=small]
			0 \arrow[r, "", rightarrow] &
			0 \arrow[r, "", rightarrow] &
			\ker \left(\alpha^0, \beta^0\right) \arrow[r, "", rightarrow] &
			0 \arrow[r, "", rightarrow] &
			I^0/A \arrow[r, "", rightarrow] &
			J^0/B \arrow[r, "", rightarrow] &
			K^0/C \arrow[r, "", rightarrow] &
			0.
		\end{tikzcd}
		\end{equation} 
		As a consequence we obtain that $B \to J^0$ is a monomorphism, and moreover
		we have the short exact sequence
		\begin{equation}
		\begin{tikzcd}
			0 \arrow[r, "", rightarrow] &
			I^0/A \arrow[r, "", rightarrow] &
			J^0/B \arrow[r, "", rightarrow] &
			K^0/C \arrow[r, "", rightarrow] &
			0.
		\end{tikzcd}
		\end{equation} 
		Now, recalling the construction in proposition \ref{prop:ExistanceUniquenessLift},
		we see that $d^0_I$ can be decomposed in $I^0 \twoheadrightarrow I^0/A \hookrightarrow I^1$.
		Analogously for $d^0_K$ and $d^n_I, d^n_K$, for all $n$, substituting $I^0/A$ with
		$\coker d^{n-1}$.
		Instead, for $d^n_J$ we will define it to be such composition.

		Then, combining the information we have obtained so far, we can
		construct the following commutative diagram with exact rows and columns:
		\begin{equation}
		\begin{tikzcd}
			&
			0 \arrow[d, "", rightarrow] & &
			0 \arrow[d, "", rightarrow] & \\
			0 \arrow[r, "", rightarrow] &
			I^0/A \arrow[r, "", rightarrow] \arrow[d, "", rightarrow] &
			J^0/B \arrow[r, "", rightarrow] \arrow[ld, "\alpha^1"', rightarrow] 
			\arrow[d, "", dashrightarrow] \arrow[rd, "\beta^1", rightarrow] &
			K^0/C \arrow[r, "", rightarrow] \arrow[d, "", rightarrow] &
			0 \\
			0 \arrow[r, "", rightarrow] &
			I^1 \arrow[r, "", rightarrow] &
			J^1 \arrow[r, "", rightarrow] &
			K^1 \arrow[r, "", rightarrow] &
			0
		\end{tikzcd}
		.\end{equation} 
		Again $\alpha^1$ exists since $I^1$ is injective, and the dashed arrow
		is constructed by universal property of products.
		Then, applying the snake lemma to the diagram, from exactness of the columns, we obtain
		that $J^0/B \to J^1$ is injective and
		\begin{equation}
		\begin{tikzcd}[column sep=small]
			0 \arrow[r, "", rightarrow] &
			\coker \left( I^0/A \to I^1 \right) \arrow[r, "", rightarrow] &
			\coker \left( J^0/B \to J^1 \right) \arrow[r, "", rightarrow] &
			\coker \left( K^0/C \to K^1 \right) \arrow[r, "", rightarrow] &
			0
		\end{tikzcd}
		.\end{equation} 
		Then we can put $d^0_J := J^0 \twoheadrightarrow J^0/B \hookrightarrow J^1$.
		Now it is important to notice that $\coker (\beta \circ \nu) \simeq \coker \nu$ for
		any composition of $\beta$ epi and $\nu$ mono, and we are going to prove it right
		after this theorem.
		Then, in our situation, we obtain 
		$\coker \left( I^0/A \to I^1 \right) \simeq \coker d^0_I$, and analogously for
		all sequences.
		This translates in the exact sequence
		\begin{equation}
		\begin{tikzcd}
			0 \arrow[r, "", rightarrow] &
			\coker \left( d^0_I\right) \arrow[r, "", rightarrow] &
			\coker \left( d^0_J \right) \arrow[r, "", rightarrow] &
			\coker \left( d^0_K \right) \arrow[r, "", rightarrow] &
			0
		\end{tikzcd}
		.\end{equation} 
		This, combined with injectivity of all the sequences and the above mentioned
		factorization $d^n = \coker d^{n-1} \circ \nu^n$ for a mono $\nu^n$,
		let's us procede by induction in this construction.
		
		Now one can observe that kernels and cokernels commute with direc sums, granting
		that $B \to J^\bullet$ is exact, hence it is an injective resolution of $B$.
		Finally we can notice that $F$ is left exact, hence additive.
		Then $F(J^\bullet) \simeq F(I^\bullet) \oplus F(K^\bullet)$ and this implies
		that $F$ sends split exact seqeunces to split exact sequences.
		As a conseguence the following is an exact sequence of complexes:
		\begin{equation}
		\begin{tikzcd}
			0 \arrow[r, "", rightarrow] &
			F(I^\bullet) \arrow[r, "", rightarrow] &
			F(J^\bullet) \arrow[r, "", rightarrow] &
			F(K^\bullet) \arrow[r, "", rightarrow] &
			0
		\end{tikzcd}
		.\end{equation} 
		Then the fundamental theorem in cohomology let's us construct the desired
		exact sequence in cohomology.
		Notice that $H^{-1} \left( F(K^\bullet) \right) = 0$, hence the zero at the beginning.
		Moreover this whole construction is functorial in
		short exact sequence, hence the last statements hold.\qedhere
\end{enumerate}
\end{proof}

\begin{lem}
	Consider a preadditive category $\mathsf{C}$ and $f$
	a morphism in $\mathsf{C}$.
	Assume that $f = \beta \circ \nu$, with $\beta$ epi and $\nu$ mono.
	Then $\ker f \simeq \ker \beta$ and $\coker f \simeq \coker \nu$.
\end{lem} 
\begin{proof}
	We'll prove it only for kernels, for cokernels one just needs to dualize the construction.
	Let $i\colon \ker f \to A$ be a kernel for $f$, we want to prove that
	it is also a kernel for $\beta$. In fact
	$0 = f \circ i = \nu \circ \beta \circ i$ implies that
	$\beta \circ i = 0$, since $\nu$ is mono.
	Moreover, for any $g\colon D \to A$ s.t. $\beta \circ g = 0$, by bilinearity of
	composition we obtain $f \circ g = \nu \circ \beta \circ g = 0$, hence
	$g$ factors uniquely through $\ker f$, i.e. $\ker f$ is a kernel for $\beta$.
\end{proof}

We will now show that the family $\left( R^{n}F \right)_{n \in \mathbb{N}}$ together
with $\left( \delta^n \right)_{n \in \mathbb{N}}$
satisfy a universal property among a class of similar functors.
This point of view goes back to Grothendieck.

\begin{defn}[Cohomological functor]
	Let $\mathsf{C}$ and $\mathsf{D}$ be abelian categories.
	We define a {\em cohomological} functor, sometimes called
	an exact $\delta$-functor, from $\mathsf{C}$ to $\mathsf{D}$,
	as:
	\begin{enumerate}
		\item a family of additive functors, for $n \in \mathbb{N}$,
			\begin{equation}
			\begin{tikzcd}
				T^n\colon \mathsf{C} \arrow[r, "", rightarrow] &
				\mathsf{D}
			\end{tikzcd}
			,\end{equation} 
		\item for all $n \geq 0$, and for all short exact sequences
			in $\mathsf{C}$
			\begin{equation}
			\begin{tikzcd}
				0 \arrow[r, "", rightarrow] &
				A \arrow[r, "", rightarrow] &
				B \arrow[r, "", rightarrow] &
				C \arrow[r, "", rightarrow] &
				0
			\end{tikzcd}
			\end{equation} 
			a family of morphisms
			\begin{equation}
			\begin{tikzcd}
				T^n(C) \arrow[r, "\delta^n", rightarrow] &
				T^{n+1}(A)
			\end{tikzcd}
			\end{equation} 
			making the following sequence exact in $\mathsf{D}$:
			\begin{equation}
		\begin{tikzcd}[row sep=tiny]
			T^{0} \left( A \right) \arrow[r, "", rightarrow] &
			T^{0} \left( B \right) \arrow[r, "", rightarrow]
			\arrow[d, phantom, ""{coordinate, name=Z1}] &
			T^{0} \left( C \right) \arrow[r, "\ \ \delta^0"', phantom, rightarrow]
			\arrow[dll, "", rounded corners, bend left = 100,
			to path={ -- ([xshift=2ex]\tikztostart.east)
				|- (Z1) [near end]\tikztonodes
				-| ([xshift=-2ex]\tikztotarget.west)
				-- (\tikztotarget)}] & 
			\phantom{a}\\
			T^{1} \left( A \right) \arrow[r, "", rightarrow] &
			T^{1} \left( B \right) \arrow[r, "", rightarrow]
			\arrow[d, phantom, ""{coordinate, name=Z2}] &
			T^{1} \left( C \right) \arrow[r, "\ \ \delta^1"', phantom, rightarrow]
			\arrow[dll, "", rounded corners, bend left = 100,
			to path={ -- ([xshift=2ex]\tikztostart.east)
				|- (Z2) [near end]\tikztonodes
				-| ([xshift=-2ex]\tikztotarget.west)
				-- (\tikztotarget)}] & 
			\phantom{a}\\
			T^{2} \left( A \right) \arrow[r, "", rightarrow] &
			\ldots &
		\end{tikzcd}
		.\end{equation}
			Moreover we require this construction to be functorial, i.e. for all
			morphisms of short exact sequences in $\mathsf{C}$
			\begin{equation}
			\begin{tikzcd}
				0 \arrow[r, "", rightarrow] &
				A \arrow[r, "", rightarrow] 
				\arrow[d, "", rightarrow] &
				B \arrow[r, "", rightarrow] 
				\arrow[d, "", rightarrow] &
				C \arrow[r, "", rightarrow] 
			  \arrow[d, "", rightarrow] &
				0\\
				0 \arrow[r, "", rightarrow] &
				A' \arrow[r, "", rightarrow] &
				B' \arrow[r, "", rightarrow] &
				C' \arrow[r, "", rightarrow] &
				0
			\end{tikzcd}
			\end{equation} 
			then, for all $n \geq 0$, the square commutes:
			\begin{equation}
			\begin{tikzcd}
				T^n(C) \arrow[r, "\delta^n", rightarrow] 
				\arrow[d, "", rightarrow] &
				T^{n+1}(A) \arrow[d, "", rightarrow] \\
				T^n(C') \arrow[r, "\delta^n", rightarrow] &
				T^{n+1}(A')
			\end{tikzcd}
			.\end{equation} 
	\end{enumerate}
\end{defn}
\begin{defn}[Morphism of cohomological functors]
	A morphism of cohomological functors is the data of
	a family of atural transformations:
	\begin{equation}
	\begin{tikzcd}
		(f^n)_{n \in \mathbb{N}}\colon
		\left(T^n, \delta^n\right)_{n \in \mathbb{N}} \arrow[r, "", rightarrow] &
		\left(T'^n, \delta'^n\right)
	\end{tikzcd}
	\end{equation} 
	s.t. for all $n \geq 0$, the following square commutes:
			\begin{equation}
			\begin{tikzcd}
				T^n(C) \arrow[r, "\delta^n", rightarrow] 
				\arrow[d, "f^n(C)", rightarrow] &
				T^{n+1}(A) \arrow[d, "f^{n+1}(A)", rightarrow] \\
				T^n(C') \arrow[r, "\delta'^n", rightarrow] &
				T^{n+1}(A')
			\end{tikzcd}
			.\end{equation} 
\end{defn}

\begin{rem}[]
	$T^0$ is not supposed to be left exact.
	Moreover we do not ask $\mathsf{C}$ to have enough injectives.
\end{rem}

\begin{defn}[Universal cohomological functor]
	Let $\mathsf{C}$ and $\mathsf{D}$ be abelian categories.
	Then a cohomological functor $\left(T^n, \delta^n\right)_{n \in \mathbb{N}}$
	is said to be universal iff
	for any $\left(T'^n, \delta'^n\right)_{n \in \mathbb{N}}$
	cohomological functor and
	any morphism of functors
	\begin{equation}
	\begin{tikzcd}
		g\colon T^0 \arrow[r, "", rightarrow] &
		T'^0
	\end{tikzcd}
	\end{equation}
	there exists a unique morphism of cohomological functors
	\begin{equation}
	\begin{tikzcd}
		f\colon T \arrow[r, "", rightarrow] &
		T'
	\end{tikzcd}
	\end{equation}
	s.t. $f^0 = g$.
\end{defn}

\begin{lem}
	If $F\colon \mathsf{C} \to \mathsf{D}$ is an additive functor, then
	there exists at most one universal cohomological functor $T$,
	up to canonical isomorphism,
	s.t. $T^0 \simeq F$.
\end{lem} 
\begin{proof}
	Consider $T$ and $T'$ cohomological functors satisfying the
	above condition.
	In particular $T^0 \simeq T'^0 \simeq F$.
	Since $T$ is universal, there is a unique morphism
	\begin{equation}
	\begin{tikzcd}
		f\colon T \arrow[r, "", rightarrow] &
		T'
	\end{tikzcd}
	\end{equation} 
	s.t. $f^0 = id_F$.
	Analogously there is a unique morphism
	\begin{equation}
	\begin{tikzcd}
		g\colon T' \arrow[r, "", rightarrow] &
		T
	\end{tikzcd}
	\end{equation} 
	s.t. $g^0 = id_F$.
	Then $g \circ f$ is the unique morphism of cohomological functors
	\begin{equation}
	\begin{tikzcd}
		g \circ f\colon T \arrow[r, "", rightarrow] &
		T
	\end{tikzcd}
	\end{equation} 
	s.t. $(g \circ f)^0 = g^0 \circ f^0 = id_F$.
	Then $id_T$ also has this property, hence $g \circ f = id_T$.
	Analogously we prove that $f \circ g = id_{T'}$ and we are done.
\end{proof}

\begin{defn}[]
	Let $\mathsf{C}$ and $\mathsf{D}$ be abelian categories.
	We say that an additive functor $F\colon \mathsf{C} \to \mathsf{D}$
	is {\em effaceable} iff
	for any $A \in \mathrm{Ob} \left(\mathsf{C}\right)$ there exists
	a monomorphism
	\begin{equation}
	\begin{tikzcd}
		i\colon A \arrow[r, "", rightarrowtail] &
		M
	\end{tikzcd}
	\end{equation} 
	s.t. $F(i) = 0$.
\end{defn}

\begin{prop}[]
	Suppose that $\mathsf{C}$ has enough injectives.
	Then $F\colon \mathsf{C} \to \mathsf{D}$ is effaceable iff
	$F(I) = 0$ for any injective $I \in \mathrm{Ob} \left(\mathsf{C}\right)$.
\end{prop}
\begin{proof}
\begin{description}
	\item[$\Leftarrow$]
		Let $A \in \mathrm{Ob} \left(\mathsf{C}\right)$, then
		there exist $I \in \mathrm{Ob} \left(\mathsf{C}\right)$
		injective with a mono
		\begin{equation}
		\begin{tikzcd}
			0 \arrow[r, "", rightarrow] &
			A \arrow[r, "i", rightarrow] &
			I
		\end{tikzcd}
		.\end{equation} 
		Then we obtain that
		\begin{equation}
		\begin{tikzcd}
			F(A) \arrow[r, "F(i)", rightarrow] &
			F(I) = 0
		\end{tikzcd}
		\end{equation} 
		implies that $F(i) = 0$.
	\item[$\Rightarrow$] Let $I \in \mathrm{Ob} \left(\mathsf{C}\right)$ injective.
		There is a mono
		\begin{equation}
		\begin{tikzcd}
			0 \arrow[r, "", rightarrow] &
			I \arrow[r, "i", rightarrow] &
			M
		\end{tikzcd}
		\end{equation} 
		s.t. $F(i) = 0$.
		By injectivity of $I$ we can obtain the commutative
		\begin{equation}
		\begin{tikzcd}
			I \arrow[r, "i", rightarrow] 
			\arrow[d, "id_I", rightarrow] &
			M \arrow[dl, "r", rightarrow] \\
			I &
		\end{tikzcd}
		.\end{equation} 
		This means that $r \circ i = id_I$.
		Then $F(r) \circ F(i) = if_{F(I)}$, i.e.
		$id_{F(I)} = 0$.
		Check that this implies that $F(I) = 0 \in \mathrm{Ob} \left(\mathsf{D}\right)$.
\end{description} 
\end{proof}

\begin{rem}[]
	Consider $F\colon \mathsf{C} \to \mathsf{D}$ a left exact functor, from
	$\mathsf{C}$ with enough injectives.
	Then $R^nF\colon \mathsf{C} \to \mathsf{D}$ is effaceable for all $n > 0$,
	since $R^{n}F \left( I \right) = 0$ for all injectives $I$.
\end{rem}


\begin{thm}[]
	Let $\mathsf{C}$ and $\mathsf{D}$ be abelian categories.
	let $T = \left(T^n, \delta^n\right)_{n \in \mathbb{N}}$ be a cohomological
	functor from $\mathsf{C}$ to $\mathsf{D}$.
	If $T^n$ is effaceable for all $n > 0$, then $T$
	is a universal functor.
\end{thm}


\begin{cor}
	Suppose that $\mathsf{C}$ has enough injectives,
	and $F\colon \mathsf{C} \to \mathsf{D}$ is a left exact functors.
	Then $\left(R^{n}F, \delta^n\right)_{n \in \mathbb{N}}$ is a universal
	cohomological functor.
\end{cor} 

\begin{rem}[]
	$\left(R^nF, \delta^n\right)_{n \in \mathbb{N}}$, together
	with $F \simeq R^0F$ is an {\em initial}
	object in the category of cohomological functors
	\begin{equation}
	\begin{tikzcd}
		T = \left(T^n, \delta^n\right)_{n \in \mathbb{N}}\colon \mathsf{C} 
		\arrow[r, "", rightarrow] &
		\mathsf{D}
	\end{tikzcd}
	\end{equation} 
	endowed with a morphism $f \to  T^0$.
\end{rem}
\begin{proof}
	$T^n$ is effaceable for all $n > 0$.
	Consider $T' := \left(T'^n, \delta'^n\right)_{n \in \mathbb{N}}$
	another cohomological functor
	and $g\colon T^0 \to T'^0$ be a natural transformation.
	Let's define by induction the family
	\begin{equation}
		\begin{tikzcd}
			f^n\colon T^n  \arrow[r, "", rightarrow] &
			T'^n
		\end{tikzcd}
	.\end{equation} 
	FOr $n = 0$ we already have the unique $f^0 = g$.
	Suppose we are given 
	\begin{equation}
		\begin{tikzcd}
			f^i\colon T^i  \arrow[r, "", rightarrow] &
			T'^i
		\end{tikzcd}
	\end{equation} 
	compatible with $\delta^i, \delta'^i$ for all $i \leq n$.
	Let's define $f^n$: consider $A \in \mathrm{Ob} \left(\mathsf{C}\right)$.
	Since $T^n$ is effaceable, there is a mono
	\begin{equation}
	\begin{tikzcd}
		0 \arrow[r, "", rightarrow] &
		A \arrow[r, "i", rightarrow] &
		M
	\end{tikzcd}
	\end{equation} 
	s.t. $T^n(i) = 0$.
	Then we can consider the exact sequence
	\begin{equation}
	\begin{tikzcd}
		0 \arrow[r, "", rightarrow] &
		A \arrow[r, "", rightarrow] &
		M \arrow[r, "", rightarrow] &
		M/A \arrow[r, "", rightarrow] &
		0
	\end{tikzcd}
	.\end{equation} 
	Since $T$ is a cohomological functor
	and by definition of $f^i$ we get the commutative diagram with exact rows:
	\begin{equation}
	\begin{tikzcd}
		0 \arrow[r, "", rightarrow] &
		T^{n}(M) \arrow[r, "", rightarrow] 
		\arrow[d, "f^n(M)", rightarrow] &
		T^{n}(M) \arrow[r, "", rightarrow] 
		\arrow[d, "f^n(M/A)", rightarrow] &
		T^{n}(M) \arrow[r, "", rightarrow] 
		\arrow[d, "\exists !\, f^{n+1}(A)", dashrightarrow] &
		0\\
		T^{n}(A) \arrow[r, "", rightarrow] 
						  \arrow[rr, "0"', rightarrow, bend right] &
		T^{n}(M) \arrow[r, "", rightarrow] &
		T^{n}(M) \arrow[r, "", rightarrow] &
		T^{n}(M) \arrow[r, "", rightarrow] &
		0
	\end{tikzcd}
	.\end{equation} 
	Notice that $f^{n+1}(A)$ exists by universal property
	of cokernels.
	Let's check that $f^{n+1}(A)$ does not depend on $M$:
	Let
	\begin{equation}
	\begin{tikzcd}
		0 \arrow[r, "", rightarrow] &
		A \arrow[r, "i'", rightarrow] &
		M'
	\end{tikzcd}
	\end{equation} 
	with $F(i') = 0$.
	Take the pushout
	\begin{equation}
	\begin{tikzcd}
		A \arrow[r, "i", rightarrowtail] 
		\arrow[d, "i'", rightarrowtail] &
		M \arrow[d, "", rightarrow] \\
		M' \arrow[r, "", rightarrow] &
		M \coprod_A M'
	\end{tikzcd}
	.\end{equation} 
	Then we have
	\begin{equation}
		\begin{tikzcd}
			A &
	M \arrow[d, "\alpha", rightarrowtail] \\
	&
	M \coprod_A M'
		\end{tikzcd}
	,\end{equation}
	which gives (see remarkable):
\end{proof}

\begin{defn}[]
	Consider $F\colon \mathsf{C} \to \mathsf{D}$ a left exact functor between
	abelian categories.
	Assume that $\mathsf{C}$ has enough injectives.
	We say that $C \in \mathrm{Ob} \left(\mathsf{C}\right)$ is $F$-acyclic iff,
	for all $n \geq 1$,
	\begin{equation}
		R^n(F)(C) = 0_{\mathsf{D}}.
	\end{equation}
\end{defn}
\begin{rem}[]
	Any $I \in \mathrm{Ob} \left(\mathsf{C}\right)$ injective is 
	$F$-acyclic for any left exact functor.
\end{rem}

Let's define $F$-acyclic resolutions, hoping to use the to compute
derived functors.

\begin{defn}[$F$-acyclic resolution]
	Let $A \in \mathrm{Ob} \left(\mathsf{C}\right)$.
	An $F$-acyclic resolution of $A$ is an exact sequence
	\begin{equation}
	\begin{tikzcd}
		0 \arrow[r, "", rightarrow] &
		A \arrow[r, "", rightarrow] &
		C^0 \arrow[r, "", rightarrow] &
		C^1 \arrow[r, "", rightarrow] &
		\ldots
	\end{tikzcd}
	.\end{equation} 
	s.t. $C^i$ is $F$-acyclic for all $i \geq 0$.
\end{defn}

\begin{prop}[]
	Let $A \in \mathrm{Ob} \left(\mathsf{C}\right)$.
	Consider an $F$-acyclic resolution and an injective resolution
	\begin{equation}
	\begin{tikzcd}
		0 \arrow[r, "", rightarrow] &
		A \arrow[r, "", rightarrow] &
		C^\bullet
	\end{tikzcd}
	\qquad \text{ and } \qquad
	\begin{tikzcd}
		0 \arrow[r, "", rightarrow] &
		A \arrow[r, "", rightarrow] &
		I^\bullet
	\end{tikzcd}
	.\end{equation} 
	Then there is a morphism of complexes $C^\bullet \to  I^\bullet$
	extending the identity of $id_A\colon A \to A$,
	which is unique up to homotopy.
\end{prop}
\begin{proof}
	COnsider the morphism $F(C^\bullet) \to  F(I^\bullet)$, in $\mathrm{Ch}(\mathsf{D})$.
	It is a quasi isomorphism. 
	In other words it acts as an iso at the level of cohomologies:
	for all $n \geq 0$
	\begin{equation}
		H^{n} \left( F(C^\bullet) \right) \simeq
		H^{n} \left( F(I^\bullet) \right) = R^{n}F \left( A \right)
	.\end{equation} 
\end{proof}

\subsection{Spectral sequences}

\section{Cohomology of sheaves}
Let $X \in \mathsf{Top}$.
We denote by $\mathsf{PSh}\left( X \right)$ the category of abelian presheaves and
by $\mathsf{Sh}\left( X \right)$ the category of abelian sheaves.

\begin{prop}[]
	Limits and colimits in $\mathsf{PSh}\left( X \right)$ are computed
	component-wise.
	More explicitly, for a functor $\beta\colon \mathsf{I} \to \mathsf{PSh}\left( X \right)$,
	from a small category $\mathsf{I}$, then we have
	\begin{align}
		\left( \varinjlim \beta \right) \left( \mathcal{U} \right) &\simeq
		\varinjlim \left( \beta(\mathcal{U}) \right)\\
		\left( \varprojlim \beta \right) \left( \mathcal{U} \right) &\simeq
		\varprojlim \left( \beta(\mathcal{U}) \right)\\
	.\end{align} 
	In particular arbitrary limits and colimits exist in $\mathsf{PSh}\left( X \right)$,
	since they do in $\mathsf{Ab}$.
\end{prop}
\begin{proof}
	Just a sketch:
\end{proof}
\begin{cor}
	$\mathsf{PSh}\left( X \right)$ is an abelian category, since it can be expressed
	in terms of limits and colimits.
\end{cor} 
\begin{proof}
	The zero presheaf is a zero object in the category of presheaves, by the above proposition.
	Then, again thanks to the above, one can define the coproduct and products.
	Then one obtains a morphism
	\begin{equation}
	\begin{tikzcd}
		F \coprod G \arrow[r, "\sim", rightarrow] &
		F \prod G,
	\end{tikzcd}
	\end{equation} 
	which is an iso, because it is sectionwise.
	Again one checks that the map
	\begin{equation}
	\begin{tikzcd}
		\mathrm{coim}\, f \arrow[r, "\sim", rightarrow] &
		\ima f
	\end{tikzcd}
	\end{equation} 
	is an isomorphism, since it is sectionwise.
\end{proof}

\begin{cor}
	The category $\mathsf{Sh}\left(X\right)$ is an abelian category.
\end{cor} 
\begin{proof}
	Recall the adjunction $\left(\iota, (-)^\#\right)$
	\begin{equation}
	\begin{tikzcd}
		\mathsf{PSh}\left( X \right) \arrow[r, "", rightarrow] &
		\mathsf{Sh}\left(X\right)
	\end{tikzcd}
	\end{equation} 
	s.t. $\left( \iota\mathcal{F} \right)^\# \simeq \mathcal{F}$.
	Recall that left adjoints preserve limits and right adjoint preserve colimits.
	Actually one functor is a right adjoint iff it preserves colimits.
	Consider a functor
	\begin{align}
		\phi\colon \Lambda &\longrightarrow \mathsf{Sh}\left(X\right) \\
		\lambda &\longmapsto F_\lambda \nonumber
	.\end{align} 
	Then $\phi \circ \iota$ sends $\lambda \mapsto \iota F_\lambda$.
	Here we can compute limits, and one can check that
	$\varprojlim \iota F_\lambda$ is already a sheaf (since the definition of
	sheaf only requires limits, hence should commute).
	Then 
	\begin{equation}
	\varprojlim F_\lambda :=  
	\left( \varprojlim \iota F_\lambda \right)^\#	
	\in \mathsf{Sh}\left(X\right) 
	\end{equation} 
	is the limit of
	the considered family.
	It follows that $\left( 0_{\mathsf{PSh}\left( X \right)} \right)^\# = 0_{\mathsf{Sh}\left(X\right)}$
	is a zero object (it is both initial and final, we are computing limits and colimits
	on appropriate categories).
	Then one can define 
	\begin{equation}
		\left( \iota \mathcal{F} \oplus \iota \mathcal{G} \right)^\# \in \mathsf{Sh}\left(X\right)
	\end{equation} 
	which is both a product and a coproduct in $\mathsf{Sh}\left(X\right)$ by the above remark.
	The same holds for cokernels and cokernel:
	\begin{equation}
		\left( \ker \iota f \right)^\# = \ker \left( \left( \iota f \right)^\# \right) = \ker f
	,\end{equation} 
	since $\left( - \right)^\# \circ \iota = id_{\mathsf{Sh}\left(X\right)}$.
	Finally, since $\mathsf{Sh}\left(X\right) \subset \mathsf{PSh}\left( X \right)$
	is a full subcategry and $\mathsf{PSh}\left( X \right)$ is abelian, then the hom
	set is always an abelian group.

	For the parallel morphism is always the same thing.
\end{proof}

\begin{rem}[]
	One can prove this result in other way.
\end{rem}

\begin{prop}[]
	Let $X \in \mathsf{Top}$ and $\mathcal{U} \subset X$ an open subset.
	Then the functor
	\begin{align}
		\Gamma_{\mathcal{U}}\colon \mathsf{Sh}\left(X\right) &\longrightarrow \mathsf{Ab} \\
		\mathcal{F} &\longmapsto \mathcal{F}(\mathcal{U}) \nonumber
	\end{align} 
	is {\em left} exact.
\end{prop}
\begin{proof}
	Consider an exact sequence in $\mathsf{Sh}\left(X\right)$
	\begin{equation}
	\begin{tikzcd}
		0 \arrow[r, "", rightarrow] &
		\mathcal{F}' \arrow[r, "\alpha", rightarrow] &
		\mathcal{F} \arrow[r, "\beta", rightarrow] &
		\mathcal{F}''
	\end{tikzcd}
	.\end{equation} 
	Then this induces an exact sequence in $\mathsf{Ab}$, at the level of stalks, for
	all $x \in X$:
	\begin{equation}
	\begin{tikzcd}
		0 \arrow[r, "", rightarrow] &
		\mathcal{F}'_x \arrow[r, "\alpha_x", rightarrow] &
		\mathcal{F} \arrow[r, "\beta_x", rightarrow] &
		\mathcal{F}''_x
	\end{tikzcd}
	.\end{equation} 
	In particular we want to look at sections, i.e. prove exactness of
	\begin{equation}
	\begin{tikzcd}
		0 \arrow[r, "", rightarrow] &
		\mathcal{F}'(\mathcal{U}) \arrow[r, "\alpha(\mathcal{U})", rightarrow] &
		\mathcal{F}(\mathcal{U}) \arrow[r, "\beta(\mathcal{U})", rightarrow] &
		\mathcal{F}''(\mathcal{U})
	.\end{tikzcd}
	\end{equation} 
	We have proved that $\alpha(\mathcal{U})$ is injective.
	We are left to prove that
	\begin{equation}
		\ima \left( \alpha(\mathcal{U}) \right) \simeq
		\ker \left( \beta(\mathcal{U}) \right)
	.\end{equation} 
	Since $\beta(\mathcal{U}) \circ \alpha(\mathcal{U}) = 0$ we obtain
	that $\ima \left( \alpha(\mathcal{U}) \right) \subset \ker \left( \beta(\mathcal{U}) \right)$.
	Consider now $s \in \ker \beta(\mathcal{U}) \subset \mathcal{F}(\mathcal{U})$.
	Let $x \in \mathcal{U}$, then $s_x = \left[ \mathcal{U}_x, s \right] \in \ker \beta_x$.
	This, by exacntess of the sequence at the level of stalks, implies that
	there is $s'_x \in \mathcal{F}'_{x}$ s.t. $\alpha_x(s'_x) = s_x$.
	In particualr there is an open subset $x \in \mathcal{U}_x$ s.t.
	$s'_x = \left[ \mathcal{U}_x, s' \right]$ and
	\begin{equation}
		\left.s\right|_{\mathcal{U}_x} = \alpha(\mathcal{U}_x)(s')
	.\end{equation} 
	Then we can find an open covering $\mathcal{U} = \bigcup_{i \in I} \mathcal{U}_i$
	and $s'_i \in \mathcal{F}'(\mathcal{U}_i)$ s.t.
	\begin{equation}
		\alpha(\mathcal{U}_i)(s'_i) = \left.s\right|_{\mathcal{U}_i} 
	.\end{equation} 
	Let's look at intersections:
	\begin{equation}
		\alpha(\mathcal{U}_{ij})(\left.s'_i\right|_{\mathcal{U}_{ij}}) =
			\left.s\right|_{\mathcal{U}_{ij}} =
				\alpha(\mathcal{U}_{ij})(\left.s'_j\right|_{\mathcal{U}_{ij}}) =
	.\end{equation} 
	By injectivity of $\alpha(\mathcal{U}_{ij})$
	we obtain that the family $\left\{ s'_i \right\}_{i \in I}$ is
	compatible on intersections, hence by sheaf properties one
	can find a unique $s' \in \mathcal{F}'(\mathcal{U})$ s.t.
	$\left.s'\right|_{\mathcal{U}_i} = s'_i$ for all $i$.
	Then
	\begin{equation}
		\alpha(\mathcal{U})(s')|_{\mathcal{U}_i} =
		\alpha(\mathcal{U}_i)(\left.s'\right|_{\mathcal{U}_i} ) =
		\alpha(\mathcal{U}_i)(s'_i)
		\left.s\right|_{\mathcal{U}_i} 
	.\end{equation} 
	By sheaf properties of $\mathcal{F}$ we obtain that $\alpha(\mathcal{U})(s') = s$.
\end{proof}
\begin{rem}[Another proof]
	One can factor the above functor by
	\begin{equation}
	\begin{tikzcd}
		\mathsf{Sh}\left(X\right) \arrow[rr, "\Gamma_{\mathcal{U}}", rightarrow] 
		\arrow[rd, "\iota", rightarrow] & &
		\mathsf{Ab}\\
		&
		\mathsf{PSh}\left( X \right) \arrow[ru, "\widetilde{\Gamma}_{\mathcal{U}}", rightarrow] &
	\end{tikzcd}
	.\end{equation} 
	One can conclude since, by construction, $\widetilde{\Gamma}_{\mathcal{U}}$ is
	exact (hence left exact), whereas $\iota$ is a right adjoint,
	hence a left exact functor.
\end{rem}

\begin{prop}[]
	$\mathsf{Sh}\left(X\right)$ has enough injectives.
\end{prop}

\begin{defn}[Cohomology]
	Let $\mathcal{F} \in \mathsf{Sh}\left(X\right)$.
	The cohomology groups of $X$, with coefficients in $\mathcal{F}$,
	are defined as
	\begin{equation}
	H^{n} \left( X, \mathcal{F} \right) := R^{n} \Gamma_X \left( \mathcal{F} \right)
	\end{equation} 
	for all $n \geq 0$.
	More generally, given $\mathcal{U} \subset X$ open, one defines:
	\begin{equation}
	H^{n} \left( \mathcal{U}, \mathcal{F} \right) := R^{n} \Gamma_{\mathcal{U}} \left( \mathcal{F} \right)
	.\end{equation} 
\end{defn}

\begin{rem}[]
	The cohomology of a t.s. $X$ with coefficients in $\mathcal{F}$,
	$H^{n} \left( X, \mathcal{F} \right)$ is defined as the homology of the complex
	\begin{equation}
	\begin{tikzcd}
		0 \arrow[r, "", rightarrow] &
		I^0 \arrow[r, "", rightarrow] &
		I^1 \arrow[r, "", rightarrow] &
		\ldots
	\end{tikzcd}
	,\end{equation} 
	for an injective resolution $0 \to \mathcal{F} \to I^\bullet$ of $\mathcal{F}$.
	One could actually choose to take the cohomology of
	a $\Gamma_{\mathcal{U}}$-acyclic resolution of $\mathcal{F}$
	\begin{equation}
	\begin{tikzcd}
		0 \arrow[r, "", rightarrow] &
		C^0 \arrow[r, "", rightarrow] &
		C^1 \arrow[r, "", rightarrow] &
		\ldots
	\end{tikzcd}
	.\end{equation} 
\end{rem}

\begin{defn}[Flasque sheaf]
	A sheaf $\mathcal{F} \in \mathsf{Sh}\left(X\right)$ is said to 
	be {\em flasque} or {\em falbby} iff, for all $\mathcal{V} \subset \mathcal{U} \subset X$
	open subsets, then restriction map
	\begin{equation}
	\begin{tikzcd}
		\mathcal{F}(\mathcal{U}) \arrow[r, "\rho^{\mathcal{F}}_{\mathcal{UV}}", rightarrow] &
		\mathcal{F}(\mathcal{V})
	\end{tikzcd}
	\end{equation} 
	is an epimorphism.
\end{defn}
\begin{rem}[]
	We shall prove that flasque sheaves are $\Gamma_X$-acyclic.
	Moreover each sheaf admits a falsque resolution.
\end{rem}

\begin{prop}[]
	Consider an exact sequence in $\mathsf{Sh}\left(X\right)$
	\begin{equation}
	\begin{tikzcd}
		0 \arrow[r, "", rightarrow] &
		\mathcal{F}' \arrow[r, "", rightarrow] &
		\mathcal{F} \arrow[r, "", rightarrow] &
		\mathcal{F}'' \arrow[r, "", rightarrow] &
		0
	\end{tikzcd}
	,\end{equation} 
	where $\mathcal{F}'$ is flasque.
	Then, for all $\mathcal{U} \subset X$ open, the sequence
	\begin{equation}
	\begin{tikzcd}
		0 \arrow[r, "", rightarrow] &
		\mathcal{F}'(\mathcal{U}) \arrow[r, "", rightarrow] &
		\mathcal{F}(\mathcal{U}) \arrow[r, "", rightarrow] &
		\mathcal{F}''(\mathcal{U}) \arrow[r, "", rightarrow] &
		0
	\end{tikzcd}
	\end{equation} 
	is still exact.
\end{prop}

\begin{rem}
	Assuming the above remark one obtains the following exact sequence
	\begin{equation}
	\begin{tikzcd}
		0 \arrow[r, "", rightarrow] &
		H^{0} \left( \mathcal{U}, \mathcal{F}' \right) \arrow[r, "", rightarrow] &
		H^{0} \left( \mathcal{U}, \mathcal{F} \right) \arrow[r, "", rightarrow] &
		H^{0} \left( \mathcal{U}, \mathcal{F}'' \right) \arrow[r, "", rightarrow] & 
		H^{1} \left( \mathcal{U}, \mathcal{F}' \right) \arrow[r, "", rightarrow] &`
		\ldots
	\end{tikzcd}
	\end{equation} 
	and $H^{n} \left( \mathcal{U}, \mathcal{F} \right) = 0$ for all $n$.
\end{rem}

\begin{proof}
	\begin{enumerate}
	\item {\em Let $s'' \in \mathcal{F}''(\mathcal{U})$.}
		{\em Consider the set $\mathcal{S}$ defined by}
		\begin{equation}
		\mathcal{S} := \left\{ \left(  \mathcal{U}_i, s_i\right) \ \middle|\
		\mathcal{U}_i \subset \mathcal{U} \text{ is open, }
		s_i \in \mathcal{F}(\mathcal{U}_i) \text{ s.t. }
		g(\mathcal{U}_i)(s_i) = \left.s''\right|_{\mathcal{U}_i} \right\}
		.\end{equation}
		{\em Define a partial order on $\mathcal{S}$ by}
		\begin{equation}
		\left(\mathcal{U}_i, s_i\right) \leq \left( \mathcal{U}_j, s_j\right) \iff
		\mathcal{U}_i \subset \mathcal{U}_j
		\text{ and }
		\left.s_j\right|_{\mathcal{U}_i} = s_i
		.\end{equation}
		{\em Show that $\mathcal{S}$ admits a maximal element.}

		Consider a totally ordered subset
		$T = \left\{ \left(\mathcal{U}_i, s_i\right) \right\}_{i \in I} \subset \mathcal{S}$.
		We now want to apply Zorn's lemma to conclude, hence we need to construct an
		upper bound for $T$ in $\mathcal{S}$.
		Then we define $\mathcal{V} := \bigcup_{i \in I} \mathcal{U}_i$, an open subset of
		$X$ (union of open subsets of $X$).
		Moreover $\mathcal{V}$ is covered by $\mathcal{U}_i$. Then we can define
		$s \in \mathcal{F}(\mathcal{V})$ as the unique section s.t.
		$\left.s\right|_{\mathcal{U}_i} = s_i$ by the gluing properties of the sheaf $\mathcal{F}$.
		This can be done since $T$ is totally ordered, hence for any $i,j$ either
		$\mathcal{U}_i \subset \mathcal{U}_j$ or the inverse. WLOG we assume the latter,
		hence $\mathcal{U}_i \cap \mathcal{U}_j = \mathcal{U}_j$.
		Then, since $T$ is totally ordered, we have $\left.s_i\right|_{\mathcal{U}_j} = s_j$,
		hence the family $\left\{ s_i \right\}_{i \in I}$ defines a family compatible
		on intersections and we can glue it to $s$.

		We are only left to prove that $g(\mathcal{V})(s) = \left.s''\right|_{\mathcal{V}}$.
		Also this follows from the gluing properties of sheaves, in fact we know that,
		for each $i$,
		\begin{equation}
			\left.g(\mathcal{V})(s)\right|_{\mathcal{U}_i} =
			g(\mathcal{U}_i)(\left.s\right|_{\mathcal{U}_i}) =
			g(\mathcal{U}_i)(s_i) =
			\left.s''\right|_{\mathcal{U}_i} =
			\left.\left(\left.s''\right|_{\mathcal{V}}\right) \right|_{\mathcal{U}_i}
		.\end{equation}
		Then $g(\mathcal{V})(s)$ and $\left.s''\right|_{\mathcal{V}}$
		coincide on an open cover of $\mathcal{V}$, hence they coincide on $\mathcal{V}$
		($\mathcal{F}''$ is a sheaf).

		We have then constructed an upper bound to $T$ in $\mathcal{S}$, then we can
		apply Zorn's lemma and conclude that $\mathcal{S}$ admits a maximal element.


	\item {\em Assume that $\mathcal{F}'$ is flasque. Show that, for any $\mathcal{U} \subset X$}
		{\em open subset, the sequence of abelian group}
		\begin{equation}
		\begin{tikzcd}
			0 \arrow[r, "", rightarrow] &
			\mathcal{F}'(\mathcal{U}) \arrow[r, "f(\mathcal{U})", rightarrow] &
			\mathcal{F}(\mathcal{U}) \arrow[r, "g(\mathcal{U})", rightarrow] &
			\mathcal{F}''(\mathcal{U}) \arrow[r, "", rightarrow] &
			0
		\end{tikzcd}
		\end{equation}
		{\em is exact.}

		As proven in exercise $2$ the functor $\Gamma_{\mathcal{U}}$ is left exact,
		then exactness of
		\begin{equation}
		\begin{tikzcd}
			0 \arrow[r, "", rightarrow] &
			\mathcal{F}'(\mathcal{U}) \arrow[r, "f(\mathcal{U})", rightarrow] &
			\mathcal{F}(\mathcal{U}) \arrow[r, "g(\mathcal{U})", rightarrow] &
			\mathcal{F}''(\mathcal{U})
		\end{tikzcd}
		\end{equation}
		follows from that.
		We are only left to prove that $g(\mathcal{U})$ is an epimorphism
		in $\mathsf{Ab}$, i.e. that it is surjective.
		Let's now use the previous point: fix $s'' \in \mathcal{F}''(\mathcal{U})$, then
		the corresponding set $\mathcal{S}$ admits a maximal element
		\begin{equation}
		\left( \mathcal{V}, s\right)
		.\end{equation}
		Our aim is to prove that $\mathcal{V} = \mathcal{U}$, in such case we have
		$g(\mathcal{U})(s) = s''$, hence surjectivity of $g(\mathcal{U})$, since we have not
		imposed any restrictions on $s'' \in \mathcal{F}''(\mathcal{U})$.

		Assume, by contradiction, that this is not the case,
		i.e. that there exists $x \in \mathcal{U} \setminus \mathcal{V}$.
		Since $g$ is an epimorphism, we obtain that there is $t_x \in \mathcal{F}_x$
		s.t.
		\begin{equation}
		\begin{tikzcd}
			t_x \arrow[r, "g_x", mapsto] & s''_x
		\end{tikzcd}
		.\end{equation}
		More explicitly there exist $\mathcal{U}_x \subset X$ open, and
		$t \in \mathcal{F}(\mathcal{U}_x)$ s.t.
		$g(\mathcal{U}_x)(t) = \left.s''\right|_{\mathcal{U}_x}$.
		Then we can concentrate on $\mathcal{U}_x \cap \mathcal{V}$.
		In case this is empty we can clearly glue together $t$ and $s$ to a section
		$u \in \mathcal{F}(\mathcal{V}\cup \mathcal{U}_x)$, getting mapped to
		$\left.s''\right|_{\mathcal{V}\cup \mathcal{U}_x}$
		(reasoning as in the previous point, $\mathcal{F}''$ is a sheaf).
		This would make
		\begin{equation}
		\left(\mathcal{V} \cup \mathcal{U}_x, u \right) >
		\left(\mathcal{V}, s\right)
		,\end{equation}
		contradicting maximality of the latter.
		Then we have $\mathcal{V}_x := \mathcal{U}_x \cap \mathcal{V} \neq \emptyset$.
		In particular
		\begin{equation}
			g(\mathcal{V}_x) \left( \left.s\right|_{\mathcal{V}_x} -
					\left.t\right|_{\mathcal{V}_x} \right) =
			g(\mathcal{V}_x) \left( \left.s\right|_{\mathcal{V}_x}\right) -
			g(\mathcal{V}_x) \left( \left.t\right|_{\mathcal{V}_x}\right) =
			\left.s''\right|_{\mathcal{V}_x} -
			\left.s''\right|_{\mathcal{V}_x} =
			0
		.\end{equation}
		(The above follows from linearity of $g(\mathcal{V}_x)$, morphism in $\mathsf{Ab}$).
		Then $\left.s\right|_{\mathcal{V}_x} -
			\left.t\right|_{\mathcal{V}_x} \in \ker g(\mathcal{V}_x)$.
		By left exactness of $\Gamma_{\mathcal{V}_x}$ we obtain that this is in the image
		of $f(\mathcal{V}_x)$, i.e. there is $\tilde{w} \in \mathcal{F}'(\mathcal{V}_x)$
		s.t. $f(\mathcal{V}_x)(\tilde{w}) =
		\left.s\right|_{\mathcal{V}_x} - \left.t\right|_{\mathcal{V}_x}$.
		But $\mathcal{F}'$ is flasque, hence $\tilde{w}$ can be extended to the whole
		$\mathcal{U}_x$, i.e. there is $w \in \mathcal{F}'(\mathcal{U}_x)$ s.t.
		$\left.w\right|_{\mathcal{V}_x} = \tilde{w}$.
		Then, again by left exactness of $\Gamma_{\mathcal{U}_x}$, we obtain that
		$f(\mathcal{U}_x)(w) \in \ker g(\mathcal{U}_x)$, hence
		for $\tilde{t} := t + f(\mathcal{U}_x)(w)$,
		by linearity of $g(\mathcal{U}_x)$, we have
		\begin{equation}
			g(\mathcal{U}_x)(\tilde{t}) =
			g(\mathcal{U}_x)(t)
		.\end{equation}
		Moreover we can look at the restriction of $\tilde{t}$ to $\mathcal{V}_x$
		and obtain
		\begin{align}
			\left.\tilde{t}\right|_{\mathcal{V}_x} &=
			\left.t\right|_{\mathcal{V}_x} +
			\left.f(\mathcal{U}_x)(w)\right|_{\mathcal{V}_x} =
			\left.t\right|_{\mathcal{V}_x} +
			f(\mathcal{V}_x)(\left.w\right|_{\mathcal{V}_x})\\
			&=
			\left.t\right|_{\mathcal{V}_x} + f(\mathcal{V}_x)(\tilde{w}) =
			\left.t\right|_{\mathcal{V}_x} +
			\left.s\right|_{\mathcal{V}_x} -
			\left.t\right|_{\mathcal{V}_x} =
			\left.s\right|_{\mathcal{V}_x}
		.\end{align}
		Then we can glue together $s$ and $\tilde{t}$ to a section in
		$\mathcal{F}(\mathcal{V}\cup \mathcal{U}_x)$.
		As before this section will get mapped to $\left.s''\right|_{\mathcal{V} \cup \mathcal{U}_x}$
		since both $s$ and $\tilde{t}$ do on their respective domains
		and $\mathcal{F}''$ is a sheaf.
		Then, again, we have contradicted maximality of $\left(\mathcal{V}, s\right)$.
		We can conclude since it can only happen that $\mathcal{V} = \mathcal{U}$.\qedhere
\end{enumerate}
\end{proof}

\begin{cor}
	Consider a sequence exact in $\mathsf{Sh}\left(X\right)$:
	\begin{equation}
	\begin{tikzcd}
		0 \arrow[r, "", rightarrow] &
		\mathcal{F}' \arrow[r, "", rightarrow] &
		\mathcal{F} \arrow[r, "", rightarrow] &
		\mathcal{F}'' \arrow[r, "", rightarrow] &
		0
	\end{tikzcd}
	\end{equation} 
	and $\mathcal{F}, \mathcal{F}'$ are flasque, then also $\mathcal{F}''$ is flasque.
\end{cor} 
\begin{proof}
	Consider $\mathcal{V} \subset \mathcal{U}$ open.
	Then we have the following commutative diagram with exact rows:
	\begin{equation}
	\begin{tikzcd}
		0 \arrow[r, "", rightarrow] &
		\mathcal{F}'(\mathcal{U}) \arrow[d, "", twoheadrightarrow] 
		\arrow[r, "", rightarrow] &
		\mathcal{F}(\mathcal{U}) \arrow[d, "", twoheadrightarrow] 
		\arrow[r, "", rightarrow] &
		\mathcal{F}''(\mathcal{U}) \arrow[d, "", rightarrow] 
		\arrow[r, "", rightarrow] &
		0\\
		0 \arrow[r, "", rightarrow] &
		\mathcal{F}'(\mathcal{V}) \arrow[r, "", rightarrow] &
		\mathcal{F}(\mathcal{V}) \arrow[r, "", rightarrow] &
		\mathcal{F}''(\mathcal{V}) \arrow[r, "", rightarrow] &
		0
	\end{tikzcd}
	.\end{equation} 
Moreover the first two vertical arrows are epimorphisms.
		Then it trivially follows that also $\rho^{Z^1(\mathcal{F})}_{\mathcal{UV}}$
		is an epimorphism. In fact, for any
		composable $f \circ \rho^{Z^1(\mathcal{F})}_{\mathcal{UV}} = 0$,
		we obtain
		\begin{equation}
			0 = f \circ \rho^{Z^1(\mathcal{F})}_{\mathcal{UV}} \circ \coker i^0(\mathcal{U}) =
			f \circ \coker i^0(\mathcal{V}) \circ \rho^{C^0(\mathcal{F})}_{\mathcal{UV}}
			= f
		,\end{equation} 
		since the last two maps are epimorphisms and composition is bilinear.
		This holds for any $\mathcal{V} \subset \mathcal{U}$, i.e.
		also $Z^1(\mathcal{F})$ is flasque.
		Then, iterating by induction, we obtain that $Z^n(\mathcal{F})$ is flasque.
		(tk: fix the names).
\end{proof}
tk: copy the definition of $C^)(\mathcal{F})$
\begin{prop}[]
	For any $\mathcal{F} \in \mathsf{Sh}\left(X\right)$, there is a canonical
	monomorphism
	\begin{equation}
	\begin{tikzcd}
		\mathcal{F} \arrow[r, "", rightarrowtail] &
		C^0(\mathcal{F}).
	\end{tikzcd}
	\end{equation} 
	into $C^0(\mathcal{F})$ a {\em flasque} sheaf.
\end{prop}
\begin{proof}
	Consider the \'etal\'e space associated to $\mathcal{F}$:
	\begin{equation}
	\begin{tikzcd}
		\widetilde{\mathcal{F}} = \coprod_{x \in X} \arrow[r, "\pi", rightarrow] &
		X.
	\end{tikzcd}
	\end{equation} 
	By definition $C^0(\mathcal{F})$ is the sheaf of not necessairily
	continuous sections of $\pi$.
	Consider $\mathcal{U} \subset X$ open, then
	\begin{equation}
		C^0(\mathcal{F})(\mathcal{U}) =
		\left\{ \sigma\colon \mathcal{U} \to \widetilde{\mathcal{F}}
		\ \middle|\ \pi \circ \sigma = id_{\mathcal{U}} \right\}
		\simeq \prod_{x \in \mathcal{U}} \mathcal{F}_{x}
	.\end{equation} 
	For all $\mathcal{V} \subset \mathcal{U}$ we than ahve a surjective map
	\begin{equation}
	\begin{tikzcd}
		\prod_{x \in \mathcal{U}} \mathcal{F}_{x} \arrow[r, "", twoheadrightarrow] 
		\arrow[d, "\sim", rightarrow] &
		\prod_{x \in \mathcal{V}} \mathcal{F}_{x} \arrow[d, "\sim", rightarrow] \\
		C^0(\mathcal{F})(\mathcal{U}) \arrow[r, "", rightarrow] &
		C^0(\mathcal{F})(\mathcal{V})
	\end{tikzcd}
	.\end{equation} 
	Then $C^0(\mathcal{F})$ is flasque.
	Let's construct the canonical monomorphism:
	\begin{align}
		\alpha\colon \mathcal{F} &\longrightarrow C^0(\mathcal{F}) \\
		\mathcal{F}(\mathcal{U}) &\longmapsto C^0(\mathcal{F})(\mathcal{U})
		= \prod_{x \in X} \mathcal{F}_{x} \nonumber \\
		s &\longmapsto \left( s_x \right)_{x \in X} \nonumber
	.\end{align} 
	$\alpha$ is injective iff $\alpha(\mathcal{U})$ is injective for all $\mathcal{U} \subset X$.
	In fact consider $s \in \ker \left( \alpha(\mathcal{U}) \right)$,
	iff $s_x = 0$ for all $x \in \mathcal{U}$.
	But this means that $s = 0$.
\end{proof}

\begin{prop}[]
	Let $\mathcal{I} \in \mathsf{Sh}\left(X\right)$ be an {\em injective} sheaf.
	Then $\mathcal{I}$ is {\em flasque}.
\end{prop}
\begin{proof}
	By the above we have a canonical monomorphism
	\begin{equation}
	\begin{tikzcd}
		0 \arrow[r, "", rightarrow] &
		\mathcal{I} \arrow[r, "", rightarrow] &
		C^0(\mathcal{I}).
	\end{tikzcd}
	\end{equation} 
	By injectivity of $\mathcal{I}$ there is a map $r$ making the diagram commute:
	\begin{equation}
	\begin{tikzcd}
		\mathcal{I} \arrow[r, "\alpha", rightarrowtail] 
		\arrow[d, "id_{\mathcal{I}}"', rightarrow] &
		C^0(\mathcal{I}) \arrow[ld, "", rightarrow] \\
		\mathcal{I} &
	\end{tikzcd}
	.\end{equation} 
	Then $r \circ \alpha = id_{\mathcal{I}}$.
	Moreover, for any $\mathcal{U} \subset X$ open, we have
	\begin{equation}
		r(\mathcal{U}) \circ \alpha(\mathcal{U}) = id_{\mathcal{I}(\mathcal{U})}
	.\end{equation} 
	But this means that $r(\mathcal{U})$ is surjective.
	Since $r$ is a morphism of sheaves, and $C^0(\mathcal{I})$ is flasque,
	we have the following diagram
	\begin{equation}
	\begin{tikzcd}
		C^0(\mathcal{I})(\mathcal{U}) \arrow[r, "r(\mathcal{U})", twoheadrightarrow] 
		\arrow[d, "\rho^{C^0(\mathcal{I})}_{\mathcal{UV}}", twoheadrightarrow] &
		\mathcal{I}(\mathcal{U}) \arrow[d, "\rho^{\mathcal{I}}_{\mathcal{UV}}", rightarrow] \\
		C^0(\mathcal{I})(\mathcal{V}) \arrow[r, "r(\mathcal{V})", twoheadrightarrow] &
		\mathcal{I}(\mathcal{V})
	\end{tikzcd}
	.\end{equation} 
	But this implies that also $\rho^{\mathcal{I}}_{\mathcal{UV}}$ is an
	epimorphism (tk: argue, but it seems easy).
\end{proof}

\begin{prop}[]
	Let $\mathcal{F} \in \mathsf{Sh}\left(X\right)$ be a flasque sheaf on $X$.
	Then $\mathcal{F}$ is $\Gamma_{\mathcal{U}}$-acyclic for any $\mathcal{U}\subset X$ open.
\end{prop}
\begin{proof}
	Consider a monomorphism
	\begin{equation}
	\begin{tikzcd}
		0 \arrow[r, "", rightarrow] &
		\mathcal{F} \arrow[r, "", rightarrow] &
		\mathcal{I},
	\end{tikzcd}
	\end{equation} 
	for $\mathcal{I}$ injective.
	This can be extended to a short exact sequence
	\begin{equation}
	\begin{tikzcd}
		0 \arrow[r, "", rightarrow] &
		\mathcal{F} \arrow[r, "", rightarrow] &
		\mathcal{I} \arrow[r, "", rightarrow] &
		\mathcal{I}/\mathcal{F} \arrow[r, "", rightarrow] &
		0,
	\end{tikzcd}
	\end{equation} 
	in which both $\mathcal{F}$ and $\mathcal{I}$ are flasque.
	Then also $\mathcal{I}/\mathcal{F}$ is flasque.
	From the fundamental theorem in cohomology we obtain the long exact cohomology sequence
	\begin{equation}
	\begin{tikzcd}
		0 \arrow[r, "", rightarrow] &
		H^{0} \left( \mathcal{U}, \mathcal{F} \right) \arrow[r, "", rightarrow] &
		H^{0} \left( \mathcal{U}, \mathcal{I} \right) \arrow[r, "", rightarrow] &
		H^{0} \left( \mathcal{U}, \mathcal{I}/\mathcal{F} \right) \arrow[r, "", rightarrow] &
		\ldots
	\end{tikzcd}
	.\end{equation} 
	In particular we obtain
	\begin{equation}
	\begin{tikzcd}
		H^{0} \left( \mathcal{U}, \mathcal{I} \right) \arrow[r, "", twoheadrightarrow] &
		H^{0} \left( \mathcal{U}, \mathcal{I}/\mathcal{F} \right) \arrow[r, "", rightarrow] &
		H^{1} \left( \mathcal{U}, \mathcal{F} \right) \arrow[r, "", rightarrow] &
		H^{0} \left( \mathcal{U}, \mathcal{I} \right) \arrow[r, "", rightarrow] &
		\ldots
	\end{tikzcd}
	.\end{equation} 
	Since the first map is surjective,
	exactness implies that $H^{1} \left( \mathcal{U}, \mathcal{F} \right) = 0$.
	Let's now argue by induction:we obtain that $H^{i} \left( \mathcal{U}, \mathcal{F} \right) = 0$
	for all $i \leq n$.
	Then, from the exact sequence (tk: copy it from someone).
	Then we are done, since $\mathcal{I}/\mathcal{F}$ is also flasque.
\end{proof}

Construction: Godement resolution.
It is a canonical resolution by flasque sheaves.
\begin{defn}[]
	Given a sheaf $\mathcal{F}$, one defines 
	\begin{equation}
		Z^1(\mathcal{F}) := C^0(\mathcal{F})/\mathcal{F}
		\qquad \text{ and } \qquad
		C^1(\mathcal{F}) := C^0( Z^1 (\mathcal{F} ))
	.\end{equation} 
	Then, by induction, one defines
	\begin{equation}
		Z^n(\mathcal{F}) := C^{n-1}(\mathcal{F})/Z^{n-1}(\mathcal{F})
		\qquad \text{ and } \qquad
		C^n(\mathcal{F}) := C^0( Z^n (\mathcal{F} ))
	.\end{equation} 
	Moreover, by induction, one defines the differential
	\begin{equation}
	\begin{tikzcd}
		d^n\colon C^n(\mathcal{F}) \arrow[r, "\coker i^n", rightarrow] &
		Z^{n+1}(\mathcal{F}) \arrow[r, "i^{n+1}", rightarrow] &
		C^{n+1}(\mathcal{F}).
	\end{tikzcd}
	.\end{equation} 
\end{defn}

\begin{rem}[]
	For $n \geq 0$ we have $\ker d^n = Z^n(\mathcal{F})$,
	where $Z^0(\mathcal{F}) := \mathcal{F}$.
	Moreover $\ima d^n = Z^n(\mathcal{F})$.
	(tk: copy the proof from the homework: factorization in mono and epi).

	It follows that the sequence
	\begin{equation}
	\begin{tikzcd}
		0 \arrow[r, "", rightarrow] &
		C^0(\mathcal{F}) \arrow[r, "", rightarrow] &
		C^1(\mathcal{F}) \arrow[r, "", rightarrow] &
		C^2(\mathcal{F}) \arrow[r, "", rightarrow] &
		\ldots
	\end{tikzcd}
	\end{equation} 
	is exact in $\mathsf{Sh}\left(X\right)$.
	In particular we have constructed a flasque resolution of $\mathcal{F}$.

	Moreover, since each part of the construction is
	functorial, then also the Godement resolution is.
	In particular any morphism $\alpha\colon \mathcal{F} \to \mathcal{G}$ of sheaves
	induces a morphism in $\mathrm{Ch}(\mathsf{Sh}\left(X\right))$
	$\alpha^\bullet\colon C^\bullet(\mathcal{F}) \to C^\bullet(\mathcal{G})$.
\end{rem}
One can apply to the cochain complex, degree-wise, the functor $\Gamma_{\mathcal{U}}$,
hence one defines the functor
\begin{equation}
\begin{tikzcd}
	\Gamma_{\mathcal{U}}\colon \mathrm{Ch}(\mathsf{Sh}\left(X\right)) \arrow[r, "", rightarrow] &
	\mathrm{Ch}(\mathsf{Ab})
\end{tikzcd}
.\end{equation} 
\begin{prop}[]
	The composition of the above two functors is exact.
	tk: write it better, copy proof from the homework, correct it as on the remarkable.
\end{prop}
\begin{rem}[]
	One applies the fundamental theorem of cohomology and obtains the long exact sequence:
	\begin{equation}
		\begin{tikzcd}[row sep=tiny]
			0 \arrow[r, "", rightarrow] &
			H^{0} \left( \mathcal{U}, \mathcal{F}' \right) \arrow[r, "", rightarrow] &
			H^{0} \left( \mathcal{U}, \mathcal{F} \right) \arrow[r, "", rightarrow]
			\arrow[d, phantom, ""{coordinate, name=Z1}] &
			H^{0} \left( \mathcal{U}, \mathcal{F}'' \right)
			\arrow[dll, "", rounded corners, bend left = 100,
			to path={ -- ([xshift=2ex]\tikztostart.east)
				|- (Z1) [near end]\tikztonodes
				-| ([xshift=-2ex]\tikztotarget.west)
				-- (\tikztotarget)}] & \\
			&
			H^{1} \left( \mathcal{U}, \mathcal{F}' \right) \arrow[r, "", rightarrow] &
			H^{1} \left( \mathcal{U}, \mathcal{F} \right) \arrow[r, "", rightarrow]
			\arrow[d, phantom, ""{coordinate, name=Z2}] &
			H^{1} \left( \mathcal{U}, \mathcal{F}'' \right)
			\arrow[dll, "", rounded corners, bend left = 100,
			to path={ -- ([xshift=2ex]\tikztostart.east)
				|- (Z2) [near end]\tikztonodes
				-| ([xshift=-2ex]\tikztotarget.west)
				-- (\tikztotarget)}] & \\
			&
			\phantom{H^{1} \left( \mathcal{U}, \mathcal{F} \right)} &
			\vdots
			\arrow[d, phantom, ""{coordinate, name=Z3}] &
			\phantom{H^{1} \left( \mathcal{U}, \mathcal{F}'' \right)}
			\arrow[dll, "", rounded corners, bend left = 100,
			to path={ -- ([xshift=2ex]\tikztostart.east)
				|- (Z3) [near end]\tikztonodes
				-| ([xshift=-2ex]\tikztotarget.west)
				-- (\tikztotarget)}] & \\
			&
			H^{n} \left( \mathcal{U}, \mathcal{F}' \right) \arrow[r, "", rightarrow] &
			H^{n} \left( \mathcal{U}, \mathcal{F} \right) \arrow[r, "", rightarrow]
			\arrow[d, phantom, ""{coordinate, name=Z4}] &
			H^{n} \left( \mathcal{U}, \mathcal{F}'' \right)
			\arrow[dll, "", rounded corners, bend left = 100,
			to path={ -- ([xshift=2ex]\tikztostart.east)
				|- (Z4) [near end]\tikztonodes
				-| ([xshift=-2ex]\tikztotarget.west)
				-- (\tikztotarget)}] & \\
			&
			H^{n+1} \left( \mathcal{U}, \mathcal{F}' \right) \arrow[r, "", rightarrow] &
			\ldots &
		\end{tikzcd}
		,\end{equation} 
		in which we recall that
		\begin{equation}
			H^{n} \left( \mathcal{U}, \mathcal{F} \right) :=
			R^{n}\Gamma_{\mathcal{U}} \left( \mathcal{F} \right) \simeq
			H^{n} \left( C^\bullet(\mathcal{F})(\mathcal{U}) \right)
		.\end{equation} 
		We have recovered the long exact sequence of cohomology groups associated
		to the short exact sequence of sheaves
		\begin{equation}
		\begin{tikzcd}
			0 \arrow[r, "", rightarrow] &
			\mathcal{F}' \arrow[r, "", rightarrow] &
			\mathcal{F} \arrow[r, "", rightarrow] &
			\mathcal{F}'' \arrow[r, "", rightarrow] &
			0
		\end{tikzcd}
		.\end{equation} 
\end{rem}

\end{document}

