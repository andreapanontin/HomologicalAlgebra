\documentclass[../Main]{subfiles}
\begin{document}
\section{Abelian categories}
\subsection{Preadditive categories}
\begin{defn}[Preadditive category]
	A category $\mathsf{C}$ is called \textbf{preadditive} iff it is a $\Z$ category, i.e. iff
	given any pair $X,Y \in \mathrm{Ob} \left(\mathsf{C}\right)$ the set $\mathrm{Hom}_{\mathsf{C}} \left( X, Y \right)$ is a $\Z$-module (an abelian group) and the composition of morphisms is a bilinear map.
\end{defn}

\begin{lem}\label{lem:OppositeIsPreadditive}
	Let $\mathsf{C}$ be a {\em preadditive} category, then $\mathsf{C}^{op}$ is a preadditive category.
\end{lem} 
\begin{proof}
	By definition of opposite category we have
	\begin{equation}
	\mathrm{Hom}_{\mathsf{C}} \left( A, B \right) :=
	\mathrm{Hom}_{\mathsf{C}^{op}} \left( B, A \right)
	\end{equation} 
	which is an abelian group. Moreover we have bilinearity of the composition,
	since it just reverses the order of composition.
\end{proof}

\begin{rem}[]
	We say that $\mathsf{C}$ has finite products iff
	given $\left\{ X_i \right\}_{i = 1}^n \subset \mathrm{Ob} \left(\mathsf{C}\right)$,
	any finite family,
	we can define $\prod_{i = 1}^n X_i  \in \mathrm{Ob} \left(\mathsf{C}\right)$,
	the universal representative of the projection maps.
\end{rem}

\begin{rem}[]\label{rem:EquivFiniteProd}
	$\mathsf{C}$ has finite products iff it has a final object $T := \prod_{\emptyset} X$
	and, for all $A, B \in \mathrm{Ob} \left(\mathsf{C}\right)$, 
	their product $A \cross B$ exists in $\mathsf{C}$.

	This is clearly enough by associativity of product.
\end{rem}

\begin{defn}[Final/initial/zero object]\leavevmode\vspace{-1\baselineskip}
	\begin{itemize}
		\item $T \in \mathrm{Ob} \left(\mathsf{C}\right)$ is a {\em final} or {\em terminal}
	object of $\mathsf{C}$ iff, given any $X \in \mathrm{Ob} \left(\mathsf{C}\right)$, then
	\begin{equation}
	\mathrm{Hom}_{\mathsf{C}} \left( X, T \right) = \left\{ * \right\}
	\end{equation} 
	is a singleton.
	Clearly $T$ is unique up to a unique isomorphism.

\item $I \in \mathrm{Ob} \left(\mathsf{C}\right)$ is an {\em initial} object of $\mathsf{C}$ iff,
	given any $X \in \mathrm{Ob} \left(\mathsf{C}\right)$, then
	\begin{equation}
	\mathrm{Hom}_{\mathsf{C}} \left( I, X \right) = \left\{ * \right\}
	\end{equation} 
	is a singleton.
	Clearly $I$ is unique up to a unique isomorphism.

\item $0 \in \mathrm{Ob} \left(\mathsf{C}\right)$ is a {\em zero} object iff it is both initial and terminal.
	\end{itemize}
\end{defn}

\begin{ex}\leavevmode\vspace{-.2\baselineskip}
	\begin{itemize}
		\item It is clear that $\left\{ * \right\} \in \mathrm{Ob} (\mathsf{Sets})$
			is a {\em final} object.
		\item Analogously $\emptyset \in \mathrm{Ob} \left(\mathsf{Sets}\right)$
			is an {\em initial} object.
		\item In $\mathsf{Rings}$ we know that $\mathbb{Z}$ is 
		{\em initial} and $\left\{ 0 \right\}$ is {\em final}.
	\item In $\mathsf{Ab}$ we know that $\left\{ 0 \right\}$ is also {\em initial}.
	\end{itemize}
\end{ex} 

\begin{rem}[Finite coproducts]
	A category $\mathsf{C}$ admits finite coproducts iff
	it admits an initial object $I =:\coprod_{\emptyset} X$
	and, for any $X, Y \in \mathsf{C}$, 
	their coproduct $X \coprod Y$ is an object of $\mathsf{C}$.
\end{rem}

\begin{lem}\label{lem:CoProdPreadditive}
	Let $\mathsf{C}$ be a {\em preadditive} category, then
	$\mathsf{C}$ has finite products iff it has finite coproducts.
	In such case the product of a finite family of objects coincide
	with its coproduct.
\end{lem} 

\begin{proof}
	Finite products in $\mathsf{C}$ correspond to finite coproducts in $\mathsf{C}^{op}$.
	Then we will prove only one direction, the other one follows by applying this proof
	to the opposite category, which is also preadditive by lemma \ref{lem:OppositeIsPreadditive}.

	Assume that $\mathsf{C}$ has finite products.
	By remark \ref{rem:EquivFiniteProd} it has
	$T \in \mathsf{C}$ a final object, and for any pair of objects $A,B \in \mathsf{C}$
	it has their product, $A \cross B$.
	\begin{enumerate}
		\item Let's show that $T$ is an initial object:
			for any $A \in \mathrm{Ob} \left(\mathsf{C}\right)$ we know that
			\begin{equation}
			\emptyset \neq \mathrm{Hom}_{\mathsf{C}} \left( T, A \right) \in \mathsf{Ab}
			.\end{equation} 
			This means it contains $0_{T,A}$ the zero morphism.
			Moreover $T$ is terminal, then
			\begin{equation}
			\left\{ 0_{T,T} \right\} \in \mathrm{Hom}_{\mathsf{C}} \left( T, T \right)
			= \left\{ * \right\}
			,\end{equation} 
			and also $id_T \in \mathrm{Hom}_{\mathsf{C}} \left( T, T \right)$.
			Then $0_{T,T} = id_T$.
			It follows that, given any $a\colon T \to A$,
			\begin{equation}
			a = a \circ id_T = a \circ 0_{T,T} = 0
			,\end{equation} 
			where the last equality follows from bilinearity of composition.

		\item Let's now show that, 
			for any pair of elements $A, B \in \mathrm{Ob} \left(\mathsf{C}\right)$,
			we have the isomorphism
			\begin{equation}
			A \cross B \simeq A \coprod B
			.\end{equation} 
			Let's denote $0_{A,B} \in \mathrm{Hom}_{\mathsf{C}} \left( A, B \right)$
			and $0_{B,A} \in \mathrm{Hom}_{\mathsf{C}} \left( B, A \right)$.
			Let's define the embedding morphisms:
			$i_A\colon A \to A \cross B$ as the unique map
			induced by the pair $\left(id_A, 0_{A,B}\right)$ from $A$.
			Analogously $i_B\colon B \to A \cross B$
			is induced by $\left( 0_{B, A}, id_B \right)$.
			Then the two maps correspond with the following commutative diagrams:
			\begin{equation}
			\begin{tikzcd}
				&
				A\\
				A \arrow[ru, "id_A", rightarrow] \arrow[r, "i_A", rightarrow] 
				\arrow[rd, "0_{A,B}"', rightarrow] &
				A \cross B \arrow[u, "p_A"', rightarrow] \arrow[d, "p_B", rightarrow] \\
				&
				B
			\end{tikzcd}
			\qquad \text{ and } \qquad
			\begin{tikzcd}
				&
				A\\
				B \arrow[ru, "0_{B,A}", rightarrow] \arrow[r, "i_A", rightarrow] 
				\arrow[rd, "id_B"', rightarrow] &
				A \cross B \arrow[u, "p_A"', rightarrow] \arrow[d, "p_B", rightarrow] \\
				&
				B
			\end{tikzcd}
			.\end{equation} 

			Let's now check the universal property of the coproduct
			on $\left(A \cross B, i_A, i_B\right)$:
			let $C \in \mathrm{Ob} \left(\mathsf{C}\right)$ and a pair of maps
			$f\colon A \to C$ and $g\colon B \to C$.
			One can define 
			\begin{equation}
				h := \left( f \circ p_A \right) +
				\left( g \circ p_B \right)
				\in \mathrm{Hom}_{\mathsf{C}} \left( A \cross B, C \right)
			.\end{equation} 
			This gives rise to a diagram and we want to show it is commutative, i.e.
			\begin{equation}\label{diag:CoprodPreadd1}
			\begin{tikzcd}
				A \arrow[rd, "f", rightarrow] \arrow[d, "i_A"', rightarrow] &
				\\
				A \cross B \arrow[r, "h", rightarrow] &
				C\\
				B \arrow[ru, "g"', rightarrow] \arrow[u, "i_B", rightarrow] &
			\end{tikzcd}
			\end{equation} 
			\begin{equation}
			h \circ i_A = f \qquad \text{ and } \qquad
			h \circ i_B = g
			.\end{equation} 
			In fact, from the definition and bilinearity, we obtain
			\begin{equation}
			h \circ i_A = 
			\left( f \circ p_A + g \circ p_B \right) \circ i_A =
			f \circ \underbrace{p_A \circ i_A}_{id_A} + 
			g \circ \underbrace{p_B \circ i_A}_{0_{A,B}} =
			f
			.\end{equation} 
			With almost a carbon copy of this argument one proves that $h \circ i_B = g$.

			Let's now show uniqueness of $h$.
			Let's, at first, show that
			\begin{equation}
			i_A \circ p_A + i_B \circ p_B = id_{A \cross B}
			.\end{equation} 
			By universal property of the product it enough to show that
			\begin{equation}
				p_A \circ (i_A \circ p_A + i_B \circ p_B) = p_A
			,\end{equation} 
			and analogously for $B$.
			In fact we have
			\begin{align}
				p_A \circ (i_A \circ p_A + i_B \circ p_B) =
				\underbrace{p_A \circ i_A}_{id_A} \circ p_A + 
				\underbrace{p_A \circ i_B}_{0_{B, A}} \circ p_B =
				p_A
			.\end{align} 
			Choose now $h' \colon A \cross B \to C$
			making the diagram in \eqref{diag:CoprodPreadd1} commute.
			Then
			\begin{align}
				h' &=
				h' \circ id_{A \cross B} =
				h' \circ \left( i_A \circ p_A +
				i_B \circ p_B\right) \\
				   &=
				\underbrace{h' \circ i_A}_{f} \circ p_A +
				\underbrace{h' \circ i_B}_{g} \circ p_B =
				f \circ p_a + g \circ p_B = h \qedhere
			.\end{align} 
	\end{enumerate}
\end{proof}

\begin{rem}[]
	The requirements in preadditive categories are crucial to have
	equality between products and coproducts.
	In $\mathsf{Sets}$, for example, products are cartesian products,
	whereas coproducts are disjoint unions.
\end{rem}

\begin{rem}[]
	Let $\mathsf{C}$ be a {\em preadditive} category and
	$B \in \mathrm{Ob} \left(\mathsf{C}\right)$.
	Suppose that $B \cross B$ exists, then there is a map
	\begin{equation}
		\begin{tikzcd}
			\delta_B\colon B \cross B \simeq B \coprod B \arrow[r, "", rightarrow] &
			B
		\end{tikzcd}
	,\end{equation} 
	called {\em codiagonal},
	defined to be the one making the following diagram commute
	\begin{equation}
	\begin{tikzcd}
		B \arrow[rd, "id_B", rightarrow] \arrow[d, "\epsilon_B"', rightarrow] & \\
		B \coprod B \arrow[r, "\delta_B", rightarrow] &
		B\\
		B \arrow[u, "\epsilon_B", rightarrow] \arrow[ru, "id_B"', rightarrow] 
	\end{tikzcd}
	.\end{equation} 
	As one could guess the name {\em codiagonal} comes from the fact that
	it satisfies the dual of the diagram for the {\em diagonal} morphism
	in a product.
\end{rem}

\subsection{Additive categories}
\begin{defn}[Additive category]
	An {\em additive category}
	is a {\em preadditive} category with finite products.
\end{defn}

\begin{rem}[]
	In an additive category, as seen in lemma \ref{lem:CoProdPreadditive}, 
	one has a zero element, since initial and terminal
	elements coincide, being (co)products indexed by the empty set.
\end{rem}

In reality the structure of abelian group, for the hom sets in a additive category,
does not depend on any external structure.
It can be defined directly from the morphisms in the category itself.

\begin{lem}\label{lem:MorphismSumAdditiveCat}
	Let $\mathsf{C}$ be an {\em additive} category and
	$B \in \mathrm{Ob} \left(\mathsf{C}\right)$.
	Then, given any couple of morphisms 
	$f,g \in \mathrm{Hom}_{\mathsf{C}} \left( A, B \right)$, 
	for any $A \in \mathrm{Ob} \left(\mathsf{C}\right)$,
	one has
	\begin{equation}
		f + g = \delta_B \circ \left( f, g \right)
	,\end{equation} 
	where $\left( f, g\right)\colon A \to B \cross B \simeq B \coprod B$
	is the map induced by the pair $f$ and $g$.
\end{lem} 
\begin{proof}
	We have already shown that
	\begin{equation}
	id_{B \cross B} = i_1 \circ p_1 + i_2 \circ p_2
	.\end{equation} 
	Then we have the commutative diagram
	\begin{equation}
	\begin{tikzcd}
		&
		B \arrow[rd, "id_B", rightarrow] \arrow[d, "i_1"', rightarrow] & \\
		B \cross B \arrow[r, "\sim", rightarrow]
		\arrow[ru, "p_1", rightarrow] \arrow[rd, "p_2"', rightarrow] &
		B \coprod B \arrow[r, "\delta_B", rightarrow] &
		B \\
		&
		B \arrow[u, "i_2", rightarrow] \arrow[ru, "id_B"', rightarrow] &
	\end{tikzcd}
	.\end{equation} 
	From this we conclude, since
	\begin{align}
		\delta_B \circ \left( f,g \right) &=
		\delta_B \circ \left( i_1 \circ p_1 + i_2 \circ p_2 \right) \circ \left( f, g \right) =
		( \underbrace{\delta_B \circ i_1}_{id_B} \circ p_1 + 
		\underbrace{\delta_B \circ i_2}_{id_B} \circ p_2) \circ ( f,g ) \nonumber \\
		&=
		p_1 \circ \left( f,g \right) + p_2 \circ \left( f, g \right) = f + g \qedhere
	.\end{align} 
\end{proof}

\begin{rem}[Notation]
	An additive category with finite products
	has also coproducts and viceversa (reason in $\mathsf{C}^{op}$).
	Then we denote them, since they are isomorphic, by
	\begin{equation}
	A \oplus B := A \cross B \simeq A \coprod B
	.\end{equation} 
\end{rem}

\begin{rem}[Zero morphism]
	Given any $A, B \in \mathrm{Ob} \left(\mathsf{C}\right)$ an additive category, one always has
	\begin{equation}
	0_{A,B} \in \mathrm{Hom}_{\mathsf{C}} \left( A, B \right)
	.\end{equation} 
	In particular this corresponds to the composition
	\begin{equation}
	\begin{tikzcd}
		0_{A,B} :
		A \arrow[r, "0", rightarrow] &
		0_{\mathsf{C}} \arrow[r, "0", rightarrow] &
		B
	\end{tikzcd}
	.\end{equation} 
	In fact one needs no group structure on $\mathrm{Hom}_{\mathsf{C}} \left( A, B \right)$
	to define a zero morphism between the two objects.
\end{rem}

\begin{defn}[Zero morphism]
	Let $\mathsf{C}$ be a category with zero object.
	Then, for all $A, B \in \mathrm{Ob} \left(\mathsf{C}\right)$,
	one can define the zero morphism 
	$0_{A,B} \in \mathrm{Hom}_{\mathsf{C}} \left( A, B \right)$ 
	as the composition
	\begin{equation}
		\begin{tikzcd}
		A \arrow[r, "", rightarrow] &
		0 \arrow[r, "", rightarrow] &
		B
		\end{tikzcd}
	.\end{equation} 
\end{defn}

\begin{lem}
	Let $\mathsf{C}$ be a category with zero object,
	finite products and coproducts
	s.t. the map induced by the following diagram is an isomorphism
	\begin{equation}
	\begin{tikzcd}
		B \arrow[rd, "(0\text{,} id_B)", rightarrow] \arrow[d, "i_B"', rightarrow] & \\
		A \coprod B \arrow[r, "\sim", rightarrow] &
		A \cross B \\
		A \arrow[u, "i_A", rightarrow] \arrow[ru, "(id_A\text{,} 0)"', rightarrow] &
	\end{tikzcd}
	.\end{equation} 
	Then $\mathrm{Hom}_{\mathsf{C}} \left( A, B \right)$ is a monoid:
\end{lem} 
\begin{proof}
	Given $f,g \in \mathrm{Hom}_{\mathsf{C}} \left( A, B \right)$, then
	one can set
	\begin{equation}
	\begin{tikzcd}
		f+g : 
		A \arrow[r, "(f\text{,} g)", rightarrow] &
		B \oplus B \arrow[r, "\delta_B", rightarrow] &
		B
	\end{tikzcd}
	\end{equation} 
	and
	\begin{equation}
	\begin{tikzcd}
		0_{A,B} :
		A \arrow[r, "", rightarrow] &
		0 \arrow[r, "", rightarrow] &
		B
	\end{tikzcd}
	.\end{equation} 
	One only need to check that this effectively gives rise to a monoid.
\end{proof}

\begin{defn}[Additive category, take 2]
	An additive category $\mathsf{C}$ is a category with a zero object,
	finite products and finite coproducts s.t.
	\begin{itemize}
		\item The map defined above is an iso
			\begin{equation}
			\begin{tikzcd}
				A \coprod B \arrow[r, "\sim", rightarrow] &
				A \cross B
			\end{tikzcd}
			,\end{equation} 
		\item The abelian monoid $\mathrm{Hom}_{\mathsf{C}} \left( A, B \right)$
			is an abelian group.
	\end{itemize}
\end{defn}

\begin{rem}[]
	The above definition makes clear that an additive category is
	just a category satisfying certain properties,
	with no additional structure.
	In fact the group structure on $\mathrm{Hom}_{\mathsf{C}} \left( A, B \right)$
	is determined by the underlying category $\mathsf{C}$.
\end{rem}

\begin{defn}[(Co)equalizer]
	Let $f,g$ be two parallel morphisms $A \rightrightarrows B$ in a category $\mathsf{C}$.
	\begin{itemize}
		\item An \textbf{equalizer} of $f$ and $g$ is a pair $\left(C, e\right)$,
			with $e\colon C \to A$, satisfying
	\begin{description}
		\item[eq1] $f \circ e = g \circ e$,
		\item[eq2] for $\left(C', e'\right)$ with $C' \xrightarrow{e'} A$ s.t. $f \circ e' = g \circ e'$, then
			$\exists\, ! \alpha: C' \to C$ s.t. $e \circ \alpha = e'$, i.e. the following diagram commutes
			\begin{equation}
			\begin{tikzcd}
				C \arrow[r, "e", rightarrow] & A \arrow[r, "f", rightarrow, shift left=.5ex] \arrow[r, "g"', rightarrow, shift right=.5ex] & B\\
				    & C' \arrow[lu, "\exists\, ! \alpha", dashrightarrow] \arrow[u, "e'"', rightarrow] & 
			\end{tikzcd}
			.\end{equation} 
	\end{description}
	\item A \textbf{coequalizer} of $f$ and $g$ is an equalizer of $f$ and $g$ in $\mathsf{C}^{op}$.
		In other words it is a pair $\left(C, p\right)$, 
		with $p\colon B \to C$ s.t.
		\begin{description}
			\item[coeq1] $p \circ f = p \circ g$,
			\item[coeq2] for $\left(C', p'\right)$ with $B \xrightarrow{p'} C'$ s.t. $p' \circ f = p' \circ g$, then $\exists\, ! \gamma: C \to C'$, with $\gamma \circ p = p'$, i.e. s.t. the following diagram commutes
			\begin{equation}
			\begin{tikzcd}
				A \arrow[r, "f", rightarrow, shift left=.5ex] \arrow[r, "g"', rightarrow, shift right=.5ex] & B \arrow[r, "P", rightarrow] \arrow[d, "p'", rightarrow] & C \arrow[dl, "\exists\, ! \gamma", dashrightarrow] \\
				    & C' & 
			\end{tikzcd}
			.\end{equation} 
		\end{description} 
	\end{itemize}
\end{defn}

\begin{defn}[(Co)kernel]
	Let $\mathsf{C}$ be a category with zero object.
	Let $f\colon A \to B$ a morphism in $\mathsf{C}$.
	One can define the kernel of $f$
	as the equalizer of $f$ and $0$, which corresponds with
	\begin{equation}
		\ker f = 
		\varprojlim \Bigg(
			\begin{tikzcd}[row sep=0.60em, column sep=1.2em]
			&
			0 \arrow[d, "", rightarrow] \\
			A \arrow[r, "f", rightarrow] &
			B
		\end{tikzcd} \Bigg)
		= A \cross_B 0
	.\end{equation} 
	Analogously one defines the cokernel as the dual of the kernel, i.e.
	\begin{equation}
	\mathrm{coker}\, f = 
		\varinjlim \Bigg(
			\begin{tikzcd}[row sep=0.60em, column sep=1.2em]
			A \arrow[r, "f", rightarrow] \arrow[d, "", rightarrow] &
			B \\
			0 &
		\end{tikzcd} \Bigg) =
	B \coprod_A 0
	.\end{equation} 
\end{defn}

\begin{rem}[]
	The object $\ker f$ is given with a map
	\begin{equation}
		\begin{tikzcd}
			\ker f \arrow[r, "", rightarrow] &
			A
		\end{tikzcd}
	\end{equation} 
	and $\mathrm{coker}\, f$ with a map
	\begin{equation}
	\begin{tikzcd}
		B \arrow[r, "", rightarrow] &
		\mathrm{coker}\,  f
	\end{tikzcd}
	.\end{equation} 
	Both of them satisfy a universal property.
	In particular the kernel satisfies the following:
	given any $g \in \mathrm{Hom}_{\mathsf{C}} \left( C, A \right)$
	for any object $C \in \mathrm{Ob} \left(\mathsf{C}\right)$ s.t.
	$f \circ g = 0$, then $g$ factorizes uniquely
	through $\ker f$.
	In pictures the diagram commutes:
	\begin{equation}
	\begin{tikzcd}
		\ker f \arrow[r, "k", rightarrow] &
		A \arrow[r, "f", rightarrow] &
		B \\
		&
		C \arrow[lu, "\exists !\, \alpha", dashrightarrow] \arrow[u, "g", rightarrow] 
		\arrow[ru, "0"', rightarrow] &
	\end{tikzcd}
	.\end{equation} 
	The cokernel, instead, satisfies the following:
	given any $g \in \mathrm{Hom}_{\mathsf{C}} \left( B, C \right)$
	for any object $C \in \mathrm{Ob} \left(\mathsf{C}\right)$ s.t.
	$g \circ f = 0$, then $g$ factorizes uniquely
	through $\mathrm{coker}\,  f$.
	In pictures the diagram commutes:
	\begin{equation}
	\begin{tikzcd}
		A \arrow[r, "f", rightarrow] \arrow[rd, "0"', rightarrow] &
		B \arrow[r, "c", rightarrow]  \arrow[d, "g", rightarrow] &
		\mathrm{coker}\, f \arrow[ld, "\exists !\, \gamma", dashrightarrow] \\
		&
		C &
	\end{tikzcd}
	.\end{equation} 
\end{rem}
\begin{rem}[]
	It follows from the definition that 
	\begin{equation}
	\begin{tikzcd}
		\ker f \arrow[r, "", rightarrowtail] &
		A
	\end{tikzcd}
	\end{equation} 
	is a monomorphism, whereas the morphism
	\begin{equation}
	\begin{tikzcd}
		B \arrow[r, "", twoheadrightarrow] &
		\mathrm{coker}\, f
	\end{tikzcd}
	\end{equation} 
	is an epimorphism.
\end{rem}
\begin{proof}
	Let's give $\alpha,\beta \colon A \to \ker f$, and denote
	by $i\colon \ker f \to A$ the natural map.
	Assume that $i \circ \alpha = i \circ \beta$.
	Then $f \circ i \circ \alpha = f \circ i \circ \beta = 0$.
	Then by universal property of the kernel there is a unique map
	$\gamma\colon X \to \ker f$ s.t. $i \circ \alpha = i \circ \gamma$.
	By uniqueness of $\gamma$ we have $\gamma = \alpha = \beta$.
\end{proof}

\begin{defn}[(Co)image]
	Let $\mathsf{C}$ be a category with zero object and $f\colon A \to B$
	be a morphism in $\mathsf{C}$.
	One can define the {\em image} of $f$, if it exists, as
	\begin{equation}
		\ima f := \ker \left( B \longrightarrow \mathrm{coker}\, f \right)
	.\end{equation} 
	Analogously, if it exists, one defines the {\em coimage} of $f$ as
	\begin{equation}
		\mathrm{coim}\, f := \mathrm{coker}\, \left( \ker f \longrightarrow A \right)
	.\end{equation} 
\end{defn}

\begin{rem}[]
	From this definition one obtains natural maps
	\begin{equation}
		\begin{tikzcd}[row sep=tiny]
			A \arrow[r, "", rightarrowtail] &
		\mathrm{Coim}\, f \\
			\ima f \arrow[r, "", twoheadrightarrow] & 
		B.
	\end{tikzcd}
	\end{equation} 
\end{rem}

\begin{exr}
	Let $\mathsf{C} := \mathsf{Ab}$ and $f\colon A \to B$ a morphism.
	Show that
	\begin{equation}
	\ima f \simeq \left\{ b \in B \ \middle|\ 
	\exists\, a \in A \text{ s.t. } f(a) = b \right\}
	.\end{equation} 
	Analogously show that
	\begin{equation}
	\mathrm{coim}\, f \simeq \frac{A}{\ker f}
	,\end{equation} 
	where the quotient is the usual quotient of abelian groups.
	tk: what did I want to say?
	Moreover check the commutativity of the required diagrams (i.e. the arrows from/to $A$, $B$).

	Notice that in $\mathsf{Ab}$ we have
	\begin{equation}
	\mathrm{coim}\, f \simeq \ima f
	.\end{equation} 
\end{exr} 

\begin{exr}
	Let $\mathsf{C}$ be a category with a zero object, kernels and cokernels.
	\begin{enumerate}
		\item Show that there exists a canonical map
			\begin{equation}
				\begin{tikzcd}
					\bar{f}\colon \mathrm{coim}\, f \arrow[r, "", rightarrow] &
				\ima f
				\end{tikzcd}
			\end{equation} 
			s.t. the following diagram commutes
			\begin{equation}
			\begin{tikzcd}
				\ker f \arrow[r, "", rightarrowtail] &
				A \arrow[r, "f", rightarrow] \arrow[d, "", twoheadrightarrow] &
				B \arrow[r, "", twoheadrightarrow] &
				\mathrm{coker}\, f\\
				&
				\mathrm{coim}\, f \arrow[r, "\bar{f}", rightarrow] &
				\ima f \arrow[u, "", rightarrowtail] &
			\end{tikzcd}
			.\end{equation} 
		\item Prove that $f = 0$ iff $\ker f = id_A$ iff
			$\mathrm{coker}\, f = id_B$.
		\item Prove that $f = id_A$ implies that
			\begin{align}
				\ker f &= \left( 0 \longrightarrow A \right)\\
				\mathrm{coker}\, f &= \left( A \longrightarrow 0 \right)
			.\end{align} 
	\end{enumerate}
\end{exr} 
\begin{proof}
	Since $\mathrm{coim}\, f = \mathrm{coker}\, \left( \ker f \right)$, 
	by the universal property of cokernels,
	there is a unique map $\alpha\colon \mathrm{coim}\, f \to B$ s.t.
	$f = \alpha \circ \pi$, for $\pi$ the map $\pi\colon A \to \mathrm{coim}\, f$.
	Then, since $\pi$ is epi, denoted $\theta\colon B \to \mathrm{coker}\, f$,
	we have that $0 = \theta \circ f = \theta \circ \alpha \circ \pi = \theta \circ \alpha$.
	This means that $\alpha$ factorizes through $\ima f$, i.e.
	we have a map $\bar{f}\colon \mathrm{coim}\, f \to \ima f$ making the diagram commute.
	Here is a diagram with all of the maps used in the proof:
			\begin{equation}
			\begin{tikzcd}
				\ker f \arrow[r, "", rightarrowtail] &
				A \arrow[r, "f", rightarrow] \arrow[d, "\pi"', twoheadrightarrow] &
				B \arrow[r, "\theta", twoheadrightarrow] &
				\mathrm{coker}\, f\\
				&
				\mathrm{coim}\, f \arrow[r, "\bar{f}"', rightarrow] 
				\arrow[ru, "\alpha", dashrightarrow] &
				\ima f \arrow[u, "", rightarrowtail]  &
			\end{tikzcd}\qedhere
			.\end{equation} 
\end{proof}

\begin{rem}[]
	Clearly $\ker$ and $\mathrm{coker}\, $ in $\mathsf{C}$ correspond respectively to
	$\mathrm{coker}\, $ and $\ker$ in $\mathsf{C}^{op}$.
\end{rem}

\begin{lem}
	Let $\mathsf{C}$ be a category with zero object, kernels and cokernels.
	Let $f\colon A \to B$ be a morphism, then
	\begin{align}
		\ker f &\simeq \ker \left( A \longrightarrow \mathrm{coim}\, f \right)\\
		\mathrm{coker}\, f &\simeq \mathrm{coker} \left( \ima f \longrightarrow B \right)
	.\end{align} 
\end{lem} 
\begin{proof}
	It is enough to show the first one, then the second follows from
	the same argument in the opposite category.
	By definition
	\begin{equation}
		\mathrm{coim}\, f = \mathrm{coker}\,  \left( \ker f \to  A \right)
	.\end{equation} 
	In particular it follows that
	\begin{equation}
	\begin{tikzcd}
		\ker f \arrow[r, "", rightarrow] \arrow[rr, bend right, "0"', rightarrow] &
		A \arrow[r, "", rightarrow] &
		\mathrm{coim}\, f
	\end{tikzcd}
	.\end{equation} 
	Let's show that $\ker f$ is universal with this property.
	Let $C \in \mathsf{C}$ and consider a morphism $C \to A$ s.t.
	\begin{equation}
	\begin{tikzcd}
		C \arrow[r, "", rightarrow] \arrow[rr, bend right, "0"', rightarrow] &
		A \arrow[r, "", rightarrow] &
		\mathrm{coim}\, f
	\end{tikzcd}
	.\end{equation} 
	Composing after this the map from the coimage to the image
	and the monomorphism from the image to $B$, we have a composition which gives zero:
	\begin{equation}
	\begin{tikzcd}
		C \arrow[r, "", rightarrow] \arrow[rrrr, bend right, "0"', rightarrow] &
		A \arrow[r, "", rightarrow] \arrow[rrr, "f", bend left, rightarrow] &
		\mathrm{coim}\, f \arrow[r, "\bar{f}", rightarrow] &
		\ima f \arrow[r, "", rightarrowtail] &
		B
	\end{tikzcd}
	.\end{equation} 
	Then it factorizes through $\ker f$ and from the universal property of $\ker$ we have our result.
\end{proof}

\subsection{Abelian categories and complexes}
\begin{defn}[Abelian category]
An {\em abelian} category is an additive category s.t.
	\begin{enumerate}
		\item any morphism has a kernel and a cokernel,
		\item given any morphism $f\colon A \to B$, the canonical map
			\begin{equation}
				\begin{tikzcd}
				\bar{f}\colon \mathrm{coim}\, f \arrow[r, "\sim", rightarrow] &
				\ima f
				\end{tikzcd}
			\end{equation} 
			is an isomorphism.
	\end{enumerate}
	A {\em preabelian} category is an additive category satisfying only $1$.
\end{defn}

\begin{rem}[]
	An abelian category behaves, in many ways, like the category of abelian groups
	(in other words, of $\mathbb{Z}$-modules).
	Actually there is a powerful theorem which states a similar result for small
	abelian category, i.e. they can be embedded into a category of modules, over
	a ring $R$, not necessairily $\mathbb{Z}$.
\end{rem}

\begin{rem}[]
	Consider a pair of composable morphism $f, g$ in an abelian category.
	Assume $g \circ f = 0$.
	Then there exists a canonical map
	\begin{equation}
	\begin{tikzcd}
		\ima f \arrow[r, "", rightarrow] &
		\ker g
	\end{tikzcd}
	.\end{equation} 
\end{rem}
\begin{proof}
	Since $\mathsf{C}$ is an abelian category,
	any morphism $f\colon A \to B$ factors through
	$\ima f$ as $f = \gamma \circ \delta$,
	for an epimorphism $\delta\colon A \to \ima f$
	(it is the composition of an epimorphism and an isomorphism).
	Then
	\begin{equation}
	0 = g \circ f = g \circ \gamma \circ \delta = g \circ \gamma
	,\end{equation} 
	since $\delta$ is epi.
	This means that $\gamma$ factors through $\ker g$, giving the desired
	morphism $\tilde{\gamma}$.
	In pictures we have the following commutative diagram:
	\begin{equation}
		\begin{tikzcd}
		A \arrow[r, "f", rightarrow] 
		\arrow[d, "\delta"', rightarrow] &
		B \arrow[r, "g", rightarrow] & 
		C\\
		\ima f \arrow[ru, "\gamma", rightarrow] 
		\arrow[r, "\tilde{\gamma}"', dashrightarrow] &
		\ker g \arrow[u, "\epsilon"', rightarrow] &
		&
	\end{tikzcd}
	\qedhere
	.\end{equation} 
%	Since the composition is $0$, we obtain that $g$
%	factors through $\mathrm{coker}\, f$, giving
%	$g = \tilde{f} \circ \pi$, for $\pi\colon B \to \mathrm{coker}\, f$.
%	Finally $\pi \circ f = 0$ factorizes through $\ker \mathrm{coker}\, f$,
%	which gives the desired factorization.
\end{proof}

\begin{defn}[]
	A sequence in a (pre)abelian category
	is a sequence of morphism
	\begin{equation}
	\begin{tikzcd}
		A^{-1} \arrow[r, "d^{-1}", rightarrow] &
		A^0 \arrow[r, "d^0", rightarrow] &
		\dots \arrow[r, "d^{n-1}", rightarrow] &
		A^n \arrow[r, "d^n", rightarrow]  &
		\
	\end{tikzcd}
	.\end{equation} 
	A sequence is called complex iff $d^{n+1} \circ d^n = 0$ for all $n$
	(in the set of indices of the complex).

	A complex is called {\em acyclic}, also called {\em exact sequence},
	iff the canonical map induced by the above condition
	\begin{equation}
		\begin{tikzcd}
			\ima d^n \arrow[r, "\sim", rightarrow] &
			\ker d^{n+1}
		\end{tikzcd}
	\end{equation} 
	is an isomorphism for all $n$.
\end{defn}

\begin{ex}
	The following is an exact sequence
	\begin{equation}
	\begin{tikzcd}
		0 \arrow[r, "", rightarrow] &
		\mathbb{Z} \arrow[r, "\cdot n", rightarrow] &
		\mathbb{Z} \arrow[r, "", rightarrow] &
		\mathbb{Z}/n\mathbb{Z} \arrow[r, "", rightarrow] &
		0
	\end{tikzcd}
	.\end{equation} 
\end{ex} 

\begin{exr}
	Let $\mathsf{C}$ be a (pre)abelian category and
	$f\colon A \to B$ a morphism in $\mathsf{C}$.
	\begin{enumerate}
		\item The following are equivalent
			\begin{enumerate}
				\item $f$ is a monomorphism,
				\item $\ker f = 0_{\mathsf{C}}$,
				\item the following is an exact sequence
					\begin{equation}
					\begin{tikzcd}
						0 \arrow[r, "", rightarrow] &
						A \arrow[r, "f", rightarrow] &
						B
					\end{tikzcd}
					.\end{equation} 
			\end{enumerate}
		\item The following are equivalent
			\begin{enumerate}
				\item $f$ is an epimorphism,
				\item $\mathrm{coker}\, f = 0_{\mathsf{C}}$,
				\item the following is an exact sequence
					\begin{equation}
					\begin{tikzcd}
						A \arrow[r, "f", rightarrow] &
						B \arrow[r, "", rightarrow] &
						0
					\end{tikzcd}
					.\end{equation} 
			\end{enumerate}
	\end{enumerate}
\end{exr} 
\begin{proof}
	Let's only prove the first point.
	The second follows applying the first in the opposite category.

	Let $f$ be a monomorphism, $C \in \mathsf{C}$ and $g\colon C \to A$
	s.t. $f \circ g = 0$.
	Then 
	\begin{equation}
	f \circ g = 0_{C,B} = f \circ 0_{C,A}
	.\end{equation} 
	Since $f$ is a mono we obtain that $g = 0_{C,A}$,
	i.e. $g$ factors through $0 = \ker f$ (uniqueness of
	the factorization, then, is obvious).

	Let's prove the converse, $\ker f = 0$.
	Let $C \in \mathsf{C}$ and consider two maps $g_0, g_1\colon C \to A$
	equalized by $f \circ g_0 = f \circ g_1$.
	Let's take the difference
	\begin{equation}
		0 = f \circ g_0 - f \circ g_1 = f \circ (g_0 - g_1)
	.\end{equation} 
	Then $g_0 - g_1$ factors through $\ker f = 0$, hence $g_0 - g_1 = 0$, 
	i.e. $g_0 = g_1$.
\end{proof}

\begin{exr}
	Let $\mathsf{C}$ be an abelian category.
	Let $f\colon A \to B$ be a morphism. Then
	\begin{enumerate}
		\item $f$ is a monomorphism iff $f = \ima f$,
		\item $f$ is an epimorphism iff $f = \mathrm{coim}\, f$,
		\item $f$ is an isomorphism iff $f$ is both a mono and an epimorphism.
	\end{enumerate}
\end{exr} 

\begin{lem}
	Let $\mathsf{C}$ be an {\em abelian} category and
	\begin{equation}
	\begin{tikzcd}
		A \arrow[r, "f", rightarrow] &
		B \arrow[r, "g", rightarrow] &
		C
	\end{tikzcd}
	\end{equation} 
	a sequence in $\mathsf{C}$.
	TFAE:
	\begin{enumerate}
		\item The sequence is exact.
		\item The following is exact in $\mathsf{C}^{op}$:
			\begin{equation}
			\begin{tikzcd}
				C \arrow[r, "g^{op}", rightarrow] &
				B \arrow[r, "f^{op}", rightarrow] &
				A
			\end{tikzcd}
			.\end{equation} 
		\item $\mathrm{coker}\, f = \mathrm{coim}\,  g$.
	\end{enumerate}
\end{lem} 
\begin{proof}\leavevmode\vspace{-.2\baselineskip}
	\begin{description}
		\item[1 $\implies$ 3:]
			Notice that exactness is equivalent to $\ima f = \ker g$.
			Then
			\begin{equation}
			\begin{tikzcd}
				\mathrm{coker} \left( \ima f \right) \arrow[r, "", equals]
				\arrow[d, "", equals] &
				\mathrm{coker} \left( \ker g \right) \arrow[d, "", equals] \\
				\mathrm{coker}\, f \arrow[r, "", equals] &
				\mathrm{coim}\, g
			\end{tikzcd}
			.\end{equation} 

		\item[3 $\implies$ 1:] By hypothesis
			\begin{align}
				\mathrm{coker} \left( \ima f \right) &\simeq
				\mathrm{coker} \left( \ker g \right)\\
				\ker \left( \mathrm{coker} \left( \ima f \right) \right) &\simeq
				\ker \left( \mathrm{coker} \left( \ker g \right) \right)\\
				\ima \left( \ima f \right) &\simeq \ima \left( \ker g \right)
			.\end{align} 
			By exercise $3$, if $h$ is a monomorphism, $\ima h = h$.
			Then
			\begin{equation}
			\ima f = \ker g
			.\end{equation} 

		\item[3 $\iff$ 2:] We know that images and kernels, in the opposite category, correspond
			(respectively) to coimages and cokernels, hence the following two lines are equivalent:
			\begin{align}
				\ima \left( g^{op} \right) &\simeq \ker \left( g^{op} \right)\\
				\mathrm{coim}\, g &\simeq \mathrm{coker}\, f\nonumber\qedhere
			.\end{align} 
	\end{description} 
\end{proof}

\begin{lem}
	Let $\mathsf{C}$ be an {\em abelian} category. TFAE:
	\begin{enumerate}
		\item $f \simeq \ker g$ iff the following is exact
			\begin{equation}
			\begin{tikzcd}
				0 \arrow[r, "", rightarrow] &
				A \arrow[r, "f", rightarrow] &
				B \arrow[r, "g", rightarrow] &
				C
			\end{tikzcd}
			.\end{equation} 
		\item $g \simeq \coker f$ iff the following is exact
			\begin{equation}
			\begin{tikzcd}
				A \arrow[r, "f", rightarrow] &
				B \arrow[r, "g", rightarrow] &
				C \arrow[r, "", rightarrow] &
				0
			\end{tikzcd}
			.\end{equation} 
	\end{enumerate}
\end{lem} 
\begin{proof}
	The second point follows from the first, applied in $\mathsf{C}^{op}$.
	Let's show the first one, then.
	Let's recall the definition of exactness.
			\begin{equation}
			\begin{tikzcd}
				0 \arrow[r, "", rightarrow] &
				A \arrow[r, "f", rightarrow] &
				B \arrow[r, "g", rightarrow] &
				C
			\end{tikzcd}
			\end{equation} 
	is exact iff $f$ is a monomorphism and $\ima f \simeq \ker g$,
	but then, since $f$ is a monomorphism we know that $\ima f = f$, and we have proved one direction.
	The other is also clear, since $\ker g$ is a monomorphism, hence
	$f \simeq \ima f$ and we have recovered the definition of exactness.
\end{proof}

\begin{rem}[Hom functor]
	Fix a category $\mathsf{C}$, then the hom functor
	\begin{align}
		\mathrm{Hom}_{\mathsf{C}} \colon \mathsf{C}^{op} \cross \mathsf{C} &\longrightarrow \mathsf{Sets} \\
		\left(X, Y\right) &\longmapsto \mathrm{Hom}_{\mathsf{C}} \left( X, Y \right) \nonumber \\
	\end{align} 
	is contravariant in the first entry, and covariant in the second one.
	In particolar, on morphisms, it acts as follows:
	let $g \in \mathrm{Hom}_{\mathsf{C}} \left( Y, Z \right)$, then
	\begin{align}
		g_* := \mathrm{Hom}_{\mathsf{C}} \left( X, g \right) \colon 
		\mathrm{Hom}_{\mathsf{C}} \left( X, Y \right) 
		&\longrightarrow \mathrm{Hom}_{\mathsf{C}} \left( X, Z \right) \\
		h &\longmapsto g \circ h \nonumber
	.\end{align} 
	For $f \in \mathrm{Hom}_{\mathsf{C}} \left( Z, X \right)$, instead
	\begin{align}
		f^* := \mathrm{Hom}_{\mathsf{C}} \left( f, Y \right) \colon 
		\mathrm{Hom}_{\mathsf{C}} \left( X, Y \right) 
		&\longrightarrow \mathrm{Hom}_{\mathsf{C}} \left( Z, Y \right) \\
		h &\longmapsto h \circ f \nonumber
	.\end{align} 
\end{rem}

\begin{lem}\label{lem:HomExactnessEquivalence}
	Let $\mathsf{C}$ be an {\em abelian} category.
	\begin{enumerate}
		\item A sequence in $\mathsf{C}$
			\begin{equation}
			\begin{tikzcd}
				0 \arrow[r, "", rightarrow] &
				A \arrow[r, "", rightarrow] &
				B \arrow[r, "", rightarrow] &
				C
			\end{tikzcd}
			\end{equation} 
			is exact iff, for any $M \in \mathrm{Ob} \left(\mathsf{C}\right)$,
			the image sequence
			\begin{equation}
			\begin{tikzcd}
				0 \arrow[r, "", rightarrow] &
				\mathrm{Hom}_{\mathsf{C}} \left( M, A \right) \arrow[r, "", rightarrow] &
				\mathrm{Hom}_{\mathsf{C}} \left( M, B \right) \arrow[r, "", rightarrow] &
				\mathrm{Hom}_{\mathsf{C}} \left( M, C \right)
			\end{tikzcd}
			\end{equation} 
			is exact in $\mathsf{Ab}$.

		\item A sequence in $\mathsf{C}$
			\begin{equation}
			\begin{tikzcd}
				A \arrow[r, "", rightarrow] &
				B \arrow[r, "", rightarrow] &
				C \arrow[r, "", rightarrow] &
				0
			\end{tikzcd}
			\end{equation} 
			is exact iff, for any $M \in \mathrm{Ob} \left(\mathsf{C}\right)$,
			the image sequence
			\begin{equation}
			\begin{tikzcd}
				0 \arrow[r, "", rightarrow] &
				\mathrm{Hom}_{\mathsf{C}} \left( C, M \right) \arrow[r, "", rightarrow] &
				\mathrm{Hom}_{\mathsf{C}} \left( B, M \right) \arrow[r, "", rightarrow] &
				\mathrm{Hom}_{\mathsf{C}} \left( A, M \right)
			\end{tikzcd}
			\end{equation} 
			is exact in $\mathsf{Ab}$.
	\end{enumerate}
\end{lem} 
\begin{proof}
	$2$ follows from $1$. In fact the sequence
	\begin{equation}
	\begin{tikzcd}
		A \arrow[r, "", rightarrow] &
		B \arrow[r, "", rightarrow] &
		C \arrow[r, "", rightarrow] &
		0
	\end{tikzcd}
	\end{equation} 
	is exact in $\mathsf{C}$ iff
	\begin{equation}
	\begin{tikzcd}
		0 \arrow[r, "", rightarrow] &
		A \arrow[r, "", rightarrow] &
		B \arrow[r, "", rightarrow] &
		C
	\end{tikzcd}
	\end{equation} 
	is exact in $\mathsf{C}^{op}$.
	Then one can apply $\mathrm{Hom}_{\mathsf{C}^{op}} \left( M, - \right)$,
	obtaining
	\begin{equation}
	\begin{tikzcd}
		0 \arrow[r, "", rightarrow] &
		\mathrm{Hom}_{\mathsf{C}^{op}} \left( M, C \right) \arrow[r, "", rightarrow] &
		\mathrm{Hom}_{\mathsf{C}^{op}} \left( M, B \right) \arrow[r, "", rightarrow] &
		\mathrm{Hom}_{\mathsf{C}^{op}} \left( M, A \right)
	\end{tikzcd}
	,\end{equation} 
	which is exact iff it is exact in the opposite category, i.e. iff
	\begin{equation}
	\begin{tikzcd}
		0 \arrow[r, "", rightarrow] &
		\mathrm{Hom}_{\mathsf{C}} \left( C, M \right) \arrow[r, "", rightarrow] &
		\mathrm{Hom}_{\mathsf{C}} \left( B, M \right) \arrow[r, "", rightarrow] &
		\mathrm{Hom}_{\mathsf{C}} \left( A, M \right)
	\end{tikzcd}
	\end{equation} 
	is exact in $\mathsf{C}$.
	In other words $1$ is equivalent to $2$, thanks to the previous lemma.

	Let's now prove $1$.
	Assume 
	\begin{equation}
		\begin{tikzcd}
		0 \arrow[r, "", rightarrow] &
		A \arrow[r, "i", rightarrow] &
		B \arrow[r, "f", rightarrow] &
		C
		\end{tikzcd}
	\end{equation} 
	is exact, i.e. $i = \ker f$, by the above proposition.
	But the universal property of $\ker f$ is equivalent to the exactness of the desired
	sequence.
	In fact, fixed $M \in \mathrm{Ob} \left(\mathsf{C}\right)$, one obtains the sequence
	\begin{equation}
	\begin{tikzcd}
		0 \arrow[r, "", rightarrow] &
		\mathrm{Hom}_{\mathsf{C}} \left( M, A \right) \arrow[r, "i_*", rightarrow] &
		\mathrm{Hom}_{\mathsf{C}} \left( M, B \right) \arrow[r, "f_*", rightarrow] &
		\mathrm{Hom}_{\mathsf{C}} \left( M, C \right)
	\end{tikzcd}
	.\end{equation} 
	Then, given any $h \in \mathrm{Hom}_{\mathsf{C}} \left( M, A \right)$, one
	has $f_* \circ i_* (h) = f \circ i \circ h = 0$, since $f \circ i = 0$.
	In other words $f _* \circ i_* = 0$.
	Then, by universal property of $\ker$, on obtains a morphism
	\begin{equation}\label{eq:IsoImaiKerf*}
	\begin{tikzcd}[row sep=tiny]
		\mathrm{Hom}_{\mathsf{C}} \left( M, A \right) \arrow[r, "", rightarrow] &
		\ker f_*\\
		\beta \arrow[r, "", mapsto] &
		i \circ \beta \in \mathrm{Hom}_{\mathsf{C}} \left( M, B \right)
	\end{tikzcd}
	.\end{equation} 
	Moreover, let $\alpha \in \ker f_*$, i.e. $f \circ \alpha = 0$
	then, since $i = \ker f$, $\alpha$ factorizes through $A$,
	i.e. there exists a unique $\tilde{\alpha}\colon M \to A$
	s.t. $\alpha = i \circ \tilde{\alpha}$.
	\begin{equation}
	\begin{tikzcd}
		M \arrow[r, "\exists\, ! \tilde{\alpha}", rightarrow] 
		\arrow[rd, "\alpha"', rightarrow] &
		A \arrow[d, "i", rightarrow] \\
		&
		B
	\end{tikzcd}
	.\end{equation} 
	In other words the map in \eqref{eq:IsoImaiKerf*} is an isomorphism of abelian groups.

	Let's prove the converse.
	Consider, at first, $M = A$. Then the following is exact
	\begin{equation}
	\begin{tikzcd}
		0 \arrow[r, "", rightarrow] &
		\mathrm{Hom}_{\mathsf{C}} \left( A, A \right) \arrow[r, "i_*", rightarrow] &
		\mathrm{Hom}_{\mathsf{C}} \left( A, B \right) \arrow[r, "f_*", rightarrow] &
		\mathrm{Hom}_{\mathsf{C}} \left( A, C \right)
	\end{tikzcd}
	.\end{equation} 
	By exactness $f_* \circ i_* = 0$.
	In particular, taken $id_A \in \mathrm{Hom}_{\mathsf{C}} \left( A, A \right)$,
	we have $f \circ i \circ id_A = f \circ i = 0$.

	Let's now consider a generic $M \in \mathsf{C}$.
	Given $\alpha\colon M \to B$ s.t. $f \circ \alpha = 0$, then
	there exists a unique $\tilde{a}\colon M \to A$ s.t. $\alpha = i \circ \tilde{\alpha}$.
	In fact this is true by injectivity of $i_*$ and the fact that
	$\ima i_* = \ker f_*$ (notice that we are working in $\mathsf{Ab}$).
	Then $i$ satisfies the universal property of the kernel of $f$, and
	we have exactness of the sequence in $\mathsf{C}$.
\end{proof}

\begin{cor}
	Consider a commutative diagram in an {\em abelian} category $\mathsf{C}$:
	\begin{equation}
	\begin{tikzcd}
		0 \arrow[r, "", rightarrow] &
		A \arrow[r, "i", rightarrow] \arrow[d, "f_A", rightarrow] &
		B \arrow[r, "p", rightarrow] \arrow[d, "f_B", rightarrow] &
		C \arrow[r, "", rightarrow] \arrow[d, "f_C", rightarrow] &
		0 \\
		0 \arrow[r, "", rightarrow] &
		A' \arrow[r, "i'", rightarrow] &
		B' \arrow[r, "p'", rightarrow] &
		C' \arrow[r, "", rightarrow] &
		0
	\end{tikzcd}
	\end{equation} 
	s.t. the rows are exact and $f_A$ and $f_C$ are isomorphisms.
	Then also $f_B$ is an isomorphism.
\end{cor} 
\begin{proof}
	Let's start by showing that $f_B$ is a monomorphism.
	For any $M \in \mathrm{Ob} \left(\mathsf{C}\right)$ we have a commuative diagram
	with exact rows:
	\begin{equation}
	\begin{tikzcd}
		0 \arrow[r, "", rightarrow] &
		\mathrm{Hom}_{\mathsf{C}} \left( M, A \right) \arrow[r, "i_*", rightarrow] 
		\arrow[d, "(f_A)_*", rightarrow] &
		\mathrm{Hom}_{\mathsf{C}} \left( M, B \right) \arrow[r, "p_*", rightarrow] 
		\arrow[d, "(f_B)_*", rightarrow] &
		\mathrm{Hom}_{\mathsf{C}} \left( M, C \right)
		\arrow[d, "(f_C)_*", rightarrow] \\
		0 \arrow[r, "", rightarrow] &
		\mathrm{Hom}_{\mathsf{C}} \left( M, A' \right) \arrow[r, "i'_*", rightarrow] &
		\mathrm{Hom}_{\mathsf{C}} \left( M, B' \right) \arrow[r, "p'_*", rightarrow] &
		\mathrm{Hom}_{\mathsf{C}} \left( M, C' \right)
	\end{tikzcd}
	.\end{equation} 
	Exactness of the rows follows from the previous lemma, whereas commutativity from
	the fact that $\mathrm{Hom}_{\mathsf{C}} \left( M, - \right)$ is a functor.
	Notice, moreover, that any functor sends isomorphisms to isomorphisms, hence
	both $(f_A)_*$ and $(f_C)_*$ are isomorphisms.

	With a simple diagram chase we can prove that
	$\left( f_B \right)_*$ is injective for all $M \in \mathrm{Ob} \left(\mathsf{C}\right)$.
	In fact, consider $x \in \mathrm{Hom}_{\mathsf{C}} \left( M, B \right)$
	s.t. $f_B \circ x = 0$.
	Then 
	\begin{equation}
	(f_C)_* \circ p_* (x) = p'_* \circ (f_B)_* (x) = 0
	.\end{equation} 
	Since $\left( f_C \right)_*$ is an isomorphism, it means that $p_*(x) = 0$, i.e.
	$x \in \ker p_* = \ima i_*$, hence $x = i_*(y)$ for $y \in \mathrm{Hom}_{\mathsf{C}} \left( M, A \right)$.
	Moreover $\left( f_B \right)_* (x) = 0$ implies that $p'_* \circ (f_B)_* (x) = 0$,
	i.e. $(f_B)_*(x) \in \ker p'_* = \ima i'_*$.
	Moreover $i'_*$ is injective, its only inverse image is $0$.
	Then $(f_A)_*(y) = 0 \in \mathrm{Hom}_{\mathsf{C}} \left( M, A' \right)$.
	Since $\left( f_A \right)_*$ is an isomorphism by hypothesis, we have $y = 0$, hence $x = 0$.

	But $\left( f_B \right)_*$ injective means, exactly, that $f_B$ is a monomorphism.
	In fact, given $\alpha, \beta \in \mathrm{Hom}_{\mathsf{C}} \left( M, B \right)$
	on has that $f_B \circ \alpha = f_B \circ \beta$ iff
	$\left( f_B \right)_* (\alpha) = \left( f_B \right)_* (\beta)$.
	Being $(f_B)_*$ injective this implies $\alpha = \beta$, i.e. $f_B$ is a mono.

	Reasoning with a similar diagram chase one proves that $(f_B)_*$
	is also an epimorphism.
	(tk: copy the proof of epimorphism)

	Finally we can conclude, since $\left( f_B \right)_*$ is both a monomorphism
	and an epimorphism
	which, in an abelian category, is equivalent to being an isomorphism.
\end{proof}

\begin{defn}[Split sequence]
	An exact sequence in an abelian category
	\begin{equation}
	\begin{tikzcd}
		0 \arrow[r, "", rightarrow] &
		A \arrow[r, "i", rightarrow] &
		B \arrow[r, "p", rightarrow] &
		C \arrow[r, "", rightarrow] \arrow[l, "s", bend left, rightarrow] &
		0
	\end{tikzcd}
	\end{equation} 
	is said to {\em split} iff there exists a morphism $s\colon C \to B$
	s.t. $p \circ s = id_C$.
	Such morphism is called a {\em splitting} of $p$.
\end{defn}

\begin{lem}
	Let $\mathsf{C}$ be an abelian category, and consider an exact sequence in $\mathsf{C}$.
	\begin{equation}
	\begin{tikzcd}
		0 \arrow[r, "", rightarrow] &
		A \arrow[r, "i", rightarrow] &
		B \arrow[r, "p", rightarrow] &
		C \arrow[r, "", rightarrow] &
		0
	\end{tikzcd}
	.\end{equation} 
	TFAE:
	\begin{enumerate}
		\item The sequence splits.
		\item We have an isomorphism $\varphi\colon A \oplus C \to B$
			making the following diagram commute:
	\begin{equation}
	\begin{tikzcd}
		0 \arrow[r, "", rightarrow] &
		A \arrow[r, "i", rightarrow] \arrow[d, "", equals] &
		B \arrow[r, "p", rightarrow] &
		C \arrow[r, "", rightarrow] \arrow[d, "", equals] &
		0 \\
		0 \arrow[r, "", rightarrow] &
		A \arrow[r, "i_A", rightarrow] &
		A \oplus C \arrow[r, "p_C", rightarrow] \arrow[u, "\varphi", rightarrow] &
		C \arrow[r, "", rightarrow] &
		0 
	\end{tikzcd}
	.\end{equation} 

		\item There exists a map $a\colon B \to A$ s.t. $s \circ i = id_A$.
	\end{enumerate}
\end{lem} 
\begin{proof}\leavevmode\vspace{-.2\baselineskip}
	\begin{description}
		\item[1 $\implies$ 2:]
			\begin{equation}
	\begin{tikzcd}
		0 \arrow[r, "", rightarrow] &
		A \arrow[r, "i_A", rightarrow] \arrow[d, "", equals] \arrow[rd, "i", rightarrow] &
		A \oplus C \arrow[r, "p_C", rightarrow]
		\arrow[d, "f =(i s)" near start, rightarrow] &
		C \arrow[r, "", rightarrow] \arrow[d, "", equals] \arrow[ld, "s", rightarrow] &
		0 \\
		0 \arrow[r, "", rightarrow] &
		A \arrow[r, "i", rightarrow] &
		B \arrow[r, "p", rightarrow] &
		C \arrow[r, "", rightarrow]  \arrow[l, "s", bend left, rightarrow] &
		0 
	\end{tikzcd}
	.\end{equation} 
	We want to show that $f$ is an isomorphism and the above diagram commutes.

	Let's start by tackling commutativity of the diagram.
	By definition of $f$ and direct sum we clearly have that $i = f \circ i_A$.
	Then we only need to check that $p_C = p \circ f$.
	To check the above equality we need to compute the composition with $i_A$ and $i_C$.
	They clearly are given by:
	\begin{align}
		p \circ f \circ i_C &= p \circ \left( f \circ i_C \right) = p \circ s = id_C\\
		p \circ f \circ i_A &= p \circ i = 0
	.\end{align} 
	Then, by universal property of coproduct, we have $p \circ f = p_C$ as desired.

	Exactness is also clear: the second row is exact by assumption,
	whereas the first is exact by definition of direct sum: $i_A = \ker p_C$,
	one can explicitly check that it satisfies the universal property.

	Then we can conclude applying the five lemma.
\item[2 $\implies$ 1:]
	It is obvious from the identity $p_C \circ i_C = id_C$
	and commutativity of the diagram, i.e. $p_C = p \circ f$.
	In fact
	\begin{equation}
	p \circ f \circ i_C = p_C \circ i_C = id_C
	.\end{equation} 
\item[2 $\iff$ 3:] This is proved almost in the exact same way as the above two implications.
	\qedhere
	\end{description} 
\end{proof}

\begin{rem}[]
	The above lemma says that an exact sequence splits iff
	it is isomorphic, as exact sequence, to
	\begin{equation}
	\begin{tikzcd}
		0 \arrow[r, "", rightarrow] &
		A \arrow[r, "i_A", rightarrow] &
		A \oplus C \arrow[r, "p_C", rightarrow] &
		C \arrow[r, "", rightarrow] &
		0
	\end{tikzcd}
	.\end{equation} 
\end{rem}

\begin{defn}[]
	Let $R$ be a ring. We define the category
	$R\text{-}\mathsf{Mod}$ to be the category of right $R$-modules.
\end{defn}

\begin{rem}[]
	$R\text{-}\mathsf{Mod}$ is an {\em abelian} category.
\end{rem}

\begin{lem}[Snake lemma]
	Consider the commutative diagram in $R\text{-}\mathsf{Mod}$ with exact rows
	\begin{equation}
	\begin{tikzcd}
		&
		A \arrow[r, "\alpha", rightarrow] \arrow[d, "u"', rightarrow] &
		B \arrow[r, "\beta", rightarrow] \arrow[d, "v"', rightarrow] &
		C \arrow[r, "", rightarrow] \arrow[d, "w"', rightarrow] &
		0 \\
		0 \arrow[r, "", rightarrow] &
		A' \arrow[r, "\alpha'", rightarrow] &
		B' \arrow[r, "\beta'", rightarrow] &
		C' &
	\end{tikzcd}
	.\end{equation}
	Then there is a canonical exact sequence
	\begin{equation}
	\begin{tikzcd}
		\ker u \arrow[r, "\underline{\alpha}", rightarrow] &
		\ker v \arrow[r, "\underline{\beta}", rightarrow] &
		\ker w \arrow[r, "\delta", rightarrow] &
		\mathrm{coker}\, u \arrow[r, "\overline{\alpha'}", rightarrow] &
		\mathrm{coker}\, v \arrow[r, "\overline{\beta'}", rightarrow] &
		\mathrm{coker}\, w
	\end{tikzcd}
	,\end{equation} 
	where $\underline{\alpha}$ and $\underline{\beta}$ are the restrictions to the kernels,
	whereas $\overline{\alpha'}$ and $\overline{\beta'}$ the induced maps on the quotients.
\end{lem} 
\begin{proof}\leavevmode\vspace{-.2\baselineskip}
	\begin{enumerate}
		\item $\underline{\alpha}\colon \ker u \to \ker v$ is well defined,
			since $\alpha(x) \in \ker v$ for all $x \in \ker u$.
			In fact, since the diagram commutes, and $x \in \ker u$, we have
			\begin{equation}
				v \left( \alpha(x) \right) = \alpha' (u(x)) = 0
			.\end{equation} 
			Analogously for $\beta$. In fact, for any commutative diagram, the kernel
			of a vertical arrow gets mapped to the kernel of the following one, by the horizontal one.

		\item $\overline{\alpha'}\colon \mathrm{coker}\, u \to \mathrm{coker}\, v$ is well define.
			In fact consider $x' \in A'/\ima u$, then $\overline{\alpha'}(x') \in B'/\ima v$.
			Clearly $x' \in \ima u$ means that there exists $a \in A$ s.t. $u(a) = x'$.
			Then
			\begin{equation}
				\alpha'(x') = \alpha'(u(a)) = v(\alpha(a)) \in \ima v
			.\end{equation} 
			By the universal property of the quotient we obtain that $\overline{\alpha'}$ is well defined.
			Analogously one argues that $\overline{\beta'}$ is well defined.

		\item Let's define $\delta\colon \ker w \to \mathrm{coker}\, u$.
			Consider $x \in \ker w$, then $w(x) = 0$.
			By exactness $\beta$ is surjective, hence there exists $b \in B$
			s.t. $\beta(b) = x$.
			Then we obtain $v(b) \in B'$ s.t. 
			\begin{equation}
			\beta'(v(b)) = w(\beta(b)) = w(x) = 0
			,\end{equation} 
			i.e. $v(b) \in \ker \beta'$.
			Then, by exactness of the second row, there exists $a \in A'$
			s.t. $\alpha'(a) = b$.
			We define $\delta(x) := \overline{a} \in \mathrm{coker}\, u$.
			Notice that $\alpha'$ is injective, hence we can identify $A' \simeq \ima \alpha'$.

			Let's show that this is a good definition, i.e. $a$ is defined up to $\ima u$:
			Consider $b_1, b_2 \in B$ s.t. $\beta(b_1) = \beta(b_2) = x \in \ker w$.
			Then, by commutativity and exacntess $v(b_1), v(b_2) \in \ima \alpha'$.
			In particular $\beta(b_1 - b_2) = x - x = 0$.
			By exactness there exists $a \in A$ s.t. $\alpha(a) = b_1 - b_2$.
			Then by commutativity of the diagram
			\begin{equation}
				v(b_1 - b_2) = v ( \alpha(a)) = \alpha'(u(a))
			.\end{equation} 
			In particular, being $\alpha'$ injective, $v(b_1)$ and
			$v(b_2)$ differ by an element in $\ima u$, i.e. $\delta$ is well defined.

		\item The sequence is exact:

			Let's start by checking exactness at $\ker v$.
			Clearly $\underline{\beta} \circ \underline{\alpha} = 0$, they are just the restrictions.
			Let $x \in \ker \underline{\beta}$,
			then there exists $a \in A$ s.t. $\alpha(a) = x$.
			Since $x \in \ker v$, $0 = v(x) = v(\alpha(a)) = \alpha'(u(a))$,
			By injectivity of $\alpha'$ we obtain that $u(a) = 0$, hence $a \in \ker u$.
			In other words $x \in \ima \underline{\alpha}$.

			Let's now check exactness at $\mathrm{coker}\, v$.
			Again the composition $\overline{\beta'} \circ \overline{\alpha'}$ is trivial,
			since it is induced by the trivial composition
			$\beta' \circ \alpha' = 0$.
			Consider now $\overline{b'} \in \ker \overline{\beta'}$, i.e.
			\begin{equation}
				\overline{\beta'}\left( \overline{b'} \right) =
				\overline{\beta'(b')} = 0 \in \mathrm{coker}\, w
			.\end{equation} 
			In other words $\beta'(b') \in \ima w$, i.e. there exists
			$c \in C$ s.t. $w(c) = \beta'(b')$.
			By surjectivity of $\beta$ there is $b \in B$ s.t. $\beta(b) = c$.
			Then
			\begin{equation}
				\beta'(v(b)) = w(c) = \beta'(b')
			.\end{equation} 
			Then $b' - v(b) \in \ker \beta' = \ima \alpha'$, i.e.
			there exists $a' \in A'$ s.t. $\alpha'(a') = b' - v(b)$.
			Then
			\begin{equation}
				\overline{\alpha'}(\overline{a'} =
				\overline{\alpha'(a')} = \overline{b' - v(b)} = \overline{b'}
				\in \mathrm{coker}\, v
			.\end{equation} 
			Then $\overline{b'} \in \ima \overline{\alpha'}$.

			Exactness at $\ker w$:
			Let $c \in \ker \delta$, i.e. $\delta(c) = 0$.
			By definition $\delta(c) = \alpha'^{-1}(v(b)) =: a'$ in the quotient.
			In particular it is zero iff $a' \in \ima (u)$, i.e. $a' = u(a)$
			for $a \in A$.
			Then $v(\alpha(a)) = v(b)$, i.e. $\alpha(a) - b \in \ker v$.
			Then $\underline{\beta}(b - \alpha(a)) = \beta(b) - \beta(\alpha(a)) = \beta(b) = c$.

			$\delta \circ \underline{\beta} = 0$: in fact $x \in \ker v$ clearly implies
			$v(x) = 0$, and $\delta(\underline{\beta}(x)) = 0$ by definition of $\delta$.

			Exactness at $\mathrm{coker}\, u$:
			Let $\overline{a'} \in \ker \overline{\alpha'}$, then
			$\alpha'(a') \in \ima v$, i.e. there is $b \in B$ s.t. $v(b) = \alpha'(a')$.
			By commutativity of the diagram (and exactness) we obtain that $\beta(b) \in \ker w$,
			since
			\begin{equation}
				w(\beta(b)) = \beta'(v(b)) = \beta'(\alpha'(a')) = 0
			.\end{equation} 
			And $\beta(b)$ gets mapped, by $\delta$, to $\overline{a'}$.
			In fact $\delta(\beta(b)) = \overline{v(b)} = \overline{a'}$.\qedhere
	\end{enumerate}
\end{proof}

\begin{rem}[]
	If $\alpha$ is injective and $\beta'$ is surjective, and we have a morphism of exact sequences
	\begin{equation}
	\begin{tikzcd}
		0 \arrow[r, "", rightarrow] &
		A \arrow[r, "", rightarrow] \arrow[d, "u", rightarrow] &
		B \arrow[r, "", rightarrow] \arrow[d, "v", rightarrow] &
		C \arrow[r, "", rightarrow] \arrow[d, "w", rightarrow] &
		0 \\
		0 \arrow[r, "", rightarrow] &
		A' \arrow[r, "", rightarrow] &
		B' \arrow[r, "", rightarrow] &
		C' \arrow[r, "", rightarrow] &
		0
	\end{tikzcd}
	,\end{equation} 
	then we obtain the exact sequence:
	\begin{equation}
	\begin{tikzcd}[column sep=1.5em]
		0 \arrow[r, "", rightarrow] &
		\ker u \arrow[r, "\underline{\alpha}", rightarrow] &
		\ker v \arrow[r, "\underline{\beta}", rightarrow] &
		\ker w \arrow[r, "\delta", rightarrow] &
		\mathrm{coker}\, u \arrow[r, "\overline{\alpha'}", rightarrow] &
		\mathrm{coker}\, v \arrow[r, "\overline{\beta'}", rightarrow] &
		\mathrm{coker}\, w \arrow[r, "", rightarrow] &
		0
	\end{tikzcd}
	,\end{equation} 
\end{rem}

\begin{thm}[Freyd-Mitchell]
	Let $\mathsf{A}$ be a small abelian category.
	Then there exists a ring $R$ and a fully faithful exact functor
	\begin{align}
		\iota\colon \mathsf{A} &\longrightarrow R\text{-}\mathsf{Mod}
	.\end{align} 
\end{thm}

\begin{rem}[]
	Notice that any fully faithful and exact functor reflects exactness.
	This allows us to check exactness of sequences in any abelian category
	by using diagram chasing in an appropriate category of modules.
\end{rem}

\begin{cor}
	Snake lemma holds also for an arbitrary abelian category $\mathsf{A}$.
\end{cor} 
\begin{proof}
	One needs to apply Freyd-Mitchell to the full subcategory of $\mathsf{A}$
	given by the only objects of $\mathsf{A}$ appearing in the snake lemma diagram.

	By exactness of $\iota$ the image of the diagram satisfies the hypothesis for snake
	lemma in $R\text{-}\mathsf{Mod}$, hence we can apply it, 
	obtain our result, and come back to $\mathsf{A}$.
	Notice that we need to recall the last remark in order to come back
	to $\mathsf{A}$.
\end{proof}

\end{document}
